% ==============================================================================
% الملخص
% ==============================================================================
\newpage
\thispagestyle{empty}

\begin{center}
{\Huge \textbf{ملخص الدراسة}}
\end{center}

\vspace{1.5cm}

\begin{flushright}
\begin{minipage}{0.95\textwidth}

تهدف هذه الدراسة إلى تسليط الضوء على التحديات التي تواجهها وكالات الأسفار التقليدية في ظل المنافسة المتزايدة مع المنصات الإلكترونية العالمية لحجز السفر. وقد شهد قطاع السياحة والسفر تحولات جذرية خلال العقدين الأخيرين بفعل الثورة الرقمية وانتشار الإنترنت، مما أدى إلى ظهور منصات إلكترونية عملاقة مثل بوكينغ وإكسبيديا وسكاي سكانر، التي أصبحت تستحوذ على حصة سوقية متنامية على حساب الوكالات التقليدية.

\vspace{0.8cm}

تنطلق الدراسة من ثلاث فرضيات أساسية: أولاً، أن المنصات الإلكترونية العالمية تسببت في انخفاض نسبة الحجوزات لدى وكالات الأسفار. ثانياً، أن وكالات الأسفار تواجه تحديات جوهرية بسبب ضعف مستوى الرقمنة مقارنة بالمنصات الإلكترونية. ثالثاً، أن التغيير في سلوك المستهلك نحو الرقمنة تسبب في تراجع الإقبال على وكالات الأسفار التقليدية.

\vspace{0.8cm}

تتكون الدراسة من ثلاثة فصول: يتناول الفصل الأول الإطار النظري لوكالات السفر والمنصات الإلكترونية من حيث التعريف والنشأة والأنواع والخدمات المقدمة. ويعالج الفصل الثاني طبيعة المنافسة بين الطرفين وتأثير المنصات الإلكترونية على أداء وكالات الأسفار. أما الفصل الثالث فيركز على التحديات الرئيسية التي تواجه الوكالات والحلول والاستراتيجيات المقترحة للتطوير والتكيف مع البيئة الرقمية الجديدة.

\vspace{0.8cm}

\textbf{الكلمات المفتاحية:} وكالات الأسفار، المنصات الإلكترونية، المنافسة الرقمية، التحول الرقمي، سلوك المستهلك، السياحة الإلكترونية، حجز السفر عبر الإنترنت.

\end{minipage}
\end{flushright}

\newpage
