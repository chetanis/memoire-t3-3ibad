% ==============================================================================
% ملخص المذكرة (حوالي 10 صفحات)
% تحديات وكالات الأسفار في ظل المنافسة مع المنصات الإلكترونية
% ==============================================================================
\documentclass[a4paper, 14pt]{extreport}

\usepackage{fontspec}
\usepackage{polyglossia}
\setmainlanguage[numerals=maghrib]{arabic}
\setotherlanguage{english}

\setmainfont{Amiri}[
  Path = /usr/local/texlive/2025basic/texmf-dist/fonts/truetype/public/amiri/,
  Extension = .ttf,
  UprightFont = *-Regular,
  BoldFont = *-Bold,
  ItalicFont = *-Italic,
  BoldItalicFont = *-BoldItalic,
  Script = Arabic
]
\newfontfamily\arabicfont{Amiri}[
  Path = /usr/local/texlive/2025basic/texmf-dist/fonts/truetype/public/amiri/,
  Extension = .ttf,
  UprightFont = *-Regular,
  BoldFont = *-Bold,
  ItalicFont = *-Italic,
  BoldItalicFont = *-BoldItalic,
  Script = Arabic,
  Scale = 1.1
]
\newfontfamily\arabicfontsf{Amiri}[
  Path = /usr/local/texlive/2025basic/texmf-dist/fonts/truetype/public/amiri/,
  Extension = .ttf,
  UprightFont = *-Regular,
  BoldFont = *-Bold,
  ItalicFont = *-Italic,
  BoldItalicFont = *-BoldItalic,
  Script = Arabic,
  Scale = 1.1
]

\usepackage[top=2.5cm, bottom=2.5cm, right=3cm, left=2cm]{geometry}
\usepackage{setspace}
\onehalfspacing

\usepackage{enumitem}
\usepackage{titlesec}
\usepackage{hyperref}

\hypersetup{
  colorlinks=true,
  linkcolor=black,
  citecolor=blue!60!black,
  urlcolor=blue!60!black,
}

\titleformat{\chapter}[display]
  {\normalfont\Huge\bfseries\centering}
  {\chaptertitlename\ \thechapter}{20pt}{\Huge}
\titlespacing*{\chapter}{0pt}{-20pt}{40pt}

\titleformat{\section}
  {\normalfont\Large\bfseries}{\thesection}{1em}{}
\titleformat{\subsection}
  {\normalfont\large\bfseries}{\thesubsection}{1em}{}

\makeatletter
\ifdefined\UseMathForPositioningText\else
  \newcommand{\UseMathForPositioningText}{}
\fi
\ifdefined\@kernel@tabular@init\else
  \newcommand{\@kernel@tabular@init}{}
\fi
\makeatother

\begin{document}

% ==============================================================================
\begin{center}
{\Huge \textbf{ملخص شامل للمذكرة}}\\[0.8cm]
{\LARGE \textbf{تحديات وكالات الأسفار في ظل المنافسة مع المنصات الإلكترونية}}\\[1.5cm]
\end{center}

% ==============================================================================
% المقدمة
% ==============================================================================
\section*{أولاً: المقدمة العامة والإطار المنهجي}
\addcontentsline{toc}{section}{أولاً: المقدمة العامة والإطار المنهجي}

يُعدّ قطاع السياحة والسفر من أهم القطاعات الاقتصادية عالمياً، إذ يُسهم بنسبة كبيرة في الناتج المحلي الإجمالي ويوفر ملايين فرص العمل. وقد شكّلت وكالات الأسفار تاريخياً الحلقة الأساسية في سلسلة التوزيع السياحي بوصفها الوسيط الرئيسي بين مقدمي الخدمات السياحية والمسافرين. غير أن الثورة الرقمية وانتشار الإنترنت أحدثا تحولات جذرية في بنية هذا القطاع، حيث ظهرت منصات إلكترونية عملاقة مثل بوكينغ دوت كوم وإكسبيديا وسكاي سكانر وإير بي إن بي، استحوذت على حصص سوقية متنامية وشكّلت تهديداً حقيقياً لاستمرارية الوكالات التقليدية.

تتمحور إشكالية هذه الدراسة حول السؤال الرئيسي: \textbf{ما هي أبرز التحديات التي تواجهها وكالات الأسفار في ظل المنافسة المتزايدة مع المنصات الإلكترونية، وما هي الحلول والاستراتيجيات الممكنة للتكيف مع هذا الواقع الجديد؟}

وقد صيغت ثلاث فرضيات للإجابة على هذه الإشكالية: الأولى أن المنصات الإلكترونية تسببت في انخفاض الحجوزات لدى الوكالات، والثانية أن ضعف الرقمنة لدى الوكالات يُضعف تنافسيتها، والثالثة أن تغير سلوك المستهلك نحو الرقمنة أدى إلى تراجع الإقبال على الوكالات التقليدية.

اعتمدت الدراسة على المنهج الوصفي التحليلي، مستندة إلى المسح المكتبي والتوثيقي وتحليل التقارير الدولية ودراسات الحالة، وتغطي الفترة من بداية الألفية الثالثة (2000) حتى الوقت الحاضر، مع تركيز على التطورات بعد جائحة كوفيد-19.

% ==============================================================================
% الفصل الأول
% ==============================================================================
\section*{ثانياً: الإطار النظري لوكالات السفر والمنصات الإلكترونية}
\addcontentsline{toc}{section}{ثانياً: الإطار النظري لوكالات السفر والمنصات الإلكترونية}

\subsection*{1. تعريف وكالات الأسفار ونشأتها وأنواعها}

تُعرّف وكالة الأسفار بأنها مؤسسة تجارية ذات طابع خدمي تعمل كوسيط بين مقدمي الخدمات السياحية (شركات الطيران، الفنادق، شركات النقل) والمسافرين، وتقدم مجموعة متنوعة من الخدمات المتعلقة بالسفر مقابل عمولة أو رسوم خدمة، وتخضع لترخيص من الجهات المختصة.

تعود جذور صناعة السفر المنظم إلى القرن التاسع عشر، حيث يُعتبر البريطاني توماس كوك رائد هذا المجال بتنظيمه أول رحلة جماعية عام 1841 وإنشائه أول وكالة سفر تجارية عام 1845. وقد مرّ تطور الوكالات بعدة مراحل: مرحلة التأسيس في القرن التاسع عشر، ومرحلة النمو في النصف الأول من القرن العشرين مع ظهور شركات مثل أمريكان إكسبريس، ثم مرحلة الازدهار (1950-1990) التي شهدت السياحة الجماهيرية وتطور أنظمة الحجز المحوسبة مثل سيبر وأماديوس، وأخيراً مرحلة التحول الرقمي من التسعينيات حتى اليوم.

تتنوع وكالات الأسفار وفق عدة معايير: حسب النشاط (وكالات تجزئة، منظمو رحلات، وكالات جملة)، وحسب التخصص (عامة أو متخصصة في سياحة الأعمال أو الدينية أو المغامرات أو الفاخرة)، وحسب نمط التشغيل (مستقلة، سلاسل، امتياز)، وحسب الوسيلة (تقليدية بمكاتب فعلية، إلكترونية عبر الإنترنت، أو هجينة تجمع بين الاثنين).

\subsection*{2. الخدمات التي تقدمها وكالات الأسفار}

تقدم وكالات الأسفار مجموعة واسعة من الخدمات تشمل: خدمات النقل والحجز (تذاكر الطيران باستخدام أنظمة GDS، القطارات، تأجير السيارات)، وخدمات الإقامة (فنادق بمختلف فئاتها، منتجعات، شقق مفروشة)، وتنظيم الرحلات (باقات شاملة ورحلات مخصصة حسب الطلب)، والخدمات الإدارية (التأشيرات، التأمين على السفر)، والاستشارات (نصائح حول الوجهات والأوقات المناسبة). كما تتخصص بعض الوكالات في سياحة الأعمال (تنظيم المؤتمرات والمعارض) وخدمات الحج والعمرة في العالم الإسلامي.

وتمثل الخدمة الاستشارية المبنية على الخبرة الشخصية والمعرفة الميدانية إحدى أهم نقاط القوة التي تتميز بها الوكالات التقليدية مقارنة بالمنصات الإلكترونية.

\subsection*{3. المنصات الإلكترونية: التعريف والأنواع والمزايا}

المنصات الإلكترونية للسفر هي بيئات رقمية تفاعلية تجمع بين مقدمي خدمات السفر والمسافرين، وتوفر أدوات للبحث والمقارنة والحجز والدفع والتقييم على مدار الساعة. وتتنوع هذه المنصات إلى عدة أنواع: وكالات السفر عبر الإنترنت (OTAs) مثل بوكينغ (أكثر من 28 مليون وحدة إقامة في 220 دولة) وإكسبيديا وتريب دوت كوم؛ ومحركات البحث عن السفر (Meta-search) مثل سكاي سكانر وغوغل فلايتس وتريفاغو التي تقارن الأسعار من مصادر متعددة؛ ومنصات الاقتصاد التشاركي مثل إير بي إن بي (أكثر من 7 ملايين مسكن)؛ ومنصات التقييم مثل تريب أدفايزر (أكثر من مليار تقييم).

تتمتع المنصات الإلكترونية بمزايا عديدة جعلتها تستقطب أعداداً متزايدة من المسافرين: التوفر على مدار الساعة طوال أيام الأسبوع، والوصول من أي مكان عبر الهاتف المحمول (أكثر من 70\% من عمليات البحث تتم عبر الأجهزة المحمولة)، وسهولة واجهة المستخدم، ووفرة المعلومات والتقييمات (80\% من المسافرين يقرؤون التقييمات قبل الحجز)، والشفافية في الأسعار، والأسعار التنافسية (أقل بنسبة 10-30\% في كثير من الحالات)، والعروض والخصومات المستمرة، والتخصيص الذكي باستخدام الذكاء الاصطناعي.

تعتمد هذه المنصات على نماذج أعمال متنوعة: نموذج العمولة (15-25\% لحجوزات الفنادق)، ونموذج التاجر (شراء بالجملة وإعادة البيع)، ونموذج الإعلانات، ونموذج الاشتراك. ومن الاتجاهات الحديثة: استخدام الذكاء الاصطناعي وروبوتات المحادثة، والواقع الافتراضي، والتخصيص الفائق، والاهتمام بالاستدامة، وتوسيع خيارات الدفع الرقمي.


% ==============================================================================
% الفصل الثاني
% ==============================================================================
\section*{ثالثاً: المنافسة بين وكالات الأسفار والمنصات الإلكترونية}
\addcontentsline{toc}{section}{ثالثاً: المنافسة بين وكالات الأسفار والمنصات الإلكترونية}

\subsection*{1. طبيعة المنافسة وأبعادها}

تتسم المنافسة بين وكالات الأسفار والمنصات الإلكترونية بطبيعتها غير المتكافئة لصالح المنصات في معظم الأبعاد. ووفق نموذج بورتر للقوى التنافسية الخمس المطبّق على قطاع السفر: أدى التحول الرقمي إلى تخفيض حواجز الدخول أمام منافسين جدد؛ وتغيرت العلاقة مع الموردين حيث أصبحت شركات الطيران والفنادق تبيع مباشرة للمستهلكين؛ وازدادت القوة التفاوضية للمسافرين بفضل وصولهم السهل للمعلومات والأسعار؛ وتعددت البدائل المتاحة (حجز مباشر، تخطيط ذاتي، وسائل تواصل اجتماعي)؛ واشتدت المنافسة بين المتنافسين الحاليين بسبب تعددهم وتجانس خدماتهم.

تتعدد أبعاد هذه المنافسة: \textbf{البعد التكنولوجي} حيث تستثمر المنصات مبالغ ضخمة (مجموعة بوكينغ هولدينغز أنفقت أكثر من 5 مليارات دولار على التكنولوجيا في 2022) في البنية التحتية والذكاء الاصطناعي وتجربة المستخدم، بينما تعاني الوكالات من محدودية استثماراتها التكنولوجية. \textbf{البعد التسويقي} حيث تُنفق المنصات مبالغ هائلة على التسويق الرقمي (بوكينغ أنفقت أكثر من 6 مليارات دولار في 2022) وتهيمن على نتائج محركات البحث، بينما تمتلك الوكالات ميزانيات محدودة. \textbf{البعد السعري} حيث تستفيد المنصات من وفورات الحجم وانخفاض التكاليف التشغيلية والتسعير الديناميكي لتقديم أسعار أكثر تنافسية. و\textbf{البعد الجغرافي} حيث تغطي المنصات أكثر من 220 دولة بعشرات اللغات والعملات.

ومع ذلك، تحتفظ وكالات الأسفار بنقاط قوة مهمة: الخدمة الشخصية والعلاقة الإنسانية (خاصة للرحلات المعقدة والمسافرين كبار السن)، والخبرة والمعرفة المتخصصة الميدانية، والأمان والثقة المحلية، والحماية القانونية بموجب التشريعات المنظمة، وخدمة ما بعد البيع الشخصية والفعالة في أوقات الأزمات.

\subsection*{2. تأثير المنصات الإلكترونية على وكالات الأسفار}

\textbf{التأثير على الحصة السوقية:} ارتفعت حصة الحجوزات عبر الإنترنت من أقل من 5\% في عام 2000 إلى أكثر من 65\% في عام 2022 عالمياً (وتتجاوز 75\% في بعض الأسواق المتقدمة). وفي منطقة الشرق الأوسط وشمال إفريقيا تجاوزت 40\% في 2023. أما حصة الوكالات التقليدية فانخفضت من أكثر من 60\% إلى نحو 15\% فقط. وانخفض عدد الوكالات بشكل كبير: في أمريكا من 34,000 إلى 15,000 وكالة، وفي فرنسا تراجع بنحو 30\%. ومن أبرز الأحداث الرمزية إفلاس شركة توماس كوك العريقة في 2019 بعد 178 عاماً.

أصبح السوق يهيمن عليه عدد قليل من العمالقة: مجموعة بوكينغ هولدينغز (إيرادات تجاوزت 17 مليار دولار في 2022)، مجموعة إكسبيديا (أكثر من 12 مليار دولار)، ومجموعة تريب دوت كوم (أكثر من 3 مليارات دولار).

\textbf{التأثير على نموذج الأعمال:} شهدت الوكالات انهياراً في نظام العمولات التقليدي: خُفّضت عمولات شركات الطيران تدريجياً من 10\% إلى 8\% ثم 5\% ثم أُلغيت بالكامل في معظم الحالات. وانخفض متوسط هامش الربح الصافي من 8-10\% في التسعينيات إلى 2-4\% حالياً. مما اضطر الوكالات إلى البحث عن بدائل: فرض رسوم خدمة، والتركيز على خدمات ذات هامش ربح أعلى كالتأمين والرحلات المخصصة.

\textbf{التأثير على سلوك المستهلك:} تغيّر مسار رحلة العميل بالكامل، إذ يمر المسافر المعاصر بخمس مراحل (الإلهام عبر التواصل الاجتماعي، البحث عبر المحركات والمنصات، المقارنة عبر محركات المقارنة، الحجز الإلكتروني، المشاركة عبر التقييمات). ويزور المسافر المتوسط 38 موقعاً إلكترونياً قبل إتمام حجزه، وتغيب وكالات الأسفار عن معظم هذه المراحل. كما ظهرت ظاهرة ``ROBO'' (البحث في الوكالة والحجز عبر الإنترنت). وتظهر فجوة واضحة بين الأجيال: 25\% فقط من كبار السن يستخدمون المنصات الإلكترونية مقابل أكثر من 90\% من جيل Z.

\textbf{التأثير على سوق العمل:} انخفض عدد العاملين في وكالات الأسفار الأمريكية من 124,000 إلى 65,000 موظف، مع تغيّر جوهري في المهارات المطلوبة نحو المهارات الرقمية والاستشارية.

\textbf{تأثير جائحة كوفيد-19:} شكّلت الجائحة ضربة قاصمة للوكالات (إغلاقات، فقدان موظفين، تسريع التحول الرقمي)، لكنها أبرزت أيضاً قيمة الخدمة الشخصية في الأزمات، حيث عاد بعض المسافرين للوكالات بعد صعوبات في التعامل مع المنصات الإلكترونية أثناء فوضى الإلغاءات.

\textbf{التأثير على سلسلة القيمة:} حدثت ظاهرتان متوازيتان: ``إلغاء الوساطة'' حيث أصبح المسافرون والموردون يتعاملون مباشرة، و``إعادة الوساطة'' عبر استبدال الوسيط التقليدي بوسيط رقمي (المنصات الإلكترونية). والسيناريو الأكثر ترجيحاً للمستقبل هو ``النموذج الهجين'' الذي تتطور فيه الوكالات الناجحة نحو نموذج يجمع بين الخدمة الشخصية والتواجد الرقمي القوي.


% ==============================================================================
% الفصل الثالث
% ==============================================================================
\section*{رابعاً: التحديات والحلول المقترحة لوكالات السفر}
\addcontentsline{toc}{section}{رابعاً: التحديات والحلول المقترحة لوكالات السفر}

\subsection*{1. التحديات الرئيسية}

\textbf{أ. التحديات التكنولوجية:} تُعدّ الفجوة الرقمية من أخطر التحديات، حيث تفتقر نسبة كبيرة من الوكالات إلى مواقع إلكترونية احترافية وتطبيقات هاتف وأنظمة إدارة علاقات العملاء (CRM). وتتسارع وتيرة التطور التكنولوجي (الذكاء الاصطناعي، المساعدون الصوتيون، البلوك تشين، الواقع الافتراضي) بما يصعب مواكبته. كما يشكّل الأمن السيبراني تحدياً إضافياً في ظل التعامل مع بيانات حساسة.

\textbf{ب. التحديات الاقتصادية والمالية:} يتراجع الدخل بسبب إلغاء العمولات والمنافسة السعرية الشديدة وتحوّل الحجوزات نحو المنصات. وترتفع التكاليف التشغيلية (إيجارات، رواتب، تكنولوجيا، تسويق). وتواجه الوكالات صعوبة في الحصول على تمويل للتطوير بسبب حذر المؤسسات المالية تجاه القطاع.

\textbf{ج. التحديات المرتبطة بسلوك المستهلك:} أصبح المسافر الرقمي أكثر استقلالية ومهارة في البحث الذاتي، ويثق بالمراجعات الإلكترونية ويفضل التخصيص. كما أن الأجيال الجديدة (جيل الألفية وجيل Z) التي تمثل أكثر من 50\% من المسافرين عالمياً تفضل الحجز الرقمي عبر الهاتف وتبحث عن تجارب فريدة وأصيلة. ويتزايد التوجه نحو ``اقتصاد التجربة'' (الإقامة في منازل محلية، التجارب الثقافية) بعيداً عن الباقات المعلّبة.

\textbf{د. التحديات التنظيمية:} يوجد تفاوت في الإطار التنظيمي حيث تخضع الوكالات لشروط صارمة بينما تعمل المنصات العابرة للحدود في فراغ تنظيمي نسبي. كما يضعف الدعم المؤسسي الموجه لمساعدة الوكالات على التكيف، وتتزايد المنافسة من داخل القطاع (بيع مباشر من شركات الطيران والفنادق، دخول لاعبين جدد من خارج القطاع).

\subsection*{2. الحلول والاستراتيجيات المقترحة}

\textbf{أ. استراتيجية التحول الرقمي:} تُعدّ الأكثر إلحاحاً، وتتضمن: بناء منصة رقمية متكاملة (موقع إلكتروني متجاوب بنظام حجز ودفع، تطبيق للهاتف المحمول، تكامل مع أنظمة الحجز العالمية عبر APIs)؛ وتبني أنظمة إدارة علاقات العملاء (CRM) لتتبع تفاعلات العملاء وتخصيص العروض؛ والاستفادة من أدوات الذكاء الاصطناعي المتاحة (روبوتات محادثة، تحليل بيانات، أنظمة توصيات، تسعير ديناميكي).

\textbf{ب. استراتيجية التمايز والتخصص:} بدلاً من المنافسة العامة مع المنصات العملاقة، يُنصح بالتركيز على أسواق تتطلب معرفة عميقة وخدمة شخصية: السياحة الفاخرة (سوق يقدّر بأكثر من 900 مليار دولار)، سياحة المغامرات، السياحة الصحية، سياحة الأعمال والمؤتمرات، السياحة الدينية (الحج والعمرة)، سياحة الزفاف وشهر العسل، والسفر لذوي الاحتياجات الخاصة. ويمكن التمايز أيضاً من خلال تقديم تجارب فريدة حصرية وبناء علامة تجارية قائمة على الخبرة والمصداقية والعلاقة الإنسانية.

\textbf{ج. استراتيجية التسويق الرقمي:} تشمل: تحسين محركات البحث (SEO) باستهداف كلمات مفتاحية متخصصة؛ والتسويق عبر وسائل التواصل الاجتماعي (إنستغرام للصور الملهمة، فيسبوك لبناء المجتمع، يوتيوب للفيديو، تيك توك للأجيال الشابة)؛ والتسويق بالمحتوى (مقالات، دلائل سفر، قصص، بودكاست)؛ والتسويق عبر البريد الإلكتروني بعروض مخصصة.

\textbf{د. استراتيجية تحسين تجربة العميل:} التحول من نموذج ``الوسيط'' إلى نموذج ``مستشار السفر'' الذي يقدم استشارات معمّقة وعلاقات طويلة الأمد ومتابعة شاملة. وتحسين خدمة ما بعد البيع (خط تواصل مباشر أثناء الرحلة، تدخل فوري لحل المشاكل). وإنشاء برامج ولاء (نقاط، خصومات حصرية، معاملة تفضيلية).

\textbf{هـ. استراتيجية الشراكات والتحالفات:} تشمل: شراكات بين وكالات صغيرة لتعزيز القوة التفاوضية ومشاركة تكاليف التكنولوجيا والتسويق؛ وشراكات مع مقدمي خدمات للحصول على أسعار حصرية ومنتجات غير متاحة عبر المنصات؛ والتعاون مع المنصات الإلكترونية نفسها كقناة إضافية للتوزيع بدلاً من اعتبارها عدواً.

\textbf{و. استراتيجية تطوير الموارد البشرية:} الاستثمار في تكوين الموظفين على المهارات الرقمية والاستشارية واللغات؛ واستقطاب كفاءات جديدة في التكنولوجيا والتسويق الرقمي؛ وتحسين بيئة العمل برواتب تنافسية ورحلات تعريفية وثقافة إيجابية.

\textbf{ز. استراتيجية تنويع مصادر الدخل:} تطوير خدمات ذات قيمة مضافة عالية (استشارات مدفوعة، خدمات كونسيرج، فعاليات حصرية، إدارة سفر الشركات)؛ واستحداث مصادر دخل جديدة (تأمينات بهوامش ربح عالية، رسوم عضوية، محتوى سياحي مموّل).

\subsection*{3. نماذج ناجحة وإطار عمل مقترح}

أثبتت عدة نماذج نجاحها في التكيف: \textbf{النموذج الهجين} الذي يجمع بين التواجد المادي والرقمي؛ و\textbf{نموذج التخصص العميق} في أسواق دقيقة؛ و\textbf{نموذج مستشار السفر المستقل} الذي يعمل تحت مظلة شبكة أكبر بتكاليف منخفضة؛ و\textbf{نموذج الوكالة المتخصصة في السوق المحلي} التي تستفيد من معرفتها الثقافية والعلاقاتية.

تم اقتراح إطار عمل من ست مراحل للتحول الاستراتيجي: (1) التشخيص والتقييم (تحليل SWOT، تقييم الجاهزية الرقمية)، (2) بناء الرؤية والاستراتيجية (تحديد عرض القيمة الفريد)، (3) البناء التكنولوجي (موقع، CRM، APIs)، (4) تطوير الكفاءات (تدريب رقمي واستشاري)، (5) الإطلاق والتنفيذ (حملات تسويقية، مراقبة الأداء)، (6) التقييم والتحسين المستمر (مراجعة دورية لمؤشرات الأداء KPIs).

كما تمثل \textbf{السياحة المستدامة} فرصة استراتيجية حيث يعبّر 73\% من المسافرين عن رغبتهم في السفر بشكل أكثر استدامة، ويمكن للوكالات تقديم خيارات مسؤولة بيئياً واجتماعياً واقتصادياً لا تركز عليها المنصات الإلكترونية.


% ==============================================================================
% الخاتمة والنتائج
% ==============================================================================
\section*{خامساً: النتائج والتوصيات}
\addcontentsline{toc}{section}{خامساً: النتائج والتوصيات}

\subsection*{1. أبرز نتائج الدراسة}

\begin{enumerate}[label=\textbf{\arabic*.}]
\item وكالات الأسفار عرفت تاريخاً طويلاً من التطور، لكنها تعيش اليوم أصعب مراحلها بسبب التحول الرقمي.
\item المنصات الإلكترونية تتمتع بمزايا تنافسية قوية في السعر والسهولة والتوفر والتكنولوجيا.
\item المنافسة متعددة الأبعاد وغير متكافئة لصالح المنصات في معظم الجوانب.
\item الحصة السوقية للوكالات التقليدية تراجعت من أكثر من 60\% إلى أقل من 15\%.
\item انهار نظام العمولات التقليدي وانخفضت هوامش الربح بشكل حاد.
\item تغيّر سلوك المستهلك السياحي جذرياً نحو الاستقلالية والرقمنة.
\item تحتفظ الوكالات بنقاط قوة مهمة في الخدمة الشخصية والخبرة والثقة المحلية.
\item التحديات متعددة ومتشابكة (تكنولوجية، اقتصادية، سلوكية، تنظيمية) لكنها ليست قدراً محتوماً.
\item يوجد حزمة متكاملة من الاستراتيجيات يمكن أن تمكّن الوكالات من التكيف والاستمرار.
\end{enumerate}

\subsection*{2. مناقشة الفرضيات}

\textbf{الفرضية الأولى -- تم تأكيدها:} البيانات والإحصائيات تثبت تراجعاً مستمراً في حجوزات وكالات الأسفار لصالح المنصات الإلكترونية بفضل مزاياها في السعر والسهولة والتوفر.

\textbf{الفرضية الثانية -- تم تأكيدها:} الفجوة الرقمية تُعدّ من أخطر التحديات، وغياب التواجد الرقمي الفعال يُضعف بشكل مباشر تنافسية الوكالات في سوق أصبح رقمياً بالدرجة الأولى.

\textbf{الفرضية الثالثة -- تم تأكيدها:} تحول سلوك المستهلك نحو الرقمنة، خاصة لدى الأجيال الجديدة (أكثر من 50\% من المسافرين)، يُعدّ عاملاً رئيسياً في تراجع الإقبال على الوكالات التقليدية.

\subsection*{3. أبرز التوصيات}

\textbf{لوكالات الأسفار:} الإسراع في التحول الرقمي كأولوية قصوى؛ التخصص والتمايز بدلاً من المنافسة الشاملة؛ الاستثمار في الكفاءات البشرية؛ بناء شراكات وتحالفات؛ وتنويع مصادر الدخل.

\textbf{للجهات الحكومية:} تحديث الإطار التنظيمي ليتواءم مع الواقع الرقمي؛ ضمان المنافسة العادلة بإخضاع المنصات الإلكترونية لنفس المتطلبات؛ إطلاق برامج دعم التحول الرقمي للوكالات الصغيرة والمتوسطة؛ ودعم التكوين المهني في المجالات الرقمية.

\subsection*{4. آفاق مستقبلية}

تفتح الدراسة المجال أمام عدة محاور بحثية: دراسة ميدانية حول التحول الرقمي في وكالات الأسفار الجزائرية أو العربية؛ تأثير الذكاء الاصطناعي على مستقبل الوكالات؛ تحليل سلوك المستهلك العربي تجاه المنصات الإلكترونية؛ دراسة مقارنة لاستراتيجيات التكيف في دول مختلفة؛ تقييم فعالية برامج التحول الرقمي؛ ودراسة تأثير تقنيات الويب 3.0 والبلوك تشين على صناعة توزيع السفر.

\vspace{1cm}

\begin{center}
\rule{0.5\textwidth}{0.5pt}\\[0.5cm]
\textbf{والمفتاح الأساسي لنجاح وكالات الأسفار في المستقبل هو القدرة على الجمع بين ما تتميز به من خدمة شخصية وخبرة وثقة، وبين ما تفرضه البيئة الرقمية من متطلبات التواجد الرقمي والسرعة والشفافية والتكنولوجيا.}
\end{center}

\end{document}
