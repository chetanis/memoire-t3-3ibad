% ==============================================================================
% الفصل الأول: الإطار النظري لوكالات السفر والمنصات الإلكترونية
% ==============================================================================
\chapter{الإطار النظري لوكالات السفر والمنصات الإلكترونية}

\section*{تمهيد}

يُعتبر قطاع السياحة والسفر من أكثر القطاعات حيوية وديناميكية في الاقتصاد العالمي، حيث يشهد تطورات مستمرة ومتسارعة تمس مختلف جوانبه ومكوناته. ومن أبرز الفاعلين في هذا القطاع نجد وكالات الأسفار التي لعبت دوراً محورياً في تنظيم وتسهيل عملية السفر والسياحة على مدى عقود طويلة. غير أن ظهور المنصات الإلكترونية أحدث تحولاً جذرياً في المشهد السياحي العالمي، مما يستوجب فهماً عميقاً لكلا الطرفين.

يهدف هذا الفصل إلى تقديم إطار نظري شامل حول وكالات الأسفار والمنصات الإلكترونية، وذلك من خلال ثلاثة مباحث: يتناول المبحث الأول تعريف وكالات الأسفار ونشأتها وأنواعها المختلفة، بينما يستعرض المبحث الثاني الخدمات التي تقدمها هذه الوكالات، أما المبحث الثالث فيركز على المنصات الإلكترونية ومزاياها في سوق السفر والسياحة.

% ======================================================================
% المبحث الأول: تعريف وكالات الأسفار، نشأتها، وأنواعها
% ======================================================================
\section{المبحث الأول: تعريف وكالات الأسفار، نشأتها، وأنواعها}

\subsection{تعريف وكالات الأسفار}

\subsubsection{المفهوم اللغوي والاصطلاحي}

تُعرّف وكالة الأسفار لغوياً بأنها المؤسسة أو الهيئة التي تتولى تنظيم وترتيب شؤون السفر والتنقل نيابة عن المسافرين. أما اصطلاحاً، فقد تعددت التعريفات التي قدمها الباحثون والمنظمات الدولية لوكالات الأسفار، ويمكن استعراض أبرزها على النحو التالي:

يُعرّف كوبر \parencite{cooper2018} وكالة الأسفار بأنها ``مؤسسة تجارية تعمل كوسيط بين مقدمي الخدمات السياحية والمسافرين، حيث تتولى بيع وتسويق المنتجات السياحية مقابل عمولة أو رسوم خدمة''. وهذا التعريف يبرز الدور الوسيط الذي تلعبه الوكالة في سلسلة القيمة السياحية.

أما منظمة السياحة العالمية \parencite{unwto2023} فتُعرّفها بأنها ``منشأة تجارية مرخصة تقدم خدمات تتعلق بتنظيم الرحلات والسفريات وحجز تذاكر النقل والإقامة وتأشيرات الدخول والتأمين على السفر وغيرها من الخدمات المرتبطة بالسفر والسياحة''. ويتميز هذا التعريف بشموليته في تحديد نطاق الخدمات التي تقدمها الوكالة.

من جهته، يُعرّف ميدلتون وفايال \parencite{middleton2009} وكالة الأسفار بأنها ``قناة توزيع رئيسية في صناعة السياحة، تربط بين العرض والطلب السياحيين من خلال تقديم مجموعة متنوعة من المنتجات والخدمات السياحية للمستهلكين النهائيين''. ويركز هذا التعريف على البعد التسويقي والتوزيعي لنشاط الوكالة.

وفي السياق العربي، يُعرّف الشمري \parencite{alshammari2018} وكالة الأسفار بأنها ``شركة تجارية متخصصة في تقديم خدمات السفر والسياحة، تعمل كحلقة وصل بين المسافر ومقدمي الخدمات السياحية، وتسعى إلى تلبية احتياجات ورغبات العملاء في مجال السفر والترفيه''.

ومن خلال استقراء التعريفات السابقة، يمكن استخلاص مجموعة من الخصائص المشتركة التي تميز وكالات الأسفار:

\begin{itemize}[label=\textbf{--}]
\item أنها مؤسسة تجارية ذات طابع خدمي.
\item تعمل كوسيط بين مقدمي الخدمات السياحية والمسافرين.
\item تقدم مجموعة متنوعة من الخدمات المتعلقة بالسفر والسياحة.
\item تحصل على مقابل مادي (عمولة أو رسوم) نظير خدماتها.
\item تخضع لتنظيم وترخيص من الجهات المختصة.
\end{itemize}

\subsubsection{التعريف القانوني لوكالات الأسفار}

تحرص معظم الدول على تنظيم نشاط وكالات الأسفار من خلال إطار قانوني وتشريعي يحدد شروط ممارسة هذا النشاط وحقوق والتزامات الأطراف المعنية. ففي الجزائر مثلاً، يُنظّم القانون رقم 99-06 المتعلق بالسياحة نشاط وكالات الأسفار والسياحة، حيث يُعرّفها بأنها ``كل شخص طبيعي أو معنوي يمارس بصفة دائمة نشاطاً يتعلق بتنظيم الرحلات والأسفار وبيع تذاكر النقل وحجز أماكن الإقامة وتقديم خدمات سياحية أخرى''.

وتشترط معظم التشريعات لممارسة نشاط وكالة الأسفار الحصول على ترخيص أو اعتماد من الجهة المختصة (عادة وزارة السياحة)، وتقديم ضمان مالي، وتوفر مؤهلات مهنية لدى المسيّرين، والتأمين على المسؤولية المهنية. وتهدف هذه الشروط إلى حماية المستهلك وضمان جودة الخدمات المقدمة.

\subsection{نشأة وتطور وكالات الأسفار}

\subsubsection{الجذور التاريخية لصناعة السفر}

تعود جذور صناعة السفر المنظم إلى حقب تاريخية بعيدة، حيث عرفت الحضارات القديمة أشكالاً مختلفة من الأسفار والرحلات. فقد كان الإغريق والرومان يسافرون لأغراض تجارية ودينية وعلاجية وترفيهية. كما عرف العالم الإسلامي رحلات استكشافية مهمة مثل رحلات ابن بطوطة في القرن الرابع عشر الميلادي. غير أن صناعة السفر بمفهومها الحديث لم تبدأ في التبلور إلا في القرن التاسع عشر.

\subsubsection{مرحلة التأسيس (القرن التاسع عشر)}

يُعتبر البريطاني توماس كوك (Thomas Cook) رائد صناعة السفر المنظم في العصر الحديث. ففي عام 1841م، نظّم كوك أول رحلة جماعية بالقطار من مدينة ليستر إلى مدينة لوبورو في إنجلترا لحضور تجمع مناهض للكحول، وشارك فيها نحو 570 شخصاً. وعلى الرغم من أن هذه الرحلة لم تكن تجارية بالمعنى الدقيق، إلا أنها شكّلت البذرة الأولى لنشاط وكالات الأسفار.

في عام 1845م، أنشأ كوك أول وكالة سفر تجارية في التاريخ، وبدأ في تنظيم رحلات سياحية بشكل احترافي. وفي عام 1855م، نظّم أول رحلة دولية من إنجلترا إلى فرنسا بمناسبة المعرض العالمي في باريس. ثم توسعت أعمال كوك لتشمل رحلات إلى مصر وفلسطين والولايات المتحدة الأمريكية. كما ابتكر كوك نظام القسائم الفندقية (Hotel Vouchers) الذي يُعدّ من أوائل أدوات الدفع المسبق في صناعة السياحة \parencite{cooper2018}.

\subsubsection{مرحلة النمو والتوسع (النصف الأول من القرن العشرين)}

شهدت بداية القرن العشرين توسعاً ملحوظاً في نشاط وكالات الأسفار، مدفوعاً بعدة عوامل من أبرزها: تطور وسائل النقل (السفن البخارية، السكك الحديدية، وبداية الطيران التجاري)، وتحسن المستوى المعيشي في الدول الصناعية، وتزايد الوعي بأهمية السياحة والترفيه. وقد ظهرت في هذه المرحلة شركات سفر كبرى مثل أمريكان إكسبريس (American Express) التي بدأت نشاطها في مجال السفر عام 1915م.

غير أن الحربين العالميتين الأولى والثانية شكّلتا عائقاً كبيراً أمام نمو هذا القطاع، حيث توقفت معظم الأنشطة السياحية خلال فترتي الحرب. ومع ذلك، أسهمت الحروب بشكل غير مباشر في تطوير البنية التحتية للنقل، خاصة الطيران، مما مهّد الطريق للطفرة السياحية في النصف الثاني من القرن العشرين.

\subsubsection{مرحلة الازدهار (1950-1990)}

تُعدّ الفترة الممتدة من خمسينيات إلى تسعينيات القرن العشرين العصر الذهبي لوكالات الأسفار. فقد شهدت هذه المرحلة ظهور السياحة الجماهيرية (Mass Tourism) بفضل تطور الطيران التجاري وانخفاض تكاليفه النسبية، وتحسن مستويات الدخل في الدول المتقدمة، وتزايد أوقات الفراغ نتيجة تقنين ساعات العمل والإجازات المدفوعة.

في هذه المرحلة، أصبحت وكالات الأسفار الوسيط الأساسي والإلزامي تقريباً بين شركات الطيران والفنادق من جهة والمسافرين من جهة أخرى. وقد تطورت أنظمة الحجز المحوسبة (Computer Reservation Systems - CRS) في السبعينيات والثمانينيات، مثل نظام سيبر (Sabre) ونظام أماديوس (Amadeus) ونظام غاليليو (Galileo)، مما ساهم في تحسين كفاءة عمل الوكالات وسرعة تقديم الخدمات \parencite{kracht2010}.

وقد شهدت هذه الفترة أيضاً ظهور مفهوم الرحلات الشاملة (Package Tours) التي تجمع بين النقل والإقامة والبرنامج السياحي في باقة واحدة بسعر موحد، وهو ما أسهم في تعزيز مكانة وكالات الأسفار كمنظم رئيسي للرحلات.

\subsubsection{مرحلة التحول الرقمي (من التسعينيات إلى اليوم)}

مع انتشار شبكة الإنترنت في التسعينيات، بدأت ملامح تحول جذري في صناعة السفر والسياحة. ففي عام 1996م، أُنشئت منصة إكسبيديا (Expedia) كواحدة من أوائل وكالات السفر عبر الإنترنت (Online Travel Agencies - OTAs). وتوالى بعدها ظهور منصات أخرى مثل بوكينغ دوت كوم (Booking.com) عام 1996م، وتريب أدفايزر (TripAdvisor) عام 2000م، وإير بي إن بي (Airbnb) عام 2008م.

وقد أدى هذا التحول إلى إعادة تشكيل سلسلة التوزيع السياحي بالكامل، حيث أصبح بإمكان المسافرين الوصول مباشرة إلى مقدمي الخدمات دون الحاجة إلى وسيط تقليدي. كما قامت شركات الطيران بتخفيض أو إلغاء العمولات المدفوعة لوكالات الأسفار، مما شكّل ضربة اقتصادية كبيرة لهذه الأخيرة \parencite{standing2014}.

ويوضح الجدول \ref{tab:evolution} أهم المحطات التاريخية في تطور وكالات الأسفار:

\begin{table}[H]
\centering
\caption{المحطات التاريخية الرئيسية في تطور وكالات الأسفار}
\label{tab:evolution}
\begin{tabular}{|r|r|}
\hline
\textbf{السنة} & \textbf{الحدث} \\
\hline
1841 & أول رحلة جماعية منظمة بواسطة توماس كوك \\
\hline
1845 & إنشاء أول وكالة سفر تجارية \\
\hline
1915 & دخول أمريكان إكسبريس مجال السفر \\
\hline
1946 & تأسيس الاتحاد الدولي للنقل الجوي (إياتا) \\
\hline
1960 & بداية عصر السياحة الجماهيرية \\
\hline
1976 & إطلاق نظام الحجز المحوسب سيبر \\
\hline
1987 & إطلاق نظام أماديوس للحجز \\
\hline
1996 & ظهور أولى وكالات السفر عبر الإنترنت \\
\hline
2000 & إطلاق تريب أدفايزر \\
\hline
2008 & إطلاق إير بي إن بي \\
\hline
2020 & تأثير جائحة كورونا على قطاع السفر \\
\hline
\end{tabular}
\end{table}


\subsection{أنواع وكالات الأسفار}

تتعدد أنواع وكالات الأسفار وتتنوع وفقاً لعدة معايير تصنيف، يمكن استعراض أبرزها على النحو التالي:

\subsubsection{التصنيف حسب طبيعة النشاط}

\textbf{أ. وكالات السفر بالتجزئة (Retail Travel Agencies):}

وهي الوكالات التي تتعامل مباشرة مع المستهلك النهائي (المسافر)، وتقوم ببيع المنتجات والخدمات السياحية التي يقدمها منظمو الرحلات وشركات الطيران والفنادق وغيرهم من مقدمي الخدمات. وتُعدّ هذه الوكالات الأكثر انتشاراً وعدداً في السوق السياحي، وتتميز بقربها من العملاء وقدرتها على تقديم خدمة شخصية ومباشرة.

تعمل وكالات التجزئة عادة بنظام العمولة، حيث تحصل على نسبة مئوية من قيمة الخدمات المباعة. وتتراوح هذه النسبة عادة بين 5\% و15\% حسب نوع الخدمة والاتفاقيات المبرمة مع مقدمي الخدمات. كما تحصل بعض الوكالات على رسوم خدمة ثابتة من العملاء مقابل عمليات البحث والحجز \parencite{middleton2009}.

\textbf{ب. منظمو الرحلات (Tour Operators):}

يُعرف منظم الرحلات بأنه المؤسسة التي تقوم بتجميع وتنسيق مختلف عناصر المنتج السياحي (النقل، الإقامة، الزيارات، الأنشطة) في باقة متكاملة تُباع بسعر شامل. ويختلف منظم الرحلات عن وكالة التجزئة في كونه يشتري الخدمات بالجملة من مقدمي الخدمات ثم يعيد بيعها بعد تجميعها في منتج سياحي متكامل.

يتحمل منظم الرحلات مخاطر مالية أكبر من وكالة التجزئة، حيث يلتزم بشراء كميات كبيرة من المقاعد والغرف الفندقية مسبقاً، بغض النظر عما إذا كان سيتمكن من بيعها جميعاً. وفي المقابل، يحقق هوامش ربح أعلى بفضل الشراء بالجملة والأسعار التفضيلية التي يحصل عليها.

\textbf{ج. وكالات السفر بالجملة (Wholesale Travel Agencies):}

تعمل هذه الوكالات كوسيط بين مقدمي الخدمات السياحية ووكالات التجزئة، حيث تشتري كميات كبيرة من المنتجات السياحية بأسعار تفضيلية وتعيد بيعها لوكالات التجزئة. ولا تتعامل عادة مع المستهلك النهائي بشكل مباشر.

\subsubsection{التصنيف حسب التخصص}

\textbf{أ. وكالات الأسفار العامة:}

تقدم هذه الوكالات مجموعة واسعة ومتنوعة من الخدمات السياحية دون التخصص في نوع معين. وتستهدف شرائح واسعة من المسافرين بمختلف احتياجاتهم وميزانياتهم.

\textbf{ب. وكالات الأسفار المتخصصة:}

تركز هذه الوكالات على نوع محدد من السفر أو على شريحة معينة من العملاء. ومن أمثلة التخصصات:

\begin{itemize}[label=\textbf{--}]
\item \textbf{سياحة الأعمال:} تتخصص في تنظيم سفرات رجال الأعمال والمؤتمرات والمعارض.
\item \textbf{السياحة الدينية:} تتخصص في تنظيم رحلات الحج والعمرة وزيارة الأماكن المقدسة.
\item \textbf{سياحة المغامرات:} تتخصص في تنظيم رحلات المغامرة والأنشطة الرياضية في الطبيعة.
\item \textbf{السياحة الفاخرة:} تستهدف الشريحة الراقية من المسافرين وتقدم خدمات عالية الجودة.
\item \textbf{سياحة العلاج:} تتخصص في تنظيم رحلات العلاج والاستشفاء.
\end{itemize}

\subsubsection{التصنيف حسب نمط التشغيل}

\textbf{أ. الوكالات المستقلة:}

وهي وكالات مملوكة ومدارة بشكل مستقل، تعمل بصورة فردية دون الارتباط بشبكة أو سلسلة. تتميز بمرونة أكبر في اتخاذ القرارات لكنها قد تعاني من محدودية الموارد والقدرة التفاوضية.

\textbf{ب. سلاسل وكالات الأسفار:}

وهي مجموعة وكالات تعمل تحت علامة تجارية واحدة وتتبع نظام إدارة موحد. تستفيد من وفورات الحجم والقدرة التفاوضية الأكبر مع مقدمي الخدمات، ومن الدعم التسويقي والتكنولوجي المشترك.

\textbf{ج. وكالات الامتياز (Franchise):}

تعمل وفق نظام الامتياز التجاري، حيث تحصل الوكالة على حق استخدام العلامة التجارية والنظام التشغيلي مقابل رسوم ونسبة من الأرباح. تجمع بين استقلالية الملكية ومزايا الانتماء إلى شبكة كبيرة.

\subsubsection{التصنيف حسب الوسيلة}

\textbf{أ. الوكالات التقليدية (Brick-and-Mortar):}

تمارس نشاطها من خلال مكاتب ومحلات تجارية فعلية يتوجه إليها العملاء شخصياً. وتتميز بالتواصل المباشر والشخصي مع العملاء، لكنها تتطلب تكاليف تشغيلية مرتفعة (إيجار، موظفين، تجهيزات).

\textbf{ب. وكالات السفر عبر الإنترنت (Online Travel Agencies - OTAs):}

تمارس نشاطها بالكامل عبر الإنترنت من خلال مواقع وتطبيقات إلكترونية. وتتميز بانخفاض التكاليف التشغيلية والقدرة على الوصول إلى قاعدة عملاء واسعة. ومن أبرز الأمثلة: بوكينغ وإكسبيديا وتريفاغو.

\textbf{ج. الوكالات الهجينة (Click-and-Mortar):}

تجمع بين التواجد المادي الفعلي والتواجد الرقمي عبر الإنترنت، مما يتيح لها الاستفادة من مزايا كلا النموذجين. وتتجه العديد من الوكالات التقليدية نحو هذا النموذج كاستراتيجية للتكيف مع التطورات الرقمية.

ويوضح الجدول \ref{fig:types} تصنيف أنواع وكالات الأسفار وفقاً للمعايير المختلفة:

\begin{table}[H]
\centering
\caption{تصنيف أنواع وكالات الأسفار وفقاً للمعايير المختلفة}
\label{fig:types}
\begin{tabular}{|r|r|r|}
\hline
\textbf{معيار التصنيف} & \textbf{النوع الأول} & \textbf{النوع الثاني} \\
\hline
حسب النشاط & وكالات التجزئة & وكالات الجملة \\
\hline
حسب التخصص & وكالات عامة & وكالات متخصصة \\
\hline
حسب الوسيلة & وكالات تقليدية & وكالات إلكترونية \\
\hline
حسب نمط التشغيل & وكالات مستقلة & سلاسل وكالات / امتياز \\
\hline
\end{tabular}
\end{table}


% ======================================================================
% المبحث الثاني: الخدمات التي تقدمها وكالات الأسفار
% ======================================================================
\section{المبحث الثاني: الخدمات التي تقدمها وكالات الأسفار}

تقدم وكالات الأسفار مجموعة واسعة ومتنوعة من الخدمات التي تلبي مختلف احتياجات المسافرين. وتتراوح هذه الخدمات بين الخدمات الأساسية التي تمثل جوهر نشاط الوكالة، والخدمات المساعدة والتكميلية التي تضيف قيمة للعميل وتعزز تجربة السفر. ويمكن تصنيف هذه الخدمات على النحو التالي:

\subsection{خدمات النقل والحجز}

\subsubsection{حجز تذاكر الطيران}

تُعدّ خدمة حجز تذاكر الطيران من أقدم وأهم الخدمات التي تقدمها وكالات الأسفار. وتشمل هذه الخدمة البحث عن أفضل الرحلات والأسعار، وإجراء الحجوزات، وإصدار التذاكر، وإدارة التغييرات والإلغاءات. وتستخدم الوكالات أنظمة الحجز العالمية (GDS) مثل أماديوس وسيبر وغاليليو للوصول إلى قواعد بيانات شركات الطيران وإجراء الحجوزات بشكل فوري.

وقد كانت وكالات الأسفار تاريخياً القناة الرئيسية لتوزيع تذاكر الطيران، حيث كانت تحصل على عمولات تتراوح بين 7\% و10\% من قيمة التذكرة. غير أن هذا النموذج تعرض لضغوط كبيرة منذ أواخر التسعينيات، حيث قامت معظم شركات الطيران بتخفيض العمولات تدريجياً ثم إلغائها في كثير من الحالات \parencite{iata2023}، مما اضطر الوكالات إلى التحول نحو فرض رسوم خدمة على العملاء.

\subsubsection{حجز وسائل النقل الأخرى}

بالإضافة إلى تذاكر الطيران، تقدم وكالات الأسفار خدمات حجز وسائل النقل الأخرى التي تشمل:

\begin{itemize}[label=\textbf{--}]
\item حجز تذاكر القطارات والحافلات السياحية.
\item تأجير السيارات في الوجهات السياحية.
\item حجز الرحلات البحرية والعبّارات.
\item تنظيم خدمات النقل من وإلى المطارات.
\item حجز سيارات خاصة مع سائق للتنقلات المحلية.
\end{itemize}

\subsection{خدمات الإقامة}

تتولى وكالات الأسفار حجز مختلف أنواع الإقامة للمسافرين، بما في ذلك الفنادق بمختلف فئاتها (من النجمة الواحدة إلى الخمس نجوم)، والشقق المفروشة، والمنتجعات السياحية، وبيوت الشباب، والفيلات السياحية. وتمتلك الوكالات عادة اتفاقيات مع سلاسل فندقية وفنادق مستقلة تحصل بموجبها على أسعار تفضيلية يمكنها تمريرها للعملاء.

وتتميز الوكالات في هذا المجال بقدرتها على تقديم نصائح موثوقة بشأن اختيار مكان الإقامة الأنسب بناءً على معرفتها المباشرة بالفنادق والوجهات، وهو ما يُعرف بالـ``المعرفة الميدانية'' التي لا تتوفر عادة لدى المنصات الإلكترونية بنفس العمق والدقة.

\subsection{تنظيم الرحلات والبرامج السياحية}

تُعتبر خدمة تنظيم الرحلات والبرامج السياحية من أبرز الخدمات المميزة لوكالات الأسفار، وتشمل:

\subsubsection{الرحلات الشاملة (الباقات السياحية)}

يقوم منظمو الرحلات بتصميم باقات سياحية متكاملة تجمع بين مختلف عناصر الرحلة في منتج واحد بسعر شامل. وتتضمن هذه الباقات عادةً: تذاكر الطيران ذهاباً وإياباً، والإقامة الفندقية، والنقل المحلي، والجولات السياحية المصحوبة بمرشد، والوجبات (كلياً أو جزئياً)، والتأمين على السفر.

وتوفر الباقات السياحية للمسافر مزايا عديدة أبرزها: سهولة التخطيط وتوفير الوقت والجهد، والحصول على سعر إجمالي أقل مما لو تم حجز كل عنصر على حدة، والاستفادة من خبرة المنظم في اختيار أفضل الخيارات، والحماية القانونية في حالات الإلغاء أو المشاكل أثناء الرحلة \parencite{cooper2018}.

\subsubsection{الرحلات المُصممة حسب الطلب}

تقدم بعض الوكالات خدمة تصميم رحلات مخصصة وفقاً لرغبات واحتياجات العميل الخاصة. وتتطلب هذه الخدمة مهارة ومعرفة عالية من موظفي الوكالة، حيث يتم بناء البرنامج من الصفر بناءً على تفضيلات العميل من حيث الوجهة والمدة والميزانية والأنشطة المفضلة.

\subsection{الخدمات الإدارية والاستشارية}

\subsubsection{التأشيرات والوثائق}

تقدم وكالات الأسفار خدمات مهمة تتعلق بالإجراءات الإدارية للسفر، من أبرزها:

\begin{itemize}[label=\textbf{--}]
\item المساعدة في استخراج التأشيرات للدول التي تتطلب ذلك.
\item تقديم المعلومات حول متطلبات الدخول لمختلف الدول.
\item المساعدة في إعداد ملفات طلب التأشيرة.
\item متابعة طلبات التأشيرات مع السفارات والقنصليات.
\item تقديم النصح بشأن جوازات السفر ومدة صلاحيتها.
\end{itemize}

\subsubsection{التأمين على السفر}

توفر الوكالات لعملائها مختلف أنواع التأمين المرتبط بالسفر، بما في ذلك التأمين الصحي، وتأمين إلغاء الرحلة، وتأمين الأمتعة، والتأمين ضد الحوادث. وتتعامل الوكالات مع شركات تأمين متخصصة وتقوم ببيع منتجاتها التأمينية كجزء من الخدمة الشاملة.

\subsubsection{الاستشارات والنصائح}

تقدم وكالات الأسفار خدمة استشارية ذات قيمة عالية تشمل: تقديم نصائح حول أفضل الوجهات والأوقات المناسبة للسفر، والمعلومات حول المناخ والعادات والتقاليد المحلية في الوجهات المختلفة، والتوصيات بشأن المطاعم والأماكن السياحية والأنشطة الترفيهية، والإرشادات الأمنية والصحية.

وتمثل هذه الخدمة الاستشارية إحدى أهم نقاط القوة التي تتميز بها وكالات الأسفار التقليدية مقارنة بالمنصات الإلكترونية، حيث تعتمد على الخبرة الشخصية والمعرفة المتراكمة لموظفي الوكالة \parencite{zeithaml2018}.

\subsection{خدمات سياحة الأعمال}

تخصصت العديد من وكالات الأسفار في خدمة قطاع الأعمال والمؤسسات، وتشمل خدماتها في هذا المجال:

\begin{itemize}[label=\textbf{--}]
\item إدارة سفرات رجال الأعمال والموظفين.
\item تنظيم المؤتمرات والندوات والمعارض.
\item تنظيم الحوافز السياحية للشركات (Incentive Travel).
\item إعداد تقارير مفصلة عن نفقات السفر.
\item التفاوض على اتفاقيات أسعار خاصة مع مقدمي الخدمات.
\end{itemize}

وتُعدّ سياحة الأعمال من أهم مصادر الدخل لوكالات الأسفار، حيث تتميز بحجم إنفاق مرتفع ومعدل تكرار عالٍ.

\subsection{خدمات الحج والعمرة}

في العالم العربي والإسلامي، تمثل خدمات الحج والعمرة جزءاً مهماً من نشاط وكالات الأسفار. وتشمل هذه الخدمات: تنظيم رحلات الحج والعمرة بمختلف فئاتها، وحجز الفنادق القريبة من الحرمين الشريفين، وتوفير خدمات النقل والإعاشة، وتقديم الإرشاد الديني والمناسكي، والمساعدة في استخراج تأشيرات الحج والعمرة.

ويوضح الجدول \ref{tab:services} ملخصاً لأهم الخدمات التي تقدمها وكالات الأسفار:

\begin{table}[H]
\centering
\caption{ملخص الخدمات الرئيسية لوكالات الأسفار}
\label{tab:services}
\begin{tabular}{|r|r|}
\hline
\textbf{فئة الخدمة} & \textbf{الخدمات الفرعية} \\
\hline
النقل والحجز & تذاكر طيران، قطارات، تأجير سيارات \\
\hline
الإقامة & فنادق، منتجعات، شقق مفروشة \\
\hline
تنظيم الرحلات & باقات شاملة، رحلات مخصصة \\
\hline
خدمات إدارية & تأشيرات، تأمين، وثائق \\
\hline
استشارات & نصائح سفر، معلومات وجهات \\
\hline
أعمال & مؤتمرات، حوافز، إدارة سفر \\
\hline
دينية & حج، عمرة، زيارات دينية \\
\hline
\end{tabular}
\end{table}


% ======================================================================
% المبحث الثالث: المنصات الإلكترونية ومزاياها
% ======================================================================
\section{المبحث الثالث: المنصات الإلكترونية ومزاياها}

\subsection{تعريف المنصات الإلكترونية للسفر}

\subsubsection{المفهوم والتعريف}

المنصات الإلكترونية للسفر هي مواقع ويب وتطبيقات رقمية متخصصة تتيح للمستخدمين البحث عن خدمات السفر والسياحة ومقارنتها وحجزها ودفع ثمنها عبر الإنترنت. وقد عرّف بوهاليس \parencite{buhalis2020} هذه المنصات بأنها ``أنظمة تكنولوجية متكاملة تستخدم الإنترنت كقناة رئيسية للتواصل مع المستهلكين وتقديم خدمات السفر والسياحة بشكل رقمي بالكامل''.

ويمكن تعريف المنصات الإلكترونية للسفر بشكل أكثر شمولية بأنها: بيئات رقمية تفاعلية تجمع بين مقدمي خدمات السفر والمسافرين في فضاء إلكتروني واحد، وتوفر أدوات للبحث والمقارنة والحجز والدفع والتقييم، مع تقديم تجربة مستخدم سلسة ومتكاملة على مدار الساعة ومن أي مكان.

\subsubsection{أنواع المنصات الإلكترونية للسفر}

تتنوع المنصات الإلكترونية للسفر من حيث نموذج العمل والخدمات المقدمة، ويمكن تصنيفها إلى عدة أنواع رئيسية:

\textbf{أ. وكالات السفر عبر الإنترنت (OTAs):}

وهي منصات تعمل كوسيط رقمي بين مقدمي الخدمات والمسافرين، وتحقق إيراداتها من العمولات أو الفرق بين سعر الشراء وسعر البيع. ومن أبرز هذه المنصات:

\begin{itemize}[label=\textbf{--}]
\item \textbf{بوكينغ دوت كوم (Booking.com):} تأسست عام 1996 في هولندا، وتُعدّ أكبر منصة لحجز الفنادق في العالم، بأكثر من 28 مليون وحدة إقامة مُدرجة في أكثر من 220 دولة ومنطقة. تعمل بنظام العمولة حيث تحصل على نسبة تتراوح بين 15\% و25\% من قيمة الحجز \parencite{booking2023}.

\item \textbf{إكسبيديا (Expedia):} تأسست عام 1996 كفرع لشركة مايكروسوفت، وتُقدم خدمات شاملة تشمل حجز الرحلات والفنادق وتأجير السيارات والأنشطة السياحية. تمتلك مجموعة إكسبيديا عدة علامات تجارية منها هوتيلز دوت كوم وأوربيتز وتريفاغو.

\item \textbf{تريب دوت كوم (Trip.com):} منصة صينية عملاقة تخدم أكثر من 400 مليون مستخدم حول العالم، وتُعدّ من أكبر منصات السفر في آسيا.
\end{itemize}

\textbf{ب. محركات البحث عن السفر (Meta-search Engines):}

وهي منصات لا تبيع خدمات السفر مباشرة، بل تقوم بتجميع ومقارنة الأسعار من مصادر متعددة (وكالات سفر إلكترونية ومواقع شركات الطيران والفنادق)، ثم توجّه المستخدم إلى الموقع المقدم للخدمة لإتمام الحجز. ومن أبرز هذه المحركات:

\begin{itemize}[label=\textbf{--}]
\item \textbf{سكاي سكانر (Skyscanner):} متخصص في مقارنة أسعار تذاكر الطيران، مع توفره أيضاً على خدمات مقارنة الفنادق وتأجير السيارات.
\item \textbf{غوغل فلايتس (Google Flights):} خدمة من غوغل لمقارنة أسعار الرحلات الجوية.
\item \textbf{تريفاغو (Trivago):} متخصص في مقارنة أسعار الفنادق من مصادر متعددة.
\item \textbf{كاياك (Kayak):} يقارن أسعار الرحلات والفنادق وتأجير السيارات.
\end{itemize}

\textbf{ج. منصات الاقتصاد التشاركي:}

وهي منصات تربط بين أصحاب الممتلكات (منازل، شقق، غرف) والمسافرين الباحثين عن إقامة، وتمثل نموذجاً جديداً في صناعة السياحة يُعرف بالاقتصاد التشاركي. ومن أبرزها:

\begin{itemize}[label=\textbf{--}]
\item \textbf{إير بي إن بي (Airbnb):} تأسست عام 2008 وأصبحت من أكبر مقدمي خدمات الإقامة في العالم بأكثر من 7 ملايين مسكن مُدرج في أكثر من 220 دولة.
\item \textbf{في آر بي أو (VRBO):} متخصصة في تأجير المنازل والفيلات لقضاء العطلات.
\end{itemize}

\textbf{د. منصات التقييم والمراجعات:}

وهي منصات تتيح للمسافرين مشاركة تجاربهم وتقييماتهم للفنادق والمطاعم والوجهات السياحية، وتؤثر بشكل كبير في قرارات المسافرين. ومن أبرزها:

\begin{itemize}[label=\textbf{--}]
\item \textbf{تريب أدفايزر (TripAdvisor):} أكبر منصة للمراجعات السياحية في العالم بأكثر من مليار تقييم ومراجعة.
\item \textbf{يلب (Yelp):} تتخصص في تقييم المطاعم والخدمات المحلية.
\end{itemize}

\subsection{مزايا المنصات الإلكترونية}

تتمتع المنصات الإلكترونية للسفر بمجموعة واسعة من المزايا التي جعلتها تستقطب أعداداً متزايدة من المسافرين على حساب وكالات الأسفار التقليدية. ويمكن تصنيف هذه المزايا في عدة فئات:

\subsubsection{مزايا مرتبطة بسهولة الاستخدام والوصول}

\textbf{أ. التوفر على مدار الساعة:}

تتيح المنصات الإلكترونية للمسافرين إجراء عمليات البحث والحجز في أي وقت يشاؤون، دون التقيد بمواعيد العمل الرسمية. وهذه الميزة ذات أهمية خاصة في عالم معولم يتعامل فيه المسافرون مع مناطق زمنية مختلفة. فبينما تعمل وكالات الأسفار التقليدية عادة خلال ساعات محددة (غالباً من الثامنة صباحاً إلى الخامسة مساءً)، فإن المنصات الإلكترونية متاحة 24 ساعة في اليوم و7 أيام في الأسبوع و365 يوماً في السنة \parencite{xiang2015}.

\textbf{ب. الوصول من أي مكان:}

يمكن للمسافر الوصول إلى المنصات الإلكترونية من أي مكان في العالم عبر جهاز كمبيوتر أو هاتف ذكي أو جهاز لوحي متصل بالإنترنت. وقد ساهم انتشار الهواتف الذكية في تعزيز هذه الميزة بشكل كبير، حيث أصبح أكثر من 70\% من عمليات البحث عن السفر تتم عبر الأجهزة المحمولة \parencite{google2022}.

\textbf{ج. سهولة واجهة المستخدم:}

تستثمر المنصات الإلكترونية الكبرى مبالغ ضخمة في تصميم واجهات مستخدم سهلة وبديهية ومريحة بصرياً، مما يجعل عملية البحث والحجز تجربة سلسة وممتعة. وتتضمن هذه الواجهات عادة أدوات بحث متقدمة وفلاتر دقيقة وخرائط تفاعلية وصور عالية الجودة.

\subsubsection{مزايا مرتبطة بالمعلومات والشفافية}

\textbf{أ. وفرة المعلومات:}

توفر المنصات الإلكترونية كماً هائلاً من المعلومات المفصلة حول الرحلات والفنادق والوجهات السياحية. وتشمل هذه المعلومات: أوصاف تفصيلية للخدمات والمرافق، وصور فوتوغرافية عالية الدقة، ومقاطع فيديو، وخرائط تفاعلية، ومعلومات عن الموقع والمسافات. وتساعد هذه المعلومات المسافر على اتخاذ قرار مستنير دون الحاجة إلى زيارة وكالة سفر.

\textbf{ب. تقييمات ومراجعات المستخدمين:}

تُعدّ تقييمات ومراجعات المستخدمين السابقين من أهم المزايا التي تقدمها المنصات الإلكترونية. فمنصة مثل تريب أدفايزر تضم أكثر من مليار مراجعة وتقييم من مسافرين حقيقيين، مما يوفر للمسافر المحتمل رؤية واقعية وغير متحيزة عن جودة الخدمات. وتشير الدراسات إلى أن أكثر من 80\% من المسافرين يقرؤون التقييمات عبر الإنترنت قبل اتخاذ قرار الحجز \parencite{amaro2015}.

\textbf{ج. الشفافية في الأسعار:}

تتيح المنصات الإلكترونية للمسافرين مقارنة الأسعار من مصادر متعددة بسهولة وسرعة. فمحركات البحث عن السفر مثل سكاي سكانر وكاياك تعرض أسعار نفس الرحلة أو الفندق من عشرات المصادر المختلفة، مما يمكّن المسافر من اختيار أفضل سعر. وقد أسهمت هذه الشفافية في تكثيف المنافسة السعرية في سوق السفر.

\subsubsection{مزايا مرتبطة بالتكلفة}

\textbf{أ. أسعار تنافسية:}

بفضل انخفاض تكاليفها التشغيلية مقارنة بالوكالات التقليدية (عدم الحاجة إلى مكاتب ومحلات تجارية ومقابل أقل للموظفين)، تستطيع المنصات الإلكترونية تقديم أسعار أكثر تنافسية للمسافرين. كما أن المنافسة الشديدة بين المنصات تدفعها إلى تقديم عروض وخصومات مستمرة. وتشير الدراسات إلى أن الأسعار عبر الإنترنت تكون أقل بنسبة 10\% إلى 30\% مقارنة بالأسعار المقدمة في وكالات الأسفار التقليدية في كثير من الحالات \parencite{statista2023}.

\textbf{ب. عروض وخصومات حصرية:}

تقدم المنصات الإلكترونية بشكل مستمر عروضاً ترويجية وخصومات خاصة لجذب العملاء والاحتفاظ بهم. وتشمل هذه العروض: خصومات الحجز المبكر، وعروض اللحظة الأخيرة، وبرامج الولاء والمكافآت، وعروض خاصة لمستخدمي التطبيق، وباقات مخفضة عند حجز عدة خدمات معاً.

\textbf{ج. عدم وجود رسوم خدمة:}

على عكس العديد من وكالات الأسفار التقليدية التي أصبحت تفرض رسوم خدمة على العملاء (بعد تخفيض أو إلغاء العمولات من شركات الطيران)، فإن معظم المنصات الإلكترونية لا تفرض رسوماً إضافية على المسافرين، وتحقق إيراداتها بشكل أساسي من العمولات التي تحصل عليها من مقدمي الخدمات.

\subsubsection{مزايا مرتبطة بالتكنولوجيا والابتكار}

\textbf{أ. التخصيص والتوصيات الذكية:}

تستخدم المنصات الإلكترونية تقنيات الذكاء الاصطناعي وتحليل البيانات الضخمة لتقديم توصيات مخصصة لكل مستخدم بناءً على سلوكه السابق وتفضيلاته. فعلى سبيل المثال، تقوم منصة بوكينغ بتحليل أنماط البحث والحجز السابقة للمستخدم لتقديم اقتراحات تتوافق مع اهتماماته، مما يحسّن تجربة المستخدم ويزيد من احتمالية إتمام الحجز \parencite{law2014}.

\textbf{ب. تطبيقات الهاتف المحمول:}

توفر المنصات الإلكترونية الكبرى تطبيقات متطورة للهواتف الذكية تتيح للمسافرين إجراء الحجوزات والاطلاع على تفاصيل رحلاتهم والحصول على إشعارات فورية وبطاقات صعود رقمية وغيرها من الخدمات أثناء التنقل. وقد أصبحت التطبيقات قناة أساسية للتفاعل مع المسافرين، حيث تمثل أكثر من 50\% من حجوزات بعض المنصات.

\textbf{ج. المرونة في الإدارة:}

تتيح المنصات الإلكترونية للمسافرين إدارة حجوزاتهم بمرونة عالية، بما في ذلك تعديل التواريخ وتغيير الخيارات وإلغاء الحجوزات، وذلك بضغطات بسيطة على الشاشة دون الحاجة إلى الاتصال أو زيارة مكتب. كما توفر أنظمة إشعارات تُبلّغ المسافر بأي تغييرات أو تحديثات تتعلق برحلته.

ويوضح الجدول \ref{fig:advantages} مقارنة بين أبرز مزايا المنصات الإلكترونية ووكالات الأسفار التقليدية:

\begin{table}[H]
\centering
\caption{مقارنة بين مزايا المنصات الإلكترونية ووكالات الأسفار التقليدية (من 10)}
\label{fig:advantages}
\begin{tabular}{|r|r|r|}
\hline
\textbf{المعيار} & \textbf{المنصات الإلكترونية} & \textbf{وكالات الأسفار التقليدية} \\
\hline
السعر & 9/10 & 6/10 \\
\hline
السهولة & 9/10 & 5/10 \\
\hline
التوفر & 10/10 & 4/10 \\
\hline
المعلومات & 9/10 & 7/10 \\
\hline
التخصيص & 8/10 & 7/10 \\
\hline
الخدمة الشخصية & 4/10 & 9/10 \\
\hline
\end{tabular}
\end{table}

ويوضح الشكل \ref{fig:comparison_chart} تمثيلاً بيانياً لهذه المقارنة، حيث يتضح تفوق المنصات الإلكترونية في معظم المعايير باستثناء الخدمة الشخصية التي تتفوق فيها وكالات الأسفار بشكل واضح:

\begin{figure}[H]
\centering
\begin{tikzpicture}
\begin{axis}[
    ybar,
    bar width=10pt,
    width=0.85\textwidth,
    height=7cm,
    ylabel={الدرجة (من 10)},
    xtick={1,2,3,4,5,6},
    xticklabels={السعر, السهولة, التوفر, المعلومات, التخصيص, الخدمة الشخصية},
    x tick label style={font=\small, align=center},
    ymin=0, ymax=11,
    ytick={0,2,4,6,8,10},
    legend style={at={(0.5,-0.18)}, anchor=north, legend columns=2, font=\small},
    nodes near coords,
    every node near coord/.append style={font=\scriptsize},
    enlarge x limits=0.12,
]
\addplot[fill=blue!60, draw=blue!70] coordinates {(1,9) (2,9) (3,10) (4,9) (5,8) (6,4)};
\addplot[fill=orange!50, draw=orange!60] coordinates {(1,6) (2,5) (3,4) (4,7) (5,7) (6,9)};
\legend{المنصات الإلكترونية, وكالات الأسفار التقليدية}
\end{axis}
\end{tikzpicture}
\caption{تمثيل بياني لمقارنة المزايا بين المنصات الإلكترونية ووكالات الأسفار}
\label{fig:comparison_chart}
\end{figure}

\subsection{نماذج الأعمال للمنصات الإلكترونية}

تعتمد المنصات الإلكترونية للسفر على نماذج أعمال متنوعة لتحقيق الإيرادات، ويمكن تلخيص أبرزها فيما يلي:

\subsubsection{نموذج العمولة (Commission Model)}

يُعدّ هذا النموذج الأكثر شيوعاً في صناعة السفر عبر الإنترنت. وبموجبه، تحصل المنصة على نسبة مئوية من قيمة كل حجز يتم عبرها. وتتراوح نسبة العمولة عادة بين 15\% و25\% لحجوزات الفنادق، وبين 3\% و5\% لتذاكر الطيران. وتتبع هذا النموذج منصات كبرى مثل بوكينغ دوت كوم وإكسبيديا.

\subsubsection{نموذج التاجر (Merchant Model)}

في هذا النموذج، تشتري المنصة غرفاً فندقية أو مقاعد طيران بأسعار الجملة، ثم تعيد بيعها للمسافرين بسعر أعلى مع إضافة هامش ربحها. ويتميز هذا النموذج بتحقيق هوامش ربح أعلى لكنه ينطوي على مخاطر مالية أكبر.

\subsubsection{نموذج الإعلانات (Advertising Model)}

تعتمد بعض المنصات، خاصة محركات البحث عن السفر، على الإيرادات الإعلانية كمصدر رئيسي للدخل. حيث تدفع شركات الطيران والفنادق ووكالات السفر الإلكترونية مقابل الظهور في نتائج البحث أو الحصول على مواقع بارزة في الموقع. وتتبع هذا النموذج منصات مثل تريب أدفايزر وسكاي سكانر.

\subsubsection{نموذج الاشتراك (Subscription Model)}

ظهر هذا النموذج حديثاً في بعض المنصات التي تقدم خدمات متميزة مقابل اشتراك شهري أو سنوي. ويحصل المشتركون على مزايا حصرية مثل خصومات إضافية وخدمة عملاء مميزة وضمانات أفضل.

\subsection{الاتجاهات الحديثة في المنصات الإلكترونية}

تتطور المنصات الإلكترونية للسفر بشكل مستمر مع تطور التكنولوجيا وتغير توقعات المستهلكين. ومن أبرز الاتجاهات الحديثة:

\begin{itemize}[label=\textbf{--}]
\item \textbf{الذكاء الاصطناعي والروبوتات المحادثة:} تستخدم المنصات بشكل متزايد تقنيات الذكاء الاصطناعي لتقديم خدمة عملاء آلية من خلال روبوتات المحادثة (Chatbots) التي يمكنها الإجابة على استفسارات المسافرين وتقديم المساعدة على مدار الساعة.

\item \textbf{الواقع الافتراضي والمعزز:} بدأت بعض المنصات في استخدام تقنيات الواقع الافتراضي لتمكين المسافرين من ``زيارة'' الفنادق والوجهات السياحية افتراضياً قبل الحجز، مما يساعد في اتخاذ قرارات أفضل.

\item \textbf{التخصيص الفائق:} تتجه المنصات نحو مستويات أعمق من التخصيص باستخدام تحليل البيانات الضخمة والتعلم الآلي، لتقديم تجارب مصممة خصيصاً لكل مستخدم.

\item \textbf{الاستدامة والسياحة المسؤولة:} تولي المنصات اهتماماً متزايداً بموضوع الاستدامة البيئية، حيث بدأت في عرض معلومات عن البصمة الكربونية للرحلات وتشجيع الخيارات الأكثر استدامة.

\item \textbf{المدفوعات الرقمية المتقدمة:} توسيع خيارات الدفع لتشمل المحافظ الإلكترونية والعملات المشفرة وأنظمة الدفع المحلية في مختلف الأسواق.
\end{itemize}


\subsection{البنية التحتية التكنولوجية للمنصات الإلكترونية}

لفهم أعمق لتفوق المنصات الإلكترونية، من المفيد تحليل البنية التحتية التكنولوجية التي تقوم عليها هذه المنصات والتي تمثل أحد أهم عوامل نجاحها.

\subsubsection{الحوسبة السحابية والبنية التحتية}

تعتمد المنصات الإلكترونية الكبرى على بنية تحتية تكنولوجية ضخمة ومعقدة تقوم على تقنيات الحوسبة السحابية. فمنصة مثل بوكينغ دوت كوم تعالج ملايين عمليات البحث والحجز يومياً، مما يتطلب قدرات حوسبية هائلة وأنظمة ذات موثوقية عالية. وتستخدم هذه المنصات خدمات الحوسبة السحابية من مزودين عالميين مثل أمازون ويب سيرفيسز (AWS) ومايكروسوفت أزور وغوغل كلاود، مما يتيح لها توسيع أو تقليص قدراتها الحوسبية تلقائياً حسب حجم الطلب.

وتتميز هذه البنية التحتية بعدة خصائص أساسية:

\begin{itemize}[label=\textbf{--}]
\item \textbf{التوافرية العالية:} أنظمة مكررة في مراكز بيانات متعددة حول العالم لضمان استمرارية الخدمة.
\item \textbf{المرونة:} القدرة على التوسع التلقائي لاستيعاب فترات الذروة (مواسم الإجازات، العروض الترويجية).
\item \textbf{السرعة:} أوقات استجابة لا تتجاوز أجزاء من الثانية لتحسين تجربة المستخدم.
\item \textbf{الأمان:} طبقات متعددة من الحماية للبيانات الشخصية والمالية.
\end{itemize}

\subsubsection{أنظمة التسعير الديناميكي}

من أبرز الابتكارات التكنولوجية للمنصات الإلكترونية أنظمة التسعير الديناميكي (Dynamic Pricing)، وهي أنظمة خوارزمية معقدة تقوم بتعديل الأسعار تلقائياً وفي الوقت الحقيقي بناءً على مجموعة من العوامل، منها \parencite{laudon2020}:

\begin{itemize}[label=\textbf{--}]
\item مستوى العرض والطلب الحالي.
\item الموسم والتوقيت (أوقات الذروة مقابل أوقات الهدوء).
\item سلوك المستخدم وتاريخ تصفحه.
\item أسعار المنافسين.
\item المدة المتبقية قبل تاريخ الخدمة.
\item حجم الحجوزات المؤكدة مقابل السعة المتاحة.
\end{itemize}

وتتيح هذه الأنظمة للمنصات تعظيم إيراداتها مع الحفاظ على تنافسية أسعارها، وهو ما يصعب على وكالات الأسفار التقليدية تحقيقه بالأساليب اليدوية.

\subsubsection{تقنيات تحليل البيانات الضخمة}

تجمع المنصات الإلكترونية كميات هائلة من البيانات عن سلوك المستخدمين واتجاهات السوق وأداء مقدمي الخدمات. وتستخدم تقنيات تحليل البيانات الضخمة (Big Data Analytics) لاستخراج رؤى قيّمة من هذه البيانات:

\begin{itemize}[label=\textbf{--}]
\item \textbf{تحليل سلوك المستخدم:} فهم أنماط البحث والحجز والتفضيلات الفردية.
\item \textbf{التنبؤ بالطلب:} توقع حجم الطلب على وجهات ومنتجات معينة في فترات محددة.
\item \textbf{تحسين العروض:} تحديد العروض الأكثر جاذبية لكل شريحة من العملاء.
\item \textbf{كشف الاحتيال:} تحديد المعاملات المشبوهة وحماية المنصة والمستخدمين.
\item \textbf{تحليل المنافسين:} رصد ومتابعة أسعار واستراتيجيات المنافسين في الوقت الحقيقي.
\end{itemize}


\subsection{وضع وكالات الأسفار في العالم العربي}

\subsubsection{السياق العام}

يتميز قطاع وكالات الأسفار في العالم العربي بخصوصيات تميزه عن نظيره في الدول الغربية. فعلى الرغم من أن التحول الرقمي يسير بوتيرة أبطأ نسبياً مقارنة بالأسواق المتقدمة، إلا أن التحديات التي تواجه الوكالات العربية لا تقل حدة. ويمكن تلخيص أبرز خصوصيات هذا السياق \parencite{albalushi2019}:

\begin{itemize}[label=\textbf{--}]
\item أهمية خاصة لقطاع الحج والعمرة الذي يمثل جزءاً كبيراً من نشاط الوكالات في الدول الإسلامية.
\item تأثير العوامل الثقافية والاجتماعية على تفضيلات المسافرين (الثقة في التعامل الشخصي، أهمية العلاقات).
\item تفاوت كبير في مستوى التطور الرقمي بين بلدان الخليج العربي والبلدان الأخرى.
\item دور الحكومات في تنظيم وتوجيه القطاع، خاصة في ما يتعلق بالسياحة الدينية.
\item تنامي نسبة الشباب في المجتمعات العربية وتأثير ذلك على أنماط الاستهلاك السياحي.
\end{itemize}

\subsubsection{التحديات الخاصة في السوق العربي}

تواجه وكالات الأسفار في المنطقة العربية تحديات إضافية خاصة بالسياق المحلي:

\begin{itemize}[label=\textbf{--}]
\item \textbf{ضعف البنية التحتية الرقمية:} في بعض البلدان العربية، لا تزال البنية التحتية للاتصالات والإنترنت دون المستوى المطلوب لدعم التحول الرقمي الكامل.
\item \textbf{محدودية أنظمة الدفع الإلكتروني:} لا يزال انتشار بطاقات الائتمان والمحافظ الإلكترونية محدوداً في بعض الأسواق العربية مقارنة بالدول المتقدمة.
\item \textbf{الحواجز اللغوية:} محدودية المحتوى العربي الجيد على المنصات الإلكترونية العالمية.
\item \textbf{الأوضاع السياسية والأمنية:} تأثير عدم الاستقرار في بعض مناطق العالم العربي على حركة السياحة والسفر.
\item \textbf{البيروقراطية الإدارية:} تعقيد الإجراءات المتعلقة بالتراخيص والتأشيرات في بعض الدول.
\end{itemize}

\subsubsection{الفرص المتاحة}

في المقابل، تتوفر فرص مهمة لوكالات الأسفار في العالم العربي:

\begin{itemize}[label=\textbf{--}]
\item \textbf{النمو السياحي:} تشهد عدة دول عربية (السعودية، الإمارات، المغرب، مصر) استثمارات ضخمة في قطاع السياحة.
\item \textbf{رؤية 2030 السعودية:} تستهدف المملكة العربية السعودية استقطاب 100 مليون سائح بحلول 2030، مما يخلق فرصاً هائلة.
\item \textbf{تنامي الطبقة المتوسطة:} يؤدي تحسن المستوى المعيشي في بعض الدول العربية إلى زيادة الإنفاق على السفر والسياحة.
\item \textbf{السياحة البينية العربية:} تمثل السياحة بين الدول العربية سوقاً متنامياً يمكن لوكالات الأسفار استغلاله.
\item \textbf{الحج والعمرة:} سوق مستقر ومتنامٍ يتطلب خبرة متخصصة لا تتوفر بسهولة عبر المنصات الإلكترونية.
\end{itemize}

ويوضح الجدول \ref{tab:arab_market} بعض المؤشرات السياحية في الدول العربية الرئيسية:

\begin{table}[H]
\centering
\caption{مؤشرات سياحية مختارة في الدول العربية الرئيسية (2023)}
\label{tab:arab_market}
\begin{tabular}{|r|r|r|r|}
\hline
\textbf{الدولة} & \textbf{عدد السياح (مليون)} & \textbf{إيرادات السياحة (مليار \$)} & \textbf{نسبة الحجز الإلكتروني (\%)} \\
\hline
الإمارات & 21.9 & 38.5 & 55 \\
\hline
السعودية & 27.4 & 36.2 & 40 \\
\hline
المغرب & 14.5 & 10.2 & 35 \\
\hline
مصر & 14.9 & 13.6 & 30 \\
\hline
تونس & 9.4 & 2.8 & 25 \\
\hline
الأردن & 5.8 & 6.1 & 30 \\
\hline
\end{tabular}
\end{table}


\subsection{الدراسات السابقة المتعلقة بالموضوع}

تناولت العديد من الدراسات الأكاديمية موضوع التحول الرقمي في صناعة السفر وتأثيره على وكالات الأسفار التقليدية. ويمكن استعراض أبرز هذه الدراسات:

\subsubsection{الدراسات الأجنبية}

\textbf{1. دراسة ستاندينغ وتانغ-تافي وبويار (2014):}

هدفت هذه الدراسة إلى تحليل تأثير الإنترنت على وكالات السفر التقليدية في أستراليا. وتوصلت إلى أن الإنترنت أحدث تحولاً جذرياً في سلسلة التوزيع السياحي، وأن وكالات الأسفار التي لم تتكيف مع التطورات الرقمية واجهت تراجعاً حاداً في أعمالها. كما أكدت الدراسة أن التخصص والخدمة الشخصية يمثلان فرصة للتمايز \parencite{standing2014}.

\textbf{2. دراسة بوهاليس (2020):}

ركزت هذه الدراسة على التحول الرقمي في صناعة السياحة بشكل عام، وأبرزت كيف أن التكنولوجيات الذكية (الذكاء الاصطناعي، إنترنت الأشياء، البلوك تشين) تُعيد تشكيل القطاع بالكامل. وخلصت إلى أن الابتكار التكنولوجي المستمر سيزيد من الضغط على الوسطاء التقليديين، لكنه سيخلق أيضاً فرصاً جديدة لمن يستطيع التكيف \parencite{buhalis2020}.

\textbf{3. دراسة أمارو وديوغو (2015):}

درست العوامل المؤثرة في نية شراء السفر عبر الإنترنت، وتوصلت إلى أن الثقة وسهولة الاستخدام والفائدة المدركة هي العوامل الأكثر تأثيراً في قرار المسافر بالحجز عبر الإنترنت. كما أشارت إلى أن الخبرة السابقة في التعاملات الإلكترونية تلعب دوراً مهماً في تعزيز التوجه نحو الحجز الرقمي \parencite{amaro2015}.

\subsubsection{الدراسات العربية}

\textbf{1. دراسة الحماد (2021):}

تناولت التحول الرقمي في قطاع السياحة العربي، وأبرزت الفجوة الرقمية الكبيرة بين وكالات الأسفار في المنطقة العربية ونظيراتها في الدول المتقدمة. وأوصت بضرورة تبني استراتيجيات تحول رقمي شاملة مع مراعاة الخصوصيات الثقافية والاقتصادية للمنطقة \parencite{alhammad2021}.

\textbf{2. دراسة القحطاني (2020):}

ركزت على تطبيقات التجارة الإلكترونية في قطاع السياحة بالمملكة العربية السعودية. وأظهرت أن معدل تبني التجارة الإلكترونية في وكالات الأسفار السعودية لا يزال دون المستوى المأمول، رغم الإمكانات الكبيرة المتاحة. وأوصت بتعزيز الدعم الحكومي لتسريع التحول الرقمي في القطاع \parencite{alqahtani2020}.

\textbf{3. دراسة البلوشي (2019):}

تناولت التحديات والفرص التي تواجه صناعة السياحة في العالم العربي بشكل عام. وأبرزت أهمية التكيف مع التغيرات التكنولوجية والسلوكية للحفاظ على تنافسية القطاع السياحي العربي \parencite{albalushi2019}.

\subsubsection{موقع الدراسة الحالية من الدراسات السابقة}

تتميز الدراسة الحالية عن الدراسات السابقة في عدة جوانب:

\begin{enumerate}[label=\textbf{\arabic*.}]
\item تركز بشكل خاص على التحديات التي تواجه وكالات الأسفار في مواجهة المنصات الإلكترونية، وليس على التحول الرقمي بشكل عام.
\item تقدم تحليلاً شاملاً يجمع بين الأبعاد التكنولوجية والاقتصادية والسلوكية والتنظيمية.
\item تقترح حلولاً واستراتيجيات عملية مبنية على تحليل واقعي للتحديات.
\item تأخذ بعين الاعتبار تأثير جائحة كوفيد-19 على ديناميكيات المنافسة في القطاع.
\end{enumerate}


\section*{خلاصة الفصل الأول}

من خلال ما تم عرضه في هذا الفصل، يتضح أن وكالات الأسفار قد مرت بمراحل تطور عديدة منذ تأسيسها في منتصف القرن التاسع عشر، حيث عرفت فترة ذهبية امتدت لعقود طويلة كانت خلالها الوسيط الأساسي في سلسلة التوزيع السياحي. غير أن ظهور المنصات الإلكترونية للسفر في أواخر التسعينيات وبداية الألفية الثالثة شكّل منعطفاً حاسماً في تاريخ هذا القطاع.

وقد تبيّن أن المنصات الإلكترونية تتمتع بمزايا تنافسية عديدة تشمل: التوفر الدائم، وسهولة الاستخدام، والشفافية في الأسعار، ووفرة المعلومات والتقييمات، والأسعار التنافسية، واستخدام التكنولوجيا المتقدمة في تحسين تجربة المستخدم. وهي مزايا أربكت الوكالات التقليدية وفرضت عليها تحديات غير مسبوقة.

كما تبيّن أن وكالات الأسفار في العالم العربي تواجه تحديات خاصة مرتبطة بالسياق المحلي، لكنها تتوفر أيضاً على فرص مهمة مرتبطة بنمو القطاع السياحي والخصوصيات الثقافية للمنطقة. وقد أسهمت الدراسات السابقة في إثراء فهمنا لهذه الديناميكيات، وتسعى الدراسة الحالية إلى تقديم إضافة نوعية في هذا السياق.

وفي الفصل الموالي، سيتم تناول طبيعة المنافسة القائمة بين وكالات الأسفار والمنصات الإلكترونية، وتحليل تأثير هذه المنصات على النشاط التجاري للوكالات.
