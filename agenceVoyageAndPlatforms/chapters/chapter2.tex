% ==============================================================================
% الفصل الثاني: المنافسة بين وكالات الأسفار والمنصات الإلكترونية
% ==============================================================================
\chapter{المنافسة بين وكالات الأسفار والمنصات الإلكترونية}

\section*{تمهيد}

تشهد صناعة السفر والسياحة منافسة حادة ومتعددة الأبعاد بين وكالات الأسفار التقليدية والمنصات الإلكترونية. وتتميز هذه المنافسة بطبيعتها غير المتكافئة في كثير من الجوانب، حيث تمتلك المنصات الإلكترونية موارد تكنولوجية ومالية ضخمة تمنحها تفوقاً واضحاً في العديد من المجالات. وفي المقابل، تحتفظ وكالات الأسفار التقليدية ببعض نقاط القوة التي لا تزال تمثل قيمة حقيقية لشرائح من المسافرين.

يتناول هذا الفصل طبيعة المنافسة القائمة بين وكالات الأسفار والمنصات الإلكترونية من خلال مبحثين: يحلل المبحث الأول طبيعة هذه المنافسة وأبعادها وآلياتها، بينما يدرس المبحث الثاني تأثير المنصات الإلكترونية على النشاط التجاري لوكالات الأسفار.


% ======================================================================
% المبحث الأول: طبيعة المنافسة
% ======================================================================
\section{المبحث الأول: طبيعة المنافسة}

\subsection{الإطار النظري للمنافسة في قطاع السياحة}

\subsubsection{مفهوم المنافسة وأنواعها}

تُعرّف المنافسة في سياق الأعمال بأنها التنافس بين الشركات والمؤسسات للاستحواذ على أكبر حصة ممكنة من السوق وتحقيق التفوق على المنافسين. وقد قدّم بورتر \parencite{porter2008} إطاراً نظرياً شاملاً لتحليل المنافسة في الصناعات من خلال نموذجه المعروف بـ``القوى التنافسية الخمس''، والذي يمكن تطبيقه على قطاع السفر والسياحة.

وتتعدد أنواع المنافسة التي تواجهها وكالات الأسفار في ظل التحول الرقمي:

\begin{itemize}[label=\textbf{--}]
\item \textbf{المنافسة المباشرة:} وهي المنافسة بين وكالات الأسفار فيما بينها، حيث تتنافس على نفس الشريحة من العملاء وتقدم خدمات متشابهة.
\item \textbf{المنافسة غير المباشرة:} وهي المنافسة مع المنصات الإلكترونية التي تقدم بدائل رقمية لخدمات الوكالات التقليدية.
\item \textbf{المنافسة البديلة:} وتتمثل في قدرة المسافرين على تنظيم رحلاتهم بأنفسهم دون الحاجة إلى أي وسيط (وكالة تقليدية أو منصة إلكترونية).
\end{itemize}

\subsubsection{تحليل القوى التنافسية في قطاع السفر}

يمكن تطبيق نموذج بورتر للقوى الخمس على صناعة السفر لفهم طبيعة المنافسة وكثافتها:

\textbf{أ. تهديد الداخلين الجدد:}

أدى التحول الرقمي إلى تخفيض كبير في حواجز الدخول إلى صناعة السفر. فبينما كان إنشاء وكالة سفر تقليدية يتطلب استثمارات كبيرة في المقر والتجهيزات والتراخيص والموارد البشرية المتخصصة، فإن إنشاء منصة إلكترونية أصبح أسهل بكثير وبتكاليف أقل نسبياً. وقد أدى ذلك إلى تزايد عدد المنافسين في السوق وتكثيف المنافسة \parencite{kracht2010}.

غير أنه يجب التنبيه إلى أن النجاح في سوق منصات السفر الإلكترونية يتطلب استثمارات تكنولوجية وتسويقية ضخمة للمنافسة مع اللاعبين الكبار المهيمنين، مما يحد عملياً من عدد الداخلين الجدد القادرين على المنافسة بفعالية.

\textbf{ب. القوة التفاوضية للموردين:}

يتمثل الموردون في هذا السياق في مقدمي الخدمات السياحية (شركات الطيران، الفنادق، شركات النقل). وقد تغيرت العلاقة بين الموردين وقنوات التوزيع بشكل جذري مع التحول الرقمي:

\begin{itemize}[label=\textbf{--}]
\item أصبح بإمكان شركات الطيران والفنادق البيع مباشرة للمستهلكين عبر مواقعها الإلكترونية، مما قلل من اعتمادها على الوسطاء.
\item قامت شركات الطيران بتخفيض أو إلغاء العمولات المدفوعة لوكالات الأسفار.
\item في المقابل، أصبحت الفنادق الصغيرة والمتوسطة أكثر اعتماداً على المنصات الإلكترونية الكبرى لتسويق خدماتها.
\end{itemize}

\textbf{ج. القوة التفاوضية للمشترين (المسافرين):}

ازدادت القوة التفاوضية للمسافرين بشكل كبير بفضل التحول الرقمي. فأصبح المسافر يمتلك:

\begin{itemize}[label=\textbf{--}]
\item وصولاً سهلاً إلى المعلومات والأسعار من مصادر متعددة.
\item القدرة على مقارنة العروض بسرعة وسهولة.
\item حرية الاختيار بين عدد كبير من البدائل.
\item منصات لتبادل الخبرات والتقييمات مع مسافرين آخرين.
\end{itemize}

وقد أدت هذه التطورات إلى تحويل ميزان القوة لصالح المستهلك، مما فرض ضغوطاً إضافية على وكالات الأسفار والمنصات الإلكترونية على حد سواء.

\textbf{د. تهديد المنتجات والخدمات البديلة:}

تواجه وكالات الأسفار تهديداً من عدة بدائل:

\begin{itemize}[label=\textbf{--}]
\item المنصات الإلكترونية كبديل رقمي للخدمة الوسيطة.
\item المبيعات المباشرة من شركات الطيران والفنادق عبر مواقعها.
\item التخطيط الذاتي للرحلات باستخدام أدوات الإنترنت المجانية.
\item شبكات التواصل الاجتماعي كمصدر للمعلومات والتوصيات السياحية.
\end{itemize}

\textbf{هـ. شدة المنافسة بين المتنافسين الحاليين:}

تتسم المنافسة في سوق خدمات السفر بالشدة العالية، ويعود ذلك إلى عدة عوامل:

\begin{itemize}[label=\textbf{--}]
\item تعدد المتنافسين من وكالات تقليدية ومنصات إلكترونية ومقدمي خدمات مباشرين.
\item التجانس النسبي في المنتجات والخدمات المقدمة.
\item سهولة تحويل العملاء بين مقدمي الخدمات (انخفاض تكاليف التبديل).
\item الضغط المستمر على الأسعار بسبب الشفافية السعرية.
\end{itemize}

ويلخص الجدول \ref{fig:porter} تطبيق نموذج بورتر على قطاع السفر:

\begin{table}[H]
\centering
\caption{نموذج بورتر للقوى الخمس مطبقاً على قطاع السفر}
\label{fig:porter}
\begin{tabular}{|r|r|}
\hline
\textbf{القوة التنافسية} & \textbf{التطبيق على قطاع السفر} \\
\hline
\textbf{شدة المنافسة بين المتنافسين الحاليين} & منافسة شديدة بين الوكالات التقليدية والمنصات الإلكترونية \\
\hline
\textbf{تهديد الداخلين الجدد} & سهولة إنشاء منصات رقمية جديدة بتكاليف منخفضة \\
\hline
\textbf{تهديد المنتجات البديلة} & الحجز المباشر عبر مواقع شركات الطيران والفنادق \\
\hline
\textbf{القوة التفاوضية للموردين} & شركات الطيران والفنادق تتحكم في الأسعار والعمولات \\
\hline
\textbf{القوة التفاوضية للمشترين} & المسافرون يملكون خيارات متعددة ومعلومات وافرة \\
\hline
\end{tabular}
\end{table}

ويوضح الشكل \ref{fig:porter_diagram} تمثيلاً بيانياً لنموذج بورتر مطبقاً على قطاع السفر، حيث تتفاعل القوى الخمس لتشكيل بيئة تنافسية شديدة:

\begin{figure}[H]
\centering
\begin{tikzpicture}[
    box/.style={rectangle, draw=blue!60, rounded corners=4pt, fill=blue!8, 
                text width=4.2cm, minimum height=1.5cm, align=center, font=\small\bfseries},
    cbox/.style={rectangle, draw=red!60, rounded corners=4pt, fill=red!8,
                text width=4.8cm, minimum height=1.8cm, align=center, font=\small\bfseries},
    arr/.style={->, very thick, >=stealth, draw=gray!60}
]
\node[cbox] (c) at (0,0) {شدة المنافسة بين\\المتنافسين الحاليين};
\node[box] (t) at (0,3.8) {تهديد\\الداخلين الجدد};
\node[box] (b) at (0,-3.8) {تهديد المنتجات\\البديلة};
\node[box] (l) at (-5.2,0) {القوة التفاوضية\\للموردين};
\node[box] (r) at (5.2,0) {القوة التفاوضية\\للمشترين};
\draw[arr] (t) -- (c);
\draw[arr] (b) -- (c);
\draw[arr] (l) -- (c);
\draw[arr] (r) -- (c);
\end{tikzpicture}
\caption{نموذج بورتر للقوى التنافسية الخمس مطبقاً على قطاع السفر}
\label{fig:porter_diagram}
\end{figure}


\subsection{أبعاد المنافسة بين وكالات الأسفار والمنصات الإلكترونية}

\subsubsection{البعد التكنولوجي}

يُعدّ البعد التكنولوجي من أهم أبعاد المنافسة بين وكالات الأسفار والمنصات الإلكترونية، وربما الأكثر تأثيراً. فالمنصات الإلكترونية تتمتع بتفوق تكنولوجي واضح يتجلى في عدة جوانب:

\textbf{أولاً: البنية التحتية التكنولوجية.} تستثمر المنصات الإلكترونية الكبرى مبالغ ضخمة في تطوير بنيتها التحتية التكنولوجية. فمجموعة بوكينغ هولدينغز، على سبيل المثال، أنفقت أكثر من 5 مليارات دولار على التكنولوجيا والتطوير في عام 2022 وحده. وتشمل هذه الاستثمارات: مراكز بيانات متقدمة، وأنظمة حوسبة سحابية، ومنصات معالجة بيانات ضخمة، وأنظمة أمن سيبراني متطورة.

في المقابل، تعاني معظم وكالات الأسفار التقليدية من محدودية استثماراتها التكنولوجية. فالعديد منها لا يزال يعتمد على أنظمة قديمة ولا يمتلك مواقع إلكترونية متطورة أو تطبيقات للهاتف المحمول. وحتى تلك التي تمتلك تواجداً رقمياً، فإن مستوى تطور منصاتها يبقى بعيداً عن مستوى المنصات العالمية \parencite{alhammad2021}.

\textbf{ثانياً: الذكاء الاصطناعي وتحليل البيانات.} تستخدم المنصات الإلكترونية تقنيات متقدمة في الذكاء الاصطناعي والتعلم الآلي لتحسين تجربة المستخدم وزيادة معدلات التحويل. ومن تطبيقات هذه التقنيات:

\begin{itemize}[label=\textbf{--}]
\item أنظمة التوصيات الذكية التي تقترح وجهات وفنادق بناءً على سلوك المستخدم.
\item التسعير الديناميكي الذي يعدّل الأسعار تلقائياً بناءً على العرض والطلب.
\item روبوتات المحادثة (Chatbots) لخدمة العملاء الآلية.
\item أنظمة كشف الاحتيال وحماية المعاملات المالية.
\item تحليل المشاعر والتقييمات لفهم رضا العملاء.
\end{itemize}

\textbf{ثالثاً: تجربة المستخدم الرقمية.} تولي المنصات الإلكترونية اهتماماً بالغاً بتصميم تجربة المستخدم (UX)، حيث توظف فرقاً متخصصة في التصميم وإجراء البحث المستمر لتحسين واجهات الاستخدام وتبسيط رحلة العميل من البحث إلى الحجز إلى ما بعد الرحلة.

\subsubsection{البعد التسويقي}

تتمتع المنصات الإلكترونية بقدرات تسويقية ضخمة تفوق بكثير ما تمتلكه وكالات الأسفار التقليدية:

\textbf{أ. ميزانيات التسويق الرقمي:}

تُنفق المنصات الإلكترونية الكبرى مبالغ هائلة على التسويق الرقمي. فمجموعة بوكينغ هولدينغز أنفقت أكثر من 6 مليارات دولار على التسويق عبر الإنترنت في عام 2022، معظمها على إعلانات غوغل وحملات التسويق الرقمي. وتُعدّ هذه المنصات من أكبر المعلنين على محركات البحث عالمياً \parencite{phocuswright2022}.

في المقابل، تمتلك وكالات الأسفار التقليدية ميزانيات تسويقية محدودة جداً مقارنة بالمنصات العالمية، مما يجعلها غير قادرة على المنافسة في مجال التسويق الرقمي.

\textbf{ب. تحسين محركات البحث (SEO):}

تستثمر المنصات الإلكترونية بكثافة في تحسين ظهورها في نتائج محركات البحث. وعند البحث عن أي وجهة أو فندق أو رحلة طيران على غوغل، تهيمن المنصات الإلكترونية الكبرى على النتائج الأولى، مما يجعلها النقطة الأولى التي يصل إليها المسافر عند البحث عبر الإنترنت.

\textbf{ج. التسويق عبر وسائل التواصل الاجتماعي:}

تمتلك المنصات الإلكترونية حضوراً قوياً على منصات التواصل الاجتماعي، حيث تستخدمها للتفاعل مع العملاء ومشاركة المحتوى الملهم وإطلاق الحملات الترويجية. كما تستفيد من التسويق عبر المؤثرين (Influencer Marketing) للوصول إلى شرائح جديدة من الجمهور.

\textbf{د. برامج الولاء:}

تمتلك معظم المنصات الإلكترونية الكبرى برامج ولاء متطورة تهدف إلى الاحتفاظ بالعملاء وتشجيع تكرار الحجز. ومن أمثلة هذه البرامج: برنامج Genius من بوكينغ، وبرنامج One Key من إكسبيديا. وتقدم هذه البرامج مزايا مثل خصومات حصرية، وترقيات مجانية، وخدمات ذات أولوية.

\subsubsection{البعد السعري}

تُعتبر المنافسة السعرية من أشد أبعاد المنافسة تأثيراً على وكالات الأسفار التقليدية. وتتمتع المنصات الإلكترونية بمزايا سعرية عديدة تعود إلى:

\begin{enumerate}[label=\textbf{\arabic*.}]
\item \textbf{وفورات الحجم:} بفضل حجم أعمالها الهائل، تحصل المنصات الإلكترونية على أسعار تفاوضية أفضل من مقدمي الخدمات. فمنصة بوكينغ، التي تسجل أكثر من مليون ليلة فندقية يومياً، تتمتع بقدرة تفاوضية لا يمكن لأي وكالة تقليدية مضاهاتها.

\item \textbf{انخفاض التكاليف التشغيلية:} لا تحتاج المنصات الإلكترونية إلى شبكة مكاتب ومحلات تجارية، مما يوفر تكاليف الإيجار والتجهيز والصيانة. كما أن أتمتة العمليات تقلل من الحاجة إلى عدد كبير من الموظفين.

\item \textbf{التسعير الديناميكي:} تستخدم المنصات تقنيات التسعير الديناميكي التي تعدّل الأسعار تلقائياً بناءً على العرض والطلب وسلوك المستخدم وعوامل أخرى، مما يمكّنها من تحقيق أقصى إيرادات مع الحفاظ على تنافسية الأسعار.

\item \textbf{ضمانات أفضل الأسعار:} تقدم بعض المنصات ضمانات مطابقة السعر (Price Match Guarantee)، حيث تتعهد بتقديم أفضل سعر متاح وتعويض الفرق إذا وجد العميل سعراً أقل في مكان آخر.
\end{enumerate}

\subsubsection{البعد الجغرافي}

تمتلك المنصات الإلكترونية نطاقاً جغرافياً عالمياً يفوق بمراحل ما تستطيع وكالة الأسفار التقليدية تقديمه:

\begin{itemize}[label=\textbf{--}]
\item \textbf{التغطية العالمية:} تغطي المنصات الإلكترونية الكبرى آلاف الوجهات في جميع أنحاء العالم. فمنصة بوكينغ تقدم خيارات إقامة في أكثر من 220 دولة ومنطقة.
\item \textbf{عدم التقيد بالموقع:} يمكن للمسافر الوصول إلى المنصة والحجز من أي مكان في العالم، بينما تتقيد وكالة الأسفار التقليدية بموقعها الجغرافي.
\item \textbf{تعدد اللغات والعملات:} تدعم المنصات الإلكترونية عشرات اللغات والعملات، مما يسهل استخدامها من قبل مسافرين من مختلف الجنسيات.
\end{itemize}

\subsection{نقاط قوة وكالات الأسفار في المنافسة}

على الرغم من التفوق الواضح للمنصات الإلكترونية في العديد من الأبعاد التنافسية، تحتفظ وكالات الأسفار التقليدية بنقاط قوة مهمة يمكن أن تشكل أساساً لاستراتيجية تنافسية فعالة:

\subsubsection{الخدمة الشخصية والعلاقة الإنسانية}

تُعدّ الخدمة الشخصية والتواصل الإنساني المباشر من أبرز نقاط القوة التي تتميز بها وكالات الأسفار التقليدية. فالعديد من المسافرين يفضلون التعامل مع شخص حقيقي يمكنه فهم احتياجاتهم بعمق وتقديم نصائح مخصصة بناءً على خبرة شخصية ومعرفة متراكمة. وتبرز أهمية هذه الميزة بشكل خاص في الحالات التالية:

\begin{itemize}[label=\textbf{--}]
\item الرحلات المعقدة التي تشمل وجهات متعددة وعناصر كثيرة.
\item السفر إلى وجهات غير مألوفة أو تتطلب ترتيبات خاصة.
\item المسافرون كبار السن أو ذوو الاحتياجات الخاصة.
\item رحلات شهر العسل والمناسبات الخاصة.
\item سفرات الأعمال التي تتطلب مرونة وتنسيقاً عالياً.
\end{itemize}

\subsubsection{الخبرة والمعرفة المتخصصة}

يمتلك مستشارو السفر في الوكالات التقليدية معرفة ميدانية وخبرة متراكمة لا يمكن لخوارزميات الذكاء الاصطناعي محاكاتها بالكامل. فهم يعرفون الفنادق والوجهات من تجربة شخصية، ويمكنهم تقديم توصيات دقيقة بناءً على معرفة واقعية بجودة الخدمات والمرافق. كما يمكنهم التعامل مع المواقف الاستثنائية وتقديم حلول إبداعية لا تستطيع الأنظمة الآلية توفيرها \parencite{zeithaml2018}.

\subsubsection{الأمان والثقة}

يفضل العديد من المسافرين التعامل مع وكالة أسفار معروفة وموثوقة في مجتمعهم المحلي، خاصة عند إجراء حجوزات مالية كبيرة. فالتعامل وجهاً لوجه مع موظف في مكتب فعلي يمنح العميل شعوراً أكبر بالأمان والثقة مقارنة بالتعامل مع منصة إلكترونية. كما أن وجود مقر فعلي يمكن الرجوع إليه في حالات المشاكل أو النزاعات يُعدّ عامل اطمئنان مهماً لكثير من العملاء.

\subsubsection{الحماية القانونية}

توفر وكالات الأسفار المرخصة حماية قانونية أكبر للمسافرين في العديد من الدول. فالتشريعات المنظمة لنشاط الوكالات تفرض عليها التزامات تجاه العملاء في حالات الإلغاء أو التغيير أو المشاكل أثناء الرحلة. كما أن الضمانات المالية المطلوبة قانونياً توفر حماية لأموال العملاء في حالات الإفلاس.

\subsubsection{خدمة ما بعد البيع}

تتميز وكالات الأسفار بقدرتها على تقديم خدمة ما بعد البيع شخصية وفعالة. ففي حالات المشاكل أثناء الرحلة (تأخير رحلات، إلغاء حجوزات، مشاكل في الفندق)، يمكن للمسافر الاتصال بوكالته التي تتولى التنسيق مع مقدمي الخدمات لحل المشكلة. وهذه الخدمة تمثل قيمة حقيقية خاصة في المواقف الطارئة والأزمات.

ويوضح الجدول \ref{tab:comparison} مقارنة شاملة بين نقاط القوة والضعف لكل من وكالات الأسفار والمنصات الإلكترونية:

\begin{table}[H]
\centering
\caption{مقارنة بين نقاط القوة والضعف}
\label{tab:comparison}
\begin{tabular}{|r|r|r|}
\hline
\textbf{المعيار} & \textbf{وكالات الأسفار} & \textbf{المنصات الإلكترونية} \\
\hline
الخدمة الشخصية & قوية جداً & ضعيفة \\
\hline
التوفر الزمني & محدود & دائم (24/7) \\
\hline
النطاق الجغرافي & محلي/إقليمي & عالمي \\
\hline
الأسعار & متوسطة & تنافسية \\
\hline
المعلومات & خبرة شخصية & بيانات ضخمة \\
\hline
التكنولوجيا & محدودة & متقدمة جداً \\
\hline
الثقة & عالية محلياً & متفاوتة \\
\hline
المرونة & عالية & محدودة بالنظام \\
\hline
\end{tabular}
\end{table}


% ======================================================================
% المبحث الثاني: تأثير المنصات الإلكترونية على وكالات الأسفار
% ======================================================================
\section{المبحث الثاني: تأثير المنصات الإلكترونية على وكالات الأسفار}

\subsection{التأثير على الحصة السوقية}

\subsubsection{تراجع الحصة السوقية لوكالات الأسفار}

أدى نمو المنصات الإلكترونية إلى تآكل مستمر في الحصة السوقية لوكالات الأسفار التقليدية. وتشير الإحصائيات والتقارير الدولية إلى حجم هذا التحول:

وفقاً لتقرير فوكسرايت \parencite{phocuswright2022}، ارتفعت حصة الحجوزات عبر الإنترنت من إجمالي حجوزات السفر العالمية من أقل من 5\% في عام 2000 إلى أكثر من 65\% في عام 2022. وفي بعض الأسواق المتقدمة رقمياً مثل الولايات المتحدة والمملكة المتحدة والدول الاسكندنافية، تجاوزت النسبة 75\%.

أما في الأسواق العربية، فإن نسبة الحجز عبر الإنترنت أقل نسبياً لكنها في تزايد مستمر. وتشير التقديرات إلى أن نسبة الحجز الإلكتروني في منطقة الشرق الأوسط وشمال إفريقيا تجاوزت 40\% في عام 2023، مع توقعات بارتفاعها بشكل كبير في السنوات القادمة \parencite{euromonitor2023}.

\subsubsection{تراجع عدد وكالات الأسفار}

انعكس تراجع الحصة السوقية على عدد وكالات الأسفار العاملة في العديد من الدول. فقد شهدت أسواق مثل الولايات المتحدة وأوروبا انخفاضاً كبيراً في عدد الوكالات:

\begin{itemize}[label=\textbf{--}]
\item في الولايات المتحدة، انخفض عدد وكالات الأسفار من أكثر من 34,000 وكالة في عام 2000 إلى نحو 15,000 وكالة في عام 2022.
\item في المملكة المتحدة، أغلقت شركة توماس كوك العريقة أبوابها نهائياً في عام 2019 بعد 178 عاماً من العمل، في رمزية بالغة الدلالة على حجم التحولات في القطاع.
\item في فرنسا، تراجع عدد وكالات الأسفار بنحو 30\% خلال العشرية الأخيرة.
\end{itemize}

وقد تسارع هذا التراجع بشكل كبير بعد جائحة كوفيد-19 في عام 2020، التي شكّلت ضربة قاصمة للعديد من الوكالات التي لم تستطع تحمل التوقف شبه الكامل لحركة السفر لأشهر طويلة.

\subsubsection{إعادة التوزيع السوقي}

لم يقتصر تأثير المنصات الإلكترونية على تقليص حصة وكالات الأسفار فحسب، بل أعاد رسم خريطة التوزيع السوقي بالكامل. فقد أصبحت شركات قليلة عملاقة تهيمن على السوق العالمي لحجز السفر عبر الإنترنت:

\begin{itemize}[label=\textbf{--}]
\item مجموعة بوكينغ هولدينغز (تشمل بوكينغ دوت كوم وكاياك وأجودا): إيرادات تجاوزت 17 مليار دولار في 2022.
\item مجموعة إكسبيديا (تشمل إكسبيديا وهوتيلز دوت كوم وفي آر بي أو): إيرادات تجاوزت 12 مليار دولار في 2022.
\item مجموعة تريب دوت كوم: إيرادات تجاوزت 3 مليارات دولار في 2022.
\end{itemize}

ويوضح الجدول \ref{fig:marketshare} تطور توزيع حصص السوق بين القنوات المختلفة:

\begin{table}[H]
\centering
\caption{تطور توزيع حصص سوق حجز السفر بالنسبة المئوية (2000-2023)}
\label{fig:marketshare}
\begin{tabular}{|r|r|r|r|r|r|r|}
\hline
\textbf{القناة} & \textbf{2000} & \textbf{2005} & \textbf{2010} & \textbf{2015} & \textbf{2020} & \textbf{2023} \\
\hline
وكالات الأسفار التقليدية & 60\% & 45\% & 30\% & 22\% & 18\% & 15\% \\
\hline
مبيعات مباشرة & 35\% & 30\% & 25\% & 20\% & 22\% & 20\% \\
\hline
المنصات الإلكترونية & 5\% & 25\% & 45\% & 58\% & 60\% & 65\% \\
\hline
\end{tabular}
\end{table}

ويوضح الشكل \ref{fig:marketshare_chart} بشكل بياني حجم التحول الذي شهده سوق حجز السفر، حيث يتضح بجلاء الصعود الكبير للمنصات الإلكترونية على حساب وكالات الأسفار التقليدية:

\begin{figure}[H]
\centering
\begin{tikzpicture}
\begin{axis}[
    ybar stacked,
    bar width=18pt,
    width=0.85\textwidth,
    height=8cm,
    ylabel={النسبة المئوية (\%)},
    xtick=data,
    xticklabels={2000, 2005, 2010, 2015, 2020, 2023},
    ymin=0, ymax=105,
    legend style={at={(0.5,-0.18)}, anchor=north, legend columns=3, font=\small},
    enlarge x limits=0.12,
]
\addplot[fill=blue!60, draw=blue!70] coordinates {(1,60) (2,45) (3,30) (4,22) (5,18) (6,15)};
\addplot[fill=red!50, draw=red!60] coordinates {(1,5) (2,25) (3,45) (4,58) (5,60) (6,65)};
\addplot[fill=green!40!black, draw=green!50!black] coordinates {(1,35) (2,30) (3,25) (4,20) (5,22) (6,20)};
\legend{وكالات الأسفار, المنصات الإلكترونية, مبيعات مباشرة}
\end{axis}
\end{tikzpicture}
\caption{التطور البياني لتوزيع حصص سوق حجز السفر (2000--2023)}
\label{fig:marketshare_chart}
\end{figure}


\subsection{التأثير على نموذج الأعمال}

\subsubsection{انهيار نظام العمولات التقليدي}

لعقود طويلة، اعتمدت وكالات الأسفار بشكل أساسي على نظام العمولات كمصدر رئيسي للدخل، حيث كانت تحصل على نسبة مئوية من كل عملية بيع تتم عبرها. غير أن هذا النظام تعرض لضربات متتالية بدءاً من أواخر التسعينيات:

\begin{enumerate}[label=\textbf{\arabic*.}]
\item في عام 1995، بدأت شركات الطيران الأمريكية في تخفيض عمولات وكالات الأسفار من 10\% إلى 8\%.
\item في عام 1999، تم تخفيض العمولات مرة أخرى إلى 5\%.
\item في عام 2002، ألغت معظم شركات الطيران الأمريكية العمولات بالكامل.
\item تبعت ذلك شركات الطيران الأوروبية والعالمية بتخفيضات مماثلة.
\item اليوم، لا تدفع معظم شركات الطيران أي عمولة لوكالات الأسفار التقليدية.
\end{enumerate}

وقد اضطر هذا التحول وكالات الأسفار إلى البحث عن مصادر دخل بديلة، أبرزها فرض رسوم خدمة على العملاء، وهو ما أثار استياء بعض المسافرين ودفعهم نحو الحجز المباشر عبر الإنترنت \parencite{iata2023}.

\subsubsection{الضغط على هوامش الربح}

أدى تزايد المنافسة وتراجع العمولات إلى ضغط شديد على هوامش ربح وكالات الأسفار:

\begin{itemize}[label=\textbf{--}]
\item انخفاض إجمالي العمولات المحصّلة بسبب إلغاء عمولات شركات الطيران.
\item ضغط الأسعار الناتج عن المنافسة مع المنصات الإلكترونية.
\item ارتفاع تكاليف التشغيل (إيجارات، رواتب، تكنولوجيا).
\item صعوبة فرض رسوم خدمة عالية في ظل وجود بدائل مجانية عبر الإنترنت.
\end{itemize}

وتشير التقديرات إلى أن متوسط هامش الربح الصافي لوكالات الأسفار التقليدية انخفض من نحو 8-10\% في التسعينيات إلى 2-4\% في السنوات الأخيرة \parencite{mckinsey2022}، مما يجعل استمرارية العديد من الوكالات الصغيرة أمراً صعباً.

\subsubsection{تغير في تركيبة الإيرادات}

في مواجهة تراجع العمولات، اضطرت وكالات الأسفار إلى تنويع مصادر إيراداتها:

\begin{itemize}[label=\textbf{--}]
\item \textbf{رسوم الخدمة:} فرض رسوم على العملاء مقابل خدمات البحث والحجز والاستشارة.
\item \textbf{الخدمات الإضافية:} التوسع في بيع خدمات ذات هامش ربح أعلى مثل التأمين والتأشيرات والرحلات المخصصة.
\item \textbf{التركيز على الأسواق المتخصصة:} التحول نحو أسواق ذات قيمة مضافة عالية مثل سياحة الفخامة وسياحة الأعمال.
\item \textbf{برامج الحوافز:} الحصول على مكافآت وحوافز من منظمي الرحلات وشركات السياحة مقابل تحقيق أهداف مبيعات محددة.
\end{itemize}


\subsection{التأثير على سلوك المستهلك}

\subsubsection{تحول في أنماط البحث والحجز}

أحدثت المنصات الإلكترونية تحولاً جذرياً في الطريقة التي يبحث بها المسافرون عن خدمات السفر ويحجزونها. وتشير الدراسات إلى أن المسافر المعاصر يمر عادة بالمراحل التالية قبل إتمام الحجز \parencite{google2022}:

\begin{enumerate}[label=\textbf{\arabic*.}]
\item \textbf{مرحلة الإلهام:} يستلهم المسافر أفكار الرحلات من شبكات التواصل الاجتماعي (إنستغرام، يوتيوب، تيك توك) ومن مواقع المحتوى السياحي.
\item \textbf{مرحلة البحث:} يبحث عن معلومات مفصلة عبر محركات البحث والمنصات المتخصصة.
\item \textbf{مرحلة المقارنة:} يقارن الأسعار والخيارات عبر محركات المقارنة ومواقع التقييم.
\item \textbf{مرحلة الحجز:} يختار المنصة التي تقدم أفضل سعر وشروط ويتم الحجز إلكترونياً.
\item \textbf{مرحلة المشاركة:} يشارك تجربته عبر التقييمات ومنشورات التواصل الاجتماعي.
\end{enumerate}

ويُلاحظ أن وكالات الأسفار التقليدية غائبة بشكل كبير عن معظم هذه المراحل، مما يعني أنها تفقد الاتصال مع المسافر في أهم مراحل عملية اتخاذ القرار.

\subsubsection{ظاهرة البحث في الوكالة والحجز عبر الإنترنت}

من الظواهر اللافتة التي أفرزتها المنافسة بين الوكالات والمنصات الإلكترونية ظاهرة ``ROBO'' (Research Offline, Book Online)، أي البحث في الوكالة والحجز عبر الإنترنت. حيث يتوجه بعض المسافرين إلى وكالات الأسفار للاستفادة من خبرتها واستشارتها المجانية في اختيار الوجهة والبرنامج، ثم يقومون بالحجز عبر الإنترنت بسعر أقل. وتُعدّ هذه الظاهرة مصدر إحباط كبير لوكالات الأسفار التي تستثمر وقتاً وجهداً في تقديم الاستشارة دون تحقيق عائد.

\subsubsection{تغير في توقعات المستهلك}

رفعت المنصات الإلكترونية سقف توقعات المستهلك السياحي بشكل كبير. فأصبح المسافر المعاصر يتوقع:

\begin{itemize}[label=\textbf{--}]
\item استجابة فورية لاستفساراته في أي وقت.
\item شفافية كاملة في الأسعار والشروط.
\item إمكانية المقارنة بين خيارات متعددة بسهولة.
\item تجربة حجز سلسة وسريعة.
\item مرونة عالية في التعديل والإلغاء.
\item تأكيد فوري للحجوزات.
\end{itemize}

وتجد وكالات الأسفار التقليدية صعوبة في تلبية هذه التوقعات المرتفعة بنماذجها التشغيلية الحالية، مما يدفع المزيد من المسافرين نحو المنصات الإلكترونية \parencite{bennett2012}.


\subsection{التأثير على سوق العمل في قطاع السفر}

أثر التحول الرقمي في صناعة السفر بشكل كبير على سوق العمل في هذا القطاع:

\subsubsection{تراجع في عدد الوظائف التقليدية}

أدى إغلاق العديد من وكالات الأسفار وتقليص حجم العمالة في الوكالات المتبقية إلى فقدان أعداد كبيرة من وظائف مستشاري السفر التقليديين. ففي الولايات المتحدة وحدها، انخفض عدد العاملين في وكالات الأسفار من أكثر من 124,000 موظف في عام 2000 إلى نحو 65,000 في عام 2022.

\subsubsection{تغير في المهارات المطلوبة}

تغيرت المهارات المطلوبة للعمل في قطاع السفر بشكل جوهري. فبينما كانت المهارات التقليدية (معرفة الوجهات، إجادة أنظمة الحجز) كافية في السابق، أصبح المطلوب اليوم مجموعة أوسع من المهارات تشمل:

\begin{itemize}[label=\textbf{--}]
\item مهارات رقمية متقدمة (التسويق الرقمي، إدارة المواقع، تحليل البيانات).
\item مهارات استشارية عالية المستوى لتقديم قيمة مضافة لا توفرها المنصات الإلكترونية.
\item مهارات تواصل متميزة وقدرة على بناء علاقات طويلة الأمد مع العملاء.
\item معرفة عميقة بأسواق وقطاعات متخصصة.
\end{itemize}

\subsubsection{ظهور أدوار ووظائف جديدة}

في المقابل، أوجد التحول الرقمي أنواعاً جديدة من الوظائف في قطاع السفر، مثل:

\begin{itemize}[label=\textbf{--}]
\item مديرو المحتوى الرقمي السياحي.
\item متخصصو تجربة المستخدم في منصات السفر.
\item محللو بيانات السفر والسياحة.
\item مديرو التسويق الرقمي السياحي.
\item متخصصو تحسين محركات البحث في قطاع السفر.
\end{itemize}


\subsection{تأثير جائحة كوفيد-19}

شكّلت جائحة كوفيد-19 التي اندلعت في أوائل عام 2020 نقطة تحول حاسمة في العلاقة بين وكالات الأسفار والمنصات الإلكترونية. فقد كان لها تأثيرات مزدوجة ومتناقضة في بعض جوانبها:

\subsubsection{التأثير السلبي على وكالات الأسفار}

\begin{itemize}[label=\textbf{--}]
\item توقف شبه كامل لنشاط السفر لأشهر طويلة.
\item إغلاق عدد كبير من الوكالات بشكل نهائي بسبب عدم القدرة على تحمل التكاليف.
\item فقدان أعداد كبيرة من الموظفين ذوي الخبرة الذين انتقلوا إلى قطاعات أخرى.
\item تسريع التحول نحو الحجز الإلكتروني لدى شرائح جديدة من المسافرين.
\end{itemize}

\subsubsection{الفرص التي أتاحتها الجائحة}

في المقابل، أظهرت الجائحة قيمة وكالات الأسفار في بعض الجوانب:

\begin{itemize}[label=\textbf{--}]
\item أدركت شريحة من المسافرين أهمية التعامل مع وكالة سفر عند حدوث أزمات (إلغاء رحلات، استرداد أموال).
\item أثبتت الوكالات قدرتها على التعامل مع المواقف المعقدة وتقديم الدعم الشخصي في أوقات الأزمات.
\item ازداد الوعي بأهمية التأمين على السفر والحماية من المخاطر.
\item برز دور الوكالات في تقديم معلومات محدّثة ودقيقة حول متطلبات السفر المتغيرة (فحوصات، تطعيمات، حجر صحي).
\end{itemize}

وتشير الدراسات إلى أن شريحة من المسافرين عادت إلى التعامل مع وكالات الأسفار بعد تجارب سلبية مع المنصات الإلكترونية خلال فترة الجائحة، لا سيما فيما يتعلق بصعوبة التواصل مع خدمة العملاء واسترداد الأموال \parencite{wttc2023}.


\subsection{التأثير على سلسلة القيمة في صناعة السفر}

أحدثت المنصات الإلكترونية تحولاً جذرياً في سلسلة القيمة (Value Chain) في صناعة السفر والسياحة. ولفهم عمق هذا التأثير، يمكن تحليل التغيرات التي طرأت على كل حلقة في هذه السلسلة.

\subsubsection{إعادة هيكلة قنوات التوزيع}

كانت سلسلة التوزيع التقليدية في صناعة السفر تسير وفق مسار خطي واضح: مقدم الخدمة (شركة طيران أو فندق) $\rightarrow$ منظم الرحلات $\rightarrow$ وكالة التجزئة $\rightarrow$ المسافر. غير أن التحول الرقمي أدى إلى ظهور نماذج توزيع جديدة ومتعددة، مما أفقد الوكالات التقليدية موقعها المحوري في السلسلة \parencite{kracht2010}.

فاليوم، يمكن للمسافر الحجز من خلال عدة قنوات:

\begin{enumerate}[label=\textbf{\arabic*.}]
\item مباشرة من مقدم الخدمة عبر موقعه الإلكتروني.
\item عبر منصة إلكترونية (OTA) مثل بوكينغ أو إكسبيديا.
\item عبر محرك بحث سفر (Meta-search) ثم التوجه لأحد المصادر.
\item عبر وسائل التواصل الاجتماعي التي أصبحت تتيح الحجز المباشر.
\item عبر تطبيقات الهاتف المحمول المتخصصة.
\item عبر المساعدين الرقميين الصوتيين.
\item والخيار التقليدي: عبر وكالة أسفار مادية.
\end{enumerate}

وهذا التعدد في قنوات التوزيع يعني أن وكالة الأسفار لم تعد القناة الوحيدة أو حتى القناة الرئيسية للوصول إلى خدمات السفر.

\subsubsection{ظاهرة إلغاء الوساطة (Disintermediation)}

من أبرز التأثيرات الهيكلية للتحول الرقمي في صناعة السفر ظاهرة ``إلغاء الوساطة'' (Disintermediation)، أي الاستغناء عن الوسطاء التقليديين. وقد تجلت هذه الظاهرة في عدة مظاهر:

\begin{itemize}[label=\textbf{--}]
\item لجوء شركات الطيران إلى البيع المباشر عبر مواقعها الإلكترونية، حيث أصبحت الحجوزات المباشرة تمثل أكثر من 50\% من مبيعات بعض شركات الطيران.
\item توجه سلاسل الفنادق الكبرى نحو تشجيع الحجز المباشر من خلال تقديم أسعار وضمانات حصرية على مواقعها.
\item يقوم المسافرون أنفسهم بدور ``مستشار السفر'' من خلال البحث والتخطيط الذاتي مستفيدين من الأدوات المتاحة مجاناً على الإنترنت.
\end{itemize}

\subsubsection{ظاهرة إعادة الوساطة (Reintermediation)}

في المقابل، أدى التحول الرقمي أيضاً إلى ظاهرة ``إعادة الوساطة'' عبر ظهور وسطاء رقميين جدد (المنصات الإلكترونية). فبدلاً من اختفاء الوساطة بالكامل، تم استبدال الوسيط التقليدي (وكالة الأسفار) بوسيط رقمي (منصة إلكترونية) في كثير من الحالات. وبذلك لم تختفِ الوساطة بل تغيّر شكلها وطبيعتها.

ويطرح هذا التحول سؤالاً جوهرياً: هل ستنجح وكالات الأسفار التقليدية في إعادة تعريف دورها في سلسلة القيمة الجديدة، أم ستُزاح نهائياً لصالح الوسطاء الرقميين؟


\subsection{تحليل معمّق لسلوك المستهلك السياحي في العصر الرقمي}

يُعدّ فهم سلوك المستهلك السياحي في العصر الرقمي أمراً ضرورياً لتحليل ديناميكيات المنافسة في قطاع السفر. وقد شهد هذا السلوك تحولات عميقة يمكن تحليلها من عدة زوايا.

\subsubsection{نموذج رحلة العميل الرقمي (Digital Customer Journey)}

يمر المسافر الرقمي المعاصر بعدة مراحل قبل وأثناء وبعد الرحلة، تختلف عن المسار التقليدي الذي كان يمر حتماً عبر وكالة الأسفار:

\textbf{المرحلة الأولى - الحلم والإلهام:} يبدأ المسافر بالتفكير في وجهة الرحلة متأثراً بعدة مصادر: منشورات الأصدقاء على وسائل التواصل الاجتماعي (45\% من المسافرين)، محتوى المؤثرين على إنستغرام ويوتيوب (25\%)، مقالات المدونات السياحية (15\%)، الإعلانات الرقمية (10\%)، ومصادر أخرى (5\%).

\textbf{المرحلة الثانية - البحث والتخطيط:} يجري المسافر بحثاً مكثفاً عبر الإنترنت يشمل: محركات البحث (غوغل)، مواقع التقييم (تريب أدفايزر)، المنصات الإلكترونية (بوكينغ، إكسبيديا)، ومنتديات السفر. وتشير الدراسات إلى أن المسافر المتوسط يزور 38 موقعاً إلكترونياً مختلفاً قبل إتمام حجزه \parencite{google2022}.

\textbf{المرحلة الثالثة - المقارنة والقرار:} يقارن المسافر بين الخيارات المتاحة من حيث السعر والجودة والملاءمة. وتلعب محركات البحث عن السفر دوراً حاسماً في هذه المرحلة.

\textbf{المرحلة الرابعة - الحجز والدفع:} يتم الحجز عادة عبر القناة الأرخص أو الأكثر موثوقية من وجهة نظر المسافر.

\textbf{المرحلة الخامسة - التجربة والمشاركة:} يشارك المسافر تجربته عبر التقييمات والصور والمنشورات على وسائل التواصل الاجتماعي، مما يؤثر على قرارات مسافرين آخرين.

\subsubsection{عوامل اختيار قناة الحجز}

تتعدد العوامل التي تؤثر في اختيار المسافر لقناة الحجز (وكالة تقليدية أو منصة إلكترونية أو حجز مباشر)، ويمكن تصنيفها كالتالي:

\begin{table}[H]
\centering
\caption{عوامل اختيار قناة الحجز ودرجة تأثيرها}
\label{tab:booking_factors}
\begin{tabular}{|r|r|r|}
\hline
\textbf{العامل} & \textbf{يفضل المنصة} & \textbf{يفضل الوكالة} \\
\hline
السعر & 78\% & 22\% \\
\hline
السرعة والسهولة & 85\% & 15\% \\
\hline
التوفر الزمني & 90\% & 10\% \\
\hline
الثقة والأمان & 45\% & 55\% \\
\hline
تعقيد الرحلة & 30\% & 70\% \\
\hline
الرحلات الجماعية & 25\% & 75\% \\
\hline
الاستشارة المتخصصة & 20\% & 80\% \\
\hline
خدمة ما بعد البيع & 35\% & 65\% \\
\hline
\end{tabular}
\end{table}

\subsubsection{الفجوة بين الأجيال في سلوك الحجز}

يتباين سلوك الحجز بشكل كبير بين الأجيال المختلفة. ويمكن رصد أنماط واضحة تعكس هذا التباين:

\textbf{الجيل الصامت وجيل الطفرة (فوق 60 عاماً):} يميل هذا الجيل بشكل أكبر إلى التعامل مع وكالات الأسفار التقليدية، حيث يقدّر العلاقة الشخصية والاستشارة المباشرة. نسبة استخدام المنصات الإلكترونية لا تتجاوز 25\%.

\textbf{جيل إكس (40-60 عاماً):} يستخدم مزيجاً من القنوات التقليدية والرقمية. يلجأ إلى المنصات الإلكترونية للرحلات البسيطة وإلى الوكالات للرحلات المعقدة. نسبة الحجز الإلكتروني حوالي 55\%.

\textbf{جيل الألفية (25-40 عاماً):} يفضل بشكل واضح الحجز عبر الإنترنت ويستخدم الهاتف المحمول كأداة رئيسية. نسبة الحجز الإلكتروني تتجاوز 80\%.

\textbf{جيل زد (أقل من 25 عاماً):} شبه كامل الاعتماد على الأدوات الرقمية. يتأثر بشكل كبير بوسائل التواصل الاجتماعي. نسبة الحجز الإلكتروني تتجاوز 90\%.


\subsection{السيناريوهات المستقبلية للعلاقة بين الوكالات والمنصات}

بناءً على التحليل السابق، يمكن رسم عدة سيناريوهات مستقبلية محتملة:

\subsubsection{السيناريو الأول: الزوال التدريجي}

في هذا السيناريو، تستمر وكالات الأسفار التقليدية في التراجع حتى تختفي بشكل كامل تقريباً، كما حدث مع العديد من القطاعات الأخرى التي عصفت بها الرقمنة (مثل متاجر الموسيقى التقليدية ومكاتب الصرافة). ويفترض هذا السيناريو استمرار التطور التكنولوجي بنفس الوتيرة وعدم قدرة الوكالات على التكيف.

\subsubsection{السيناريو الثاني: التحول والتكيف}

في هذا السيناريو، تنجح شريحة من وكالات الأسفار في التحول الرقمي والتكيف مع البيئة الجديدة من خلال تبني استراتيجيات التمايز والتخصص. وتبقى الوكالات فاعلة في أسواق محددة (السفر الفاخر، سياحة الأعمال، الرحلات المعقدة)، لكن حصتها الإجمالية من السوق تبقى محدودة.

\subsubsection{السيناريو الثالث: النموذج الهجين}

يُعتبر هذا السيناريو الأكثر ترجيحاً. حيث تتطور وكالات الأسفار الناجحة نحو نموذج هجين يجمع بين الخدمة الشخصية والتواجد الرقمي القوي. وتصبح الوكالة بمثابة ``مستشار سفر متعدد القنوات'' يتفاعل مع العميل عبر المكتب الفعلي والموقع الإلكتروني والتطبيق ووسائل التواصل الاجتماعي حسب تفضيلاته.


\section*{خلاصة الفصل الثاني}

يتضح من خلال ما تم عرضه في هذا الفصل أن المنافسة بين وكالات الأسفار والمنصات الإلكترونية هي منافسة متعددة الأبعاد وغير متكافئة في كثير من الجوانب. فالمنصات الإلكترونية تتفوق بوضوح في الأبعاد التكنولوجية والتسويقية والسعرية والجغرافية، بينما تحتفظ وكالات الأسفار بتفوق في الخدمة الشخصية والمعرفة المتخصصة والثقة المحلية.

وقد أدى التحول الرقمي إلى إعادة هيكلة سلسلة القيمة في صناعة السفر، مع ظاهرتي إلغاء الوساطة وإعادة الوساطة بأشكال جديدة. كما تغيّر سلوك المستهلك السياحي بشكل جذري، مع فجوة واضحة بين الأجيال في أنماط البحث والحجز.

وبشكل عام، أدت هذه المنافسة إلى تأثيرات عميقة على وكالات الأسفار تمثلت في: تراجع حصتها السوقية، وانهيار نظام العمولات، وانخفاض هوامش الربح، وتغير سلوك المستهلك، وتراجع عدد الوكالات والوظائف في القطاع. كما أسهمت جائحة كوفيد-19 في تسريع هذه التحولات مع إتاحة بعض الفرص لإعادة إبراز قيمة الخدمة الشخصية.

وفي الفصل الثالث، سيتم التركيز على التحديات الرئيسية التي تواجه وكالات الأسفار والحلول والاستراتيجيات المقترحة للتكيف مع البيئة التنافسية الجديدة.
