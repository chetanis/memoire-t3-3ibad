% ==============================================================================
% المقدمة العامة
% ==============================================================================
\chapter*{المقدمة العامة}
\addcontentsline{toc}{chapter}{المقدمة العامة}
\markboth{المقدمة العامة}{المقدمة العامة}

\vspace{1cm}

يُعدّ قطاع السياحة والسفر من أهم القطاعات الاقتصادية في العالم، إذ يُسهم بشكل كبير في الناتج المحلي الإجمالي للعديد من الدول، ويوفر ملايين فرص العمل المباشرة وغير المباشرة. وقد شكّلت وكالات الأسفار تاريخياً الحلقة الأساسية في سلسلة التوزيع السياحي، حيث كانت تمثل الوسيط الرئيسي بين مقدمي الخدمات السياحية (شركات الطيران، الفنادق، شركات النقل) والمسافرين. غير أن هذا الدور التقليدي بدأ يتعرض لتحديات جوهرية مع بداية الألفية الثالثة، حيث أحدثت الثورة الرقمية تحولات عميقة في بنية هذا القطاع وآليات عمله.

لقد أدى الانتشار الواسع لشبكة الإنترنت وتطور تكنولوجيا المعلومات والاتصالات إلى ظهور جيل جديد من المنصات الإلكترونية المتخصصة في خدمات السفر والسياحة. فمنصات مثل بوكينغ دوت كوم (Booking.com) وإكسبيديا (Expedia) وسكاي سكانر (Skyscanner) وإير بي إن بي (Airbnb) أصبحت تقدم خدمات شاملة ومتكاملة للمسافرين، بدءاً من البحث والمقارنة وصولاً إلى الحجز والدفع الإلكتروني، وذلك على مدار الساعة ومن أي مكان في العالم. وقد استطاعت هذه المنصات أن تستحوذ على حصص سوقية متنامية، مما شكّل تهديداً حقيقياً لوجود واستمرارية وكالات الأسفار التقليدية \parencite{buhalis2020}.

إن التغيرات الجذرية التي طرأت على سلوك المستهلك السياحي تُعدّ من أبرز العوامل التي عمّقت هذه التحديات. فالمسافر المعاصر أصبح أكثر استقلالية ووعياً رقمياً، ويميل بشكل متزايد إلى التخطيط لرحلاته بنفسه عبر الإنترنت، مستفيداً من الكم الهائل من المعلومات والتقييمات المتاحة على الشبكة. وتشير الإحصائيات إلى أن أكثر من 65\% من حجوزات السفر أصبحت تتم عبر الإنترنت في العديد من الدول \parencite{phocuswright2022}، مما يعكس حجم التحول في أنماط الاستهلاك السياحي.

وفي ظل هذا الواقع الجديد، تجد وكالات الأسفار نفسها أمام معادلة صعبة: كيف يمكنها الحفاظ على مكانتها ودورها في سوق أصبح يتجه بسرعة نحو الرقمنة الشاملة؟ وما هي الاستراتيجيات التي يمكن أن تتبناها لمواجهة المنافسة الشرسة من المنصات الإلكترونية العالمية؟

\vspace{0.8cm}

\section*{إشكالية الدراسة}
\addcontentsline{toc}{section}{إشكالية الدراسة}

في ضوء التحولات الرقمية المتسارعة التي يشهدها قطاع السياحة والسفر، وبروز المنصات الإلكترونية كلاعب رئيسي في سوق خدمات السفر، يمكن صياغة الإشكالية الرئيسية لهذه الدراسة على النحو التالي:

\begin{center}
\textbf{\large ما هي أبرز التحديات التي تواجهها وكالات الأسفار في ظل المنافسة المتزايدة مع المنصات الإلكترونية، وما هي الحلول والاستراتيجيات الممكنة للتكيف مع هذا الواقع الجديد؟}
\end{center}

وتتفرع عن هذه الإشكالية الرئيسية مجموعة من الأسئلة الفرعية:

\begin{enumerate}[label=\textbf{\arabic*.}]
\item ما المقصود بوكالات الأسفار والمنصات الإلكترونية للسفر، وما هي أوجه الاختلاف بينهما؟
\item كيف أثرت المنصات الإلكترونية على النشاط التجاري لوكالات الأسفار؟
\item ما هي طبيعة المنافسة القائمة بين وكالات الأسفار والمنصات الإلكترونية؟
\item ما هي أهم التحديات التي تواجهها وكالات الأسفار في العصر الرقمي؟
\item ما هي الاستراتيجيات والحلول التي يمكن أن تعتمدها وكالات الأسفار لمواجهة هذه التحديات؟
\end{enumerate}

\vspace{0.8cm}

\section*{فرضيات الدراسة}
\addcontentsline{toc}{section}{فرضيات الدراسة}

للإجابة على الإشكالية المطروحة والأسئلة الفرعية المنبثقة عنها، تم صياغة الفرضيات التالية:

\begin{itemize}[label=\textbf{--}]
\item \textbf{الفرضية الأولى:} تسبب المنصات الإلكترونية العالمية في انخفاض نسبة الحجوزات لدى وكالات الأسفار التقليدية، وذلك من خلال تقديم بدائل أسهل وأسرع وأقل تكلفة للمسافرين.

\item \textbf{الفرضية الثانية:} تواجه وكالات الأسفار تحديات أساسية ناجمة عن ضعف مستوى الرقمنة والتحول الرقمي لديها مقارنة بالمنصات الإلكترونية، مما يُضعف قدرتها التنافسية في السوق.

\item \textbf{الفرضية الثالثة:} أدى التغيير في سلوك المستهلك نحو التعاملات الإلكترونية والرقمية إلى تراجع ملحوظ في إقبال المسافرين على وكالات الأسفار التقليدية، لصالح المنصات الإلكترونية.
\end{itemize}

\vspace{0.8cm}

\section*{أهمية الدراسة}
\addcontentsline{toc}{section}{أهمية الدراسة}

تستمد هذه الدراسة أهميتها من عدة اعتبارات:

\begin{enumerate}[label=\textbf{\arabic*.}]
\item \textbf{الأهمية العلمية:} تُسهم هذه الدراسة في إثراء الأدبيات العربية المتعلقة بموضوع التحول الرقمي في قطاع السياحة والسفر، وهو مجال لا يزال يحتاج إلى مزيد من البحث والدراسة في البيئة العربية.

\item \textbf{الأهمية العملية:} تقدم الدراسة رؤى وتوصيات عملية يمكن أن تستفيد منها وكالات الأسفار في تطوير استراتيجياتها وتحسين أدائها التنافسي في مواجهة المنصات الإلكترونية.

\item \textbf{الراهنية:} يتناول البحث موضوعاً حيوياً وراهناً يمس شريحة واسعة من المتعاملين الاقتصاديين في قطاع السياحة والسفر، خاصة في ظل التسارع الكبير في وتيرة التحول الرقمي بعد جائحة كورونا.
\end{enumerate}

\vspace{0.8cm}

\section*{أهداف الدراسة}
\addcontentsline{toc}{section}{أهداف الدراسة}

تسعى هذه الدراسة إلى تحقيق مجموعة من الأهداف:

\begin{enumerate}[label=\textbf{\arabic*.}]
\item التعرف على المفاهيم الأساسية المتعلقة بوكالات الأسفار والمنصات الإلكترونية للسفر والسياحة.
\item تحليل طبيعة المنافسة القائمة بين وكالات الأسفار والمنصات الإلكترونية وأبعادها المختلفة.
\item تحديد وتحليل أهم التحديات التي تواجهها وكالات الأسفار في العصر الرقمي.
\item رصد التغيرات في سلوك المستهلك السياحي وتأثيرها على وكالات الأسفار.
\item اقتراح حلول واستراتيجيات عملية تساعد وكالات الأسفار على التكيف والتطور في البيئة الرقمية.
\end{enumerate}

\vspace{0.8cm}

\section*{منهج الدراسة}
\addcontentsline{toc}{section}{منهج الدراسة}

اعتمدت هذه الدراسة على المنهج الوصفي التحليلي، وذلك لملاءمته لطبيعة الموضوع وأهدافه. حيث تم توظيف هذا المنهج في وصف وتحليل الظاهرة المدروسة من خلال جمع المعلومات والبيانات من مصادر متنوعة تشمل: الكتب والمراجع العلمية المتخصصة، الدراسات والأبحاث السابقة، التقارير والإحصائيات الصادرة عن المنظمات الدولية المتخصصة في السياحة والسفر، بالإضافة إلى المصادر الإلكترونية الموثوقة.

وقد تم اختيار هذا المنهج لعدة أسباب:

\begin{enumerate}[label=\textbf{\arabic*.}]
\item \textbf{طبيعة الموضوع:} إن دراسة التحديات والمنافسة في قطاع السفر تتطلب وصفاً دقيقاً للواقع الحالي وتحليلاً معمقاً للعوامل المؤثرة، وهو ما يوفره المنهج الوصفي التحليلي.
\item \textbf{تعدد مصادر البيانات:} يسمح هذا المنهج بالجمع بين مصادر متنوعة من البيانات الكمية والنوعية، مما يُثري التحليل ويعزز مصداقية النتائج.
\item \textbf{إمكانية التعميم:} يتيح المنهج الوصفي التحليلي استخلاص خلاصات وتوصيات قابلة للتطبيق في سياقات متعددة.
\end{enumerate}

وتشمل أدوات جمع البيانات المستخدمة في هذه الدراسة:

\begin{itemize}[label=\textbf{--}]
\item المسح المكتبي والتوثيقي للأدبيات العلمية المتعلقة بالموضوع.
\item تحليل التقارير الإحصائية الصادرة عن المنظمات الدولية (منظمة السياحة العالمية، مجلس السفر والسياحة العالمي).
\item مراجعة دراسات الحالة والتجارب الدولية والعربية في مجال التحول الرقمي لوكالات الأسفار.
\item تحليل البيانات والإحصائيات المنشورة حول سوق السفر الإلكتروني وحصص المنصات الرقمية.
\end{itemize}

\vspace{0.8cm}

\section*{حدود الدراسة}
\addcontentsline{toc}{section}{حدود الدراسة}

\begin{itemize}[label=\textbf{--}]
\item \textbf{الحدود الموضوعية:} تتناول الدراسة التحديات التي تواجه وكالات الأسفار في ظل المنافسة مع المنصات الإلكترونية، مع التركيز على البعد الاستراتيجي والتنافسي.
\item \textbf{الحدود الزمنية:} يغطي البحث الفترة الممتدة من بداية الألفية الثالثة (2000) حتى الوقت الحاضر (2024)، مع التركيز بشكل خاص على التطورات الأخيرة بعد جائحة كوفيد-19.
\item \textbf{الحدود المكانية:} تتناول الدراسة الظاهرة على المستوى العالمي مع إشارات خاصة إلى السياق العربي.
\end{itemize}

\vspace{0.8cm}

\section*{صعوبات الدراسة}
\addcontentsline{toc}{section}{صعوبات الدراسة}

واجه الباحث عدة صعوبات أثناء إعداد هذه الدراسة، من أبرزها:

\begin{enumerate}[label=\textbf{\arabic*.}]
\item ندرة الدراسات الأكاديمية العربية المتعلقة بشكل مباشر بموضوع المنافسة بين وكالات الأسفار والمنصات الإلكترونية.
\item سرعة التغيرات في القطاع الرقمي مما يجعل بعض البيانات والإحصائيات تفقد صلاحيتها بسرعة.
\item صعوبة الحصول على بيانات دقيقة ومحدّثة حول السوق العربي لخدمات السفر الإلكتروني.
\item تعدد الزوايا والأبعاد المرتبطة بالموضوع مما يتطلب اختيارات صعبة في تحديد نطاق البحث.
\end{enumerate}

\vspace{0.8cm}

\section*{هيكل الدراسة}
\addcontentsline{toc}{section}{هيكل الدراسة}

تم تقسيم هذه الدراسة إلى ثلاثة فصول رئيسية، بالإضافة إلى المقدمة العامة والخاتمة:

\begin{itemize}[label=\textbf{$\blacktriangleright$}]
\item \textbf{الفصل الأول: الإطار النظري لوكالات السفر والمنصات الإلكترونية.} يتناول هذا الفصل التعريف بوكالات الأسفار من حيث النشأة والتطور والأنواع، والخدمات التي تقدمها، بالإضافة إلى التعريف بالمنصات الإلكترونية ومزاياها.

\item \textbf{الفصل الثاني: المنافسة بين وكالات الأسفار والمنصات الإلكترونية.} يعالج هذا الفصل طبيعة المنافسة بين الطرفين، وتأثير المنصات الإلكترونية على النشاط التجاري لوكالات الأسفار.

\item \textbf{الفصل الثالث: التحديات والحلول المقترحة لوكالات السفر.} يركز هذا الفصل على تحديد التحديات الرئيسية التي تواجه وكالات الأسفار، واقتراح الحلول والاستراتيجيات الملائمة للتطوير والتكيف.
\end{itemize}
