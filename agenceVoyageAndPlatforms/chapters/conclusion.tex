% ==============================================================================
% الخاتمة العامة
% ==============================================================================
\chapter*{الخاتمة العامة}
\addcontentsline{toc}{chapter}{الخاتمة العامة}
\markboth{الخاتمة العامة}{الخاتمة العامة}

\vspace{1cm}

تناولت هذه الدراسة موضوعاً بالغ الأهمية والراهنية يتعلق بالتحديات التي تواجهها وكالات الأسفار التقليدية في ظل المنافسة المتزايدة مع المنصات الإلكترونية العالمية لحجز السفر. وقد سعت الدراسة إلى الإجابة على الإشكالية الرئيسية المتمثلة في: ما هي أبرز التحديات التي تواجهها وكالات الأسفار في ظل المنافسة المتزايدة مع المنصات الإلكترونية، وما هي الحلول والاستراتيجيات الممكنة للتكيف مع هذا الواقع الجديد؟

\vspace{0.8cm}

\section*{ملخص نتائج الدراسة}
\addcontentsline{toc}{section}{ملخص نتائج الدراسة}

من خلال الفصول الثلاثة التي تشكّلت منها هذه الدراسة، تم التوصل إلى مجموعة من النتائج الرئيسية:

\begin{enumerate}[label=\textbf{\arabic*.}]
\item \textbf{في ما يتعلق بالإطار النظري (الفصل الأول):} تبيّن أن وكالات الأسفار مرت بمراحل تطور عديدة منذ تأسيسها في القرن التاسع عشر، وقد عرفت عصرها الذهبي في النصف الثاني من القرن العشرين حين كانت تمثل الحلقة الأساسية في سلسلة التوزيع السياحي. كما اتضح أن المنصات الإلكترونية تتنوع في أنواعها ونماذج أعمالها، وتتمتع بمزايا تنافسية عديدة أبرزها: التوفر الدائم، والأسعار التنافسية، وسهولة الاستخدام، ووفرة المعلومات والتقييمات، واستخدام التكنولوجيا المتقدمة.

\item \textbf{في ما يتعلق بطبيعة المنافسة وتأثيراتها (الفصل الثاني):} تبيّن أن المنافسة بين وكالات الأسفار والمنصات الإلكترونية هي منافسة متعددة الأبعاد (تكنولوجي، تسويقي، سعري، جغرافي)، وغير متكافئة لصالح المنصات الإلكترونية في معظم الأبعاد. وقد أدت هذه المنافسة إلى: تراجع الحصة السوقية لوكالات الأسفار (من أكثر من 60\% إلى أقل من 15\% من إجمالي الحجوزات)، وانهيار نظام العمولات التقليدي، وانخفاض هوامش الربح، وتراجع عدد الوكالات والوظائف في القطاع.

\item \textbf{في ما يتعلق بالتحديات والحلول (الفصل الثالث):} تم تحديد مجموعة من التحديات الرئيسية تصنّف في أربع فئات: تكنولوجية (الفجوة الرقمية، سرعة التطور)، اقتصادية (تراجع الإيرادات، ارتفاع التكاليف)، مرتبطة بسلوك المستهلك (استقلالية المسافر الرقمي، تأثير الأجيال الجديدة)، وتنظيمية (عدم تكافؤ الإطار التنظيمي). كما تم اقتراح حزمة من الاستراتيجيات للتطوير تشمل: التحول الرقمي، والتمايز والتخصص، والتسويق الرقمي، وتحسين تجربة العميل، والشراكات والتحالفات، وتطوير الموارد البشرية، وتنويع مصادر الدخل.
\end{enumerate}

\vspace{0.8cm}

\section*{مناقشة الفرضيات}
\addcontentsline{toc}{section}{مناقشة الفرضيات}

في ضوء النتائج المتوصل إليها، يمكن مناقشة فرضيات الدراسة على النحو التالي:

\textbf{الفرضية الأولى:} ``تسبب المنصات الإلكترونية العالمية في انخفاض نسبة الحجوزات لدى وكالات الأسفار''.

\textbf{تم تأكيد هذه الفرضية.} فقد أظهرت البيانات والإحصائيات تراجعاً مستمراً في حصة وكالات الأسفار من إجمالي حجوزات السفر العالمية لصالح المنصات الإلكترونية. حيث انخفضت حصة الوكالات التقليدية من أكثر من 60\% في بداية الألفية إلى أقل من 15\% حالياً في بعض الأسواق. وتعود أسباب هذا الانخفاض إلى المزايا التنافسية التي تتمتع بها المنصات الإلكترونية من حيث السعر والسهولة والتوفر.

\textbf{الفرضية الثانية:} ``تواجه وكالات الأسفار تحديات أساسية بسبب ضعف الرقمنة مقارنة بالمنصات الإلكترونية''.

\textbf{تم تأكيد هذه الفرضية.} فقد اتضح أن الفجوة الرقمية تُعدّ من أخطر التحديات التي تواجه وكالات الأسفار. حيث أن غياب التواجد الرقمي الفعال وعدم تبني التقنيات الحديثة يُضعف قدرة الوكالات على المنافسة في سوق أصبح رقمياً بالدرجة الأولى. وتتجلى هذه الفجوة في مظاهر عديدة من أبرزها: غياب المواقع الإلكترونية الاحترافية، وعدم استخدام أنظمة إدارة العملاء، وضعف الحضور على وسائل التواصل الاجتماعي.

\textbf{الفرضية الثالثة:} ``أدى التغيير في سلوك المستهلك نحو الإلكتروني إلى تراجع الإقبال على وكالات الأسفار''.

\textbf{تم تأكيد هذه الفرضية.} فقد تبيّن أن تحول سلوك المستهلك السياحي نحو الرقمنة يُعدّ من العوامل الرئيسية وراء تراجع إقبال المسافرين على وكالات الأسفار التقليدية. فالمسافر المعاصر أصبح أكثر استقلالية ووعياً رقمياً، ويفضل التخطيط لرحلاته بنفسه عبر الإنترنت. كما أن تأثير الأجيال الجديدة (جيل الألفية وجيل Z)، التي تشكل الشريحة الأكبر والمتنامية من المسافرين، يُسهم في تعميق هذا التوجه.

\vspace{0.8cm}

\section*{التوصيات}
\addcontentsline{toc}{section}{التوصيات}

بناءً على النتائج المتوصل إليها، يمكن تقديم التوصيات التالية:

\subsection*{توصيات موجهة لوكالات الأسفار}

\begin{enumerate}[label=\textbf{\arabic*.}]
\item \textbf{الإسراع في التحول الرقمي:} يجب على وكالات الأسفار اعتبار التحول الرقمي أولوية قصوى، وتخصيص الموارد اللازمة لبناء تواجد رقمي قوي ومتكامل (موقع إلكتروني، تطبيق، وسائل تواصل اجتماعي).

\item \textbf{التخصص والتمايز:} بدلاً من محاولة منافسة المنصات العملاقة في جميع المجالات، يُنصح بالتركيز على أسواق وشرائح محددة يمكن تقديم قيمة مضافة حقيقية فيها.

\item \textbf{الاستثمار في الكفاءات البشرية:} تطوير مهارات الموظفين واستقطاب كفاءات جديدة في المجالات الرقمية والتسويقية.

\item \textbf{بناء شراكات وتحالفات:} التعاون مع وكالات أخرى ومقدمي خدمات لتعزيز القدرة التنافسية.

\item \textbf{تنويع مصادر الدخل:} عدم الاعتماد على مصدر دخل واحد وتطوير خدمات ذات قيمة مضافة عالية.
\end{enumerate}

\subsection*{توصيات موجهة للجهات الحكومية والتنظيمية}

\begin{enumerate}[label=\textbf{\arabic*.}]
\item \textbf{تحديث الإطار التنظيمي:} مراجعة وتحديث التشريعات المنظمة لقطاع وكالات الأسفار لتتواءم مع الواقع الرقمي الجديد.

\item \textbf{ضمان المنافسة العادلة:} إخضاع المنصات الإلكترونية العاملة في السوق المحلي لنفس المتطلبات التنظيمية المفروضة على الوكالات التقليدية.

\item \textbf{تقديم برامج دعم:} إطلاق برامج لدعم التحول الرقمي لوكالات الأسفار الصغيرة والمتوسطة (تمويل، تكوين، استشارات).

\item \textbf{تشجيع التكوين والتأهيل:} دعم برامج التكوين المهني في المجالات الرقمية لفائدة العاملين في قطاع السفر والسياحة.
\end{enumerate}

\vspace{0.8cm}

\section*{آفاق الدراسة}
\addcontentsline{toc}{section}{آفاق الدراسة}

تفتح هذه الدراسة المجال أمام عدة محاور بحثية مستقبلية:

\begin{itemize}[label=\textbf{--}]
\item إجراء دراسة ميدانية حول واقع التحول الرقمي في وكالات الأسفار في الجزائر أو في بلد عربي محدد.
\item دراسة تأثير الذكاء الاصطناعي على مستقبل وكالات الأسفار.
\item تحليل سلوك المستهلك الجزائري تجاه المنصات الإلكترونية لحجز السفر.
\item دراسة مقارنة لاستراتيجيات التكيف المعتمدة من وكالات الأسفار في دول مختلفة.
\item تقييم فعالية برامج التحول الرقمي في قطاع السفر والسياحة.
\item دراسة تأثير تقنيات الويب 3.0 والبلوك تشين على صناعة توزيع السفر.
\end{itemize}

\vspace{1.5cm}

\begin{center}
\textbf{\Large وفي الأخير، نسأل الله التوفيق والسداد، وأن يكون هذا العمل المتواضع إضافة مفيدة للمكتبة العلمية العربية في مجال السياحة والسفر.}
\end{center}
