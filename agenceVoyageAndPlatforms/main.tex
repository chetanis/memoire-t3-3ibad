% ==============================================================================
% مذكرة تخرج: تحديات وكالات الأسفار في ظل المنافسة مع المنصات الإلكترونية
% ==============================================================================
\documentclass[a4paper, 14pt]{extreport}

% ==================== الحزم الأساسية ====================
\usepackage{fontspec}
\usepackage{polyglossia}
\setmainlanguage[numerals=maghrib]{arabic}
\setotherlanguage{english}

% الخطوط العربية
\setmainfont{Amiri}[
  Path = /usr/local/texlive/2025basic/texmf-dist/fonts/truetype/public/amiri/,
  Extension = .ttf,
  UprightFont = *-Regular,
  BoldFont = *-Bold,
  ItalicFont = *-Italic,
  BoldItalicFont = *-BoldItalic,
  Script = Arabic
]
\newfontfamily\arabicfont{Amiri}[
  Path = /usr/local/texlive/2025basic/texmf-dist/fonts/truetype/public/amiri/,
  Extension = .ttf,
  UprightFont = *-Regular,
  BoldFont = *-Bold,
  ItalicFont = *-Italic,
  BoldItalicFont = *-BoldItalic,
  Script = Arabic,
  Scale = 1.1
]
\newfontfamily\arabicfontsf{Amiri}[
  Path = /usr/local/texlive/2025basic/texmf-dist/fonts/truetype/public/amiri/,
  Extension = .ttf,
  UprightFont = *-Regular,
  BoldFont = *-Bold,
  ItalicFont = *-Italic,
  BoldItalicFont = *-BoldItalic,
  Script = Arabic,
  Scale = 1.1
]

% ==================== تنسيق الصفحة ====================
\usepackage[top=2.5cm, bottom=2.5cm, right=3cm, left=2cm]{geometry}
\usepackage{setspace}
\onehalfspacing

% ==================== حزم إضافية ====================
\usepackage{graphicx}
\usepackage{float}
\usepackage{array}
\usepackage{booktabs}
\usepackage{longtable}
\usepackage{multirow}
\usepackage{enumitem}
\usepackage{titlesec}
\usepackage{fancyhdr}
\usepackage{xcolor}
\usepackage{tikz}
\usepackage{pgfplots}
\pgfplotsset{compat=1.18}
\usepackage{amssymb}
\usepackage{caption}
\usepackage{subcaption}
\usepackage{tocloft}
\usepackage{hyperref}
\usepackage[backend=bibtex, style=authoryear, sorting=nyt]{biblatex}
\addbibresource{references.bib}

% ==================== تنسيق العناوين ====================
\titleformat{\chapter}[display]
  {\normalfont\Huge\bfseries\centering}
  {\chaptertitlename\ \thechapter}{20pt}{\Huge}
\titlespacing*{\chapter}{0pt}{-20pt}{40pt}

\titleformat{\section}
  {\normalfont\Large\bfseries}{\thesection}{1em}{}
\titleformat{\subsection}
  {\normalfont\large\bfseries}{\thesubsection}{1em}{}
\titleformat{\subsubsection}
  {\normalfont\normalsize\bfseries}{\thesubsubsection}{1em}{}

% ==================== تنسيق الرأس والتذييل ====================
\pagestyle{fancy}
\fancyhf{}
\fancyhead[R]{\leftmark}
\fancyfoot[C]{\thepage}
\renewcommand{\headrulewidth}{0.5pt}
\renewcommand{\footrulewidth}{0pt}

% ==================== إعدادات الروابط ====================
\hypersetup{
  colorlinks=true,
  linkcolor=black,
  citecolor=blue!60!black,
  urlcolor=blue!60!black,
  bookmarks=true,
  bookmarksopen=true,
}

% ==================== أوامر مخصصة ====================
\newcommand{\blankpage}{\newpage\thispagestyle{empty}\mbox{}\newpage}

% ==================== إصلاح توافق bidi مع LaTeX ====================
\makeatletter
\ifdefined\UseMathForPositioningText\else
  \newcommand{\UseMathForPositioningText}{}
\fi
\ifdefined\@kernel@tabular@init\else
  \newcommand{\@kernel@tabular@init}{}
\fi
\makeatother

% ==============================================================================
% بداية المستند
% ==============================================================================
\begin{document}

% ==================== الصفحات الأولية ====================
\pagenumbering{roman}

% صفحة العنوان
% ==============================================================================
% صفحة العنوان
% ==============================================================================
\begin{titlepage}
\begin{center}
{\large \textbf{المعهد الوطني المتخصص في التكوين المهني للتسيير بالبليدة}}


\vspace{2cm}

{\large \textbf{مذكــــــــــــرة نهايــــــــــــة التكويــــــــــــن}}\\[0.3cm]
{\large \textbf{للحصول على شهادة أهلية التقني السامــي في ....................}}\\[1cm]


\vspace{2cm}

\setlength{\fboxrule}{2pt}
\fcolorbox{black}{white}{
\textbf{تحديات وكالات الأسفار في ظل}
 \textbf{المنافسة مع المنصات الإلكترونية}
}

\vspace{2cm}

{\textbf{مؤسسة الإستقبال : .................}}\\[1cm]


\vspace{2cm}
\begin{minipage}[t]{0.3\textwidth}

\begin{flushleft}
{ \textbf{إعداد المتربص :}}\\[0.3cm]
{\large \ldots \ldots \ldots}
\end{flushleft}

\end{minipage}
\hfill
\begin{minipage}[t]{0.3\textwidth}
\begin{flushright}
{ \textbf{إشراف الأستاذ :}}\\[0.3cm]
{\large \ldots \ldots \ldots}

\vspace{0.5cm}

{ \textbf{تأطير السيد :}}\\[0.3cm]
{\large \ldots \ldots \ldots}
\end{flushright}
\end{minipage}

\vfill


{\large \textbf{دفعة : 2025 / 2026}}

\end{center}
\end{titlepage}


% الإهداء
% ==============================================================================
% الإهداء
% ==============================================================================
\newpage
\thispagestyle{empty}
\vspace*{3cm}

\begin{center}
{\Huge \textbf{الإهداء}}
\end{center}

\vspace{2cm}

\begin{flushright}
\begin{minipage}{0.85\textwidth}
{\Large

إلى من علّمني أن العلم نورٌ يُضيء الدروب\ldots

\vspace{0.8cm}

إلى \textbf{والديّ الكريمين}، اللذين كانا سنداً لي في كل خطوة، وغرسا فيّ حب العلم والمثابرة، أسأل الله أن يُطيل في عمرهما ويُديم عليهما الصحة والعافية.

\vspace{0.8cm}

إلى \textbf{إخوتي وأخواتي} الذين شاركوني أفراحي وأحزاني، وكانوا خير عونٍ لي في مسيرتي العلمية.

\vspace{0.8cm}

إلى \textbf{كل أساتذتي} الذين أناروا لي طريق المعرفة، ولم يبخلوا عليّ بعلمهم ونصائحهم القيّمة.

\vspace{0.8cm}

إلى \textbf{أصدقائي وزملائي} الذين رافقوني في هذا المشوار الجامعي، وتقاسمنا معاً لحظات الجد والاجتهاد.

\vspace{0.8cm}

إلى كل من ساهم من قريب أو بعيد في إنجاز هذا العمل المتواضع\ldots

\vspace{1cm}

\textbf{أُهدي هذا العمل.}

}
\end{minipage}
\end{flushright}

\newpage


% الشكر والتقدير
% ==============================================================================
% الشكر والتقدير
% ==============================================================================
\newpage
\thispagestyle{empty}
\vspace*{2cm}

\begin{center}
{\Huge \textbf{شكر وتقدير}}
\end{center}

\vspace{1.5cm}

\begin{flushright}
\begin{minipage}{0.9\textwidth}
{\large

\textbf{الحمد لله} رب العالمين، والصلاة والسلام على أشرف المرسلين سيدنا محمد وعلى آله وصحبه أجمعين.

\vspace{0.8cm}

أتقدم بجزيل الشكر والعرفان إلى \textbf{الأستاذ(ة) المشرف(ة) \ldots} على قبوله الإشراف على هذه المذكرة، وعلى توجيهاته القيّمة ونصائحه السديدة التي كان لها الأثر الكبير في إثراء هذا العمل وإتمامه على أحسن وجه.

\vspace{0.8cm}

كما أتوجه بالشكر الجزيل إلى أعضاء \textbf{لجنة المناقشة} المحترمين على تفضلهم بقبول مناقشة هذه المذكرة وتقييمها، وعلى ملاحظاتهم وتوجيهاتهم التي ستُسهم في تحسين هذا العمل.

\vspace{0.8cm}

وأتقدم بالشكر الخالص إلى جميع \textbf{أساتذة قسم العلوم التجارية} الذين رافقونا طوال مشوارنا الجامعي، وأسهموا في تكويننا العلمي والمعرفي.

\vspace{0.8cm}

ولا يفوتني أن أشكر كل من ساعدني من قريب أو بعيد في إنجاز هذا العمل، سواء بالمعلومة أو النصيحة أو الدعم المعنوي.

\vspace{0.8cm}

\textbf{جزاكم الله جميعاً خير الجزاء.}

}
\end{minipage}
\end{flushright}

\newpage


% الملخص
% ==============================================================================
% الملخص
% ==============================================================================
\newpage
\thispagestyle{empty}

\begin{center}
{\Huge \textbf{ملخص الدراسة}}
\end{center}

\vspace{1.5cm}

\begin{flushright}
\begin{minipage}{0.95\textwidth}

تهدف هذه الدراسة إلى تسليط الضوء على التحديات التي تواجهها وكالات الأسفار التقليدية في ظل المنافسة المتزايدة مع المنصات الإلكترونية العالمية لحجز السفر. وقد شهد قطاع السياحة والسفر تحولات جذرية خلال العقدين الأخيرين بفعل الثورة الرقمية وانتشار الإنترنت، مما أدى إلى ظهور منصات إلكترونية عملاقة مثل بوكينغ وإكسبيديا وسكاي سكانر، التي أصبحت تستحوذ على حصة سوقية متنامية على حساب الوكالات التقليدية.

\vspace{0.8cm}

تنطلق الدراسة من ثلاث فرضيات أساسية: أولاً، أن المنصات الإلكترونية العالمية تسببت في انخفاض نسبة الحجوزات لدى وكالات الأسفار. ثانياً، أن وكالات الأسفار تواجه تحديات جوهرية بسبب ضعف مستوى الرقمنة مقارنة بالمنصات الإلكترونية. ثالثاً، أن التغيير في سلوك المستهلك نحو الرقمنة تسبب في تراجع الإقبال على وكالات الأسفار التقليدية.

\vspace{0.8cm}

تتكون الدراسة من ثلاثة فصول: يتناول الفصل الأول الإطار النظري لوكالات السفر والمنصات الإلكترونية من حيث التعريف والنشأة والأنواع والخدمات المقدمة. ويعالج الفصل الثاني طبيعة المنافسة بين الطرفين وتأثير المنصات الإلكترونية على أداء وكالات الأسفار. أما الفصل الثالث فيركز على التحديات الرئيسية التي تواجه الوكالات والحلول والاستراتيجيات المقترحة للتطوير والتكيف مع البيئة الرقمية الجديدة.

\vspace{0.8cm}

\textbf{الكلمات المفتاحية:} وكالات الأسفار، المنصات الإلكترونية، المنافسة الرقمية، التحول الرقمي، سلوك المستهلك، السياحة الإلكترونية، حجز السفر عبر الإنترنت.

\end{minipage}
\end{flushright}

\newpage


% فهرس المحتويات
\tableofcontents
\newpage

% قائمة الجداول
\listoftables
\newpage

% قائمة الأشكال
\listoffigures
\newpage

% ==================== المتن ====================
\pagenumbering{arabic}

% المقدمة العامة
% ==============================================================================
% المقدمة العامة
% ==============================================================================
\chapter*{المقدمة العامة}
\addcontentsline{toc}{chapter}{المقدمة العامة}
\markboth{المقدمة العامة}{المقدمة العامة}

\vspace{1cm}

يُعدّ قطاع السياحة والسفر من أهم القطاعات الاقتصادية في العالم، إذ يُسهم بشكل كبير في الناتج المحلي الإجمالي للعديد من الدول، ويوفر ملايين فرص العمل المباشرة وغير المباشرة. وقد شكّلت وكالات الأسفار تاريخياً الحلقة الأساسية في سلسلة التوزيع السياحي، حيث كانت تمثل الوسيط الرئيسي بين مقدمي الخدمات السياحية (شركات الطيران، الفنادق، شركات النقل) والمسافرين. غير أن هذا الدور التقليدي بدأ يتعرض لتحديات جوهرية مع بداية الألفية الثالثة، حيث أحدثت الثورة الرقمية تحولات عميقة في بنية هذا القطاع وآليات عمله.

لقد أدى الانتشار الواسع لشبكة الإنترنت وتطور تكنولوجيا المعلومات والاتصالات إلى ظهور جيل جديد من المنصات الإلكترونية المتخصصة في خدمات السفر والسياحة. فمنصات مثل بوكينغ دوت كوم (Booking.com) وإكسبيديا (Expedia) وسكاي سكانر (Skyscanner) وإير بي إن بي (Airbnb) أصبحت تقدم خدمات شاملة ومتكاملة للمسافرين، بدءاً من البحث والمقارنة وصولاً إلى الحجز والدفع الإلكتروني، وذلك على مدار الساعة ومن أي مكان في العالم. وقد استطاعت هذه المنصات أن تستحوذ على حصص سوقية متنامية، مما شكّل تهديداً حقيقياً لوجود واستمرارية وكالات الأسفار التقليدية \parencite{buhalis2020}.

إن التغيرات الجذرية التي طرأت على سلوك المستهلك السياحي تُعدّ من أبرز العوامل التي عمّقت هذه التحديات. فالمسافر المعاصر أصبح أكثر استقلالية ووعياً رقمياً، ويميل بشكل متزايد إلى التخطيط لرحلاته بنفسه عبر الإنترنت، مستفيداً من الكم الهائل من المعلومات والتقييمات المتاحة على الشبكة. وتشير الإحصائيات إلى أن أكثر من 65\% من حجوزات السفر أصبحت تتم عبر الإنترنت في العديد من الدول \parencite{phocuswright2022}، مما يعكس حجم التحول في أنماط الاستهلاك السياحي.

وفي ظل هذا الواقع الجديد، تجد وكالات الأسفار نفسها أمام معادلة صعبة: كيف يمكنها الحفاظ على مكانتها ودورها في سوق أصبح يتجه بسرعة نحو الرقمنة الشاملة؟ وما هي الاستراتيجيات التي يمكن أن تتبناها لمواجهة المنافسة الشرسة من المنصات الإلكترونية العالمية؟

\vspace{0.8cm}

\section*{إشكالية الدراسة}
\addcontentsline{toc}{section}{إشكالية الدراسة}

في ضوء التحولات الرقمية المتسارعة التي يشهدها قطاع السياحة والسفر، وبروز المنصات الإلكترونية كلاعب رئيسي في سوق خدمات السفر، يمكن صياغة الإشكالية الرئيسية لهذه الدراسة على النحو التالي:

\begin{center}
\textbf{\large ما هي أبرز التحديات التي تواجهها وكالات الأسفار في ظل المنافسة المتزايدة مع المنصات الإلكترونية، وما هي الحلول والاستراتيجيات الممكنة للتكيف مع هذا الواقع الجديد؟}
\end{center}

وتتفرع عن هذه الإشكالية الرئيسية مجموعة من الأسئلة الفرعية:

\begin{enumerate}[label=\textbf{\arabic*.}]
\item ما المقصود بوكالات الأسفار والمنصات الإلكترونية للسفر، وما هي أوجه الاختلاف بينهما؟
\item كيف أثرت المنصات الإلكترونية على النشاط التجاري لوكالات الأسفار؟
\item ما هي طبيعة المنافسة القائمة بين وكالات الأسفار والمنصات الإلكترونية؟
\item ما هي أهم التحديات التي تواجهها وكالات الأسفار في العصر الرقمي؟
\item ما هي الاستراتيجيات والحلول التي يمكن أن تعتمدها وكالات الأسفار لمواجهة هذه التحديات؟
\end{enumerate}

\vspace{0.8cm}

\section*{فرضيات الدراسة}
\addcontentsline{toc}{section}{فرضيات الدراسة}

للإجابة على الإشكالية المطروحة والأسئلة الفرعية المنبثقة عنها، تم صياغة الفرضيات التالية:

\begin{itemize}[label=\textbf{--}]
\item \textbf{الفرضية الأولى:} تسبب المنصات الإلكترونية العالمية في انخفاض نسبة الحجوزات لدى وكالات الأسفار التقليدية، وذلك من خلال تقديم بدائل أسهل وأسرع وأقل تكلفة للمسافرين.

\item \textbf{الفرضية الثانية:} تواجه وكالات الأسفار تحديات أساسية ناجمة عن ضعف مستوى الرقمنة والتحول الرقمي لديها مقارنة بالمنصات الإلكترونية، مما يُضعف قدرتها التنافسية في السوق.

\item \textbf{الفرضية الثالثة:} أدى التغيير في سلوك المستهلك نحو التعاملات الإلكترونية والرقمية إلى تراجع ملحوظ في إقبال المسافرين على وكالات الأسفار التقليدية، لصالح المنصات الإلكترونية.
\end{itemize}

\vspace{0.8cm}

\section*{أهمية الدراسة}
\addcontentsline{toc}{section}{أهمية الدراسة}

تستمد هذه الدراسة أهميتها من عدة اعتبارات:

\begin{enumerate}[label=\textbf{\arabic*.}]
\item \textbf{الأهمية العلمية:} تُسهم هذه الدراسة في إثراء الأدبيات العربية المتعلقة بموضوع التحول الرقمي في قطاع السياحة والسفر، وهو مجال لا يزال يحتاج إلى مزيد من البحث والدراسة في البيئة العربية.

\item \textbf{الأهمية العملية:} تقدم الدراسة رؤى وتوصيات عملية يمكن أن تستفيد منها وكالات الأسفار في تطوير استراتيجياتها وتحسين أدائها التنافسي في مواجهة المنصات الإلكترونية.

\item \textbf{الراهنية:} يتناول البحث موضوعاً حيوياً وراهناً يمس شريحة واسعة من المتعاملين الاقتصاديين في قطاع السياحة والسفر، خاصة في ظل التسارع الكبير في وتيرة التحول الرقمي بعد جائحة كورونا.
\end{enumerate}

\vspace{0.8cm}

\section*{أهداف الدراسة}
\addcontentsline{toc}{section}{أهداف الدراسة}

تسعى هذه الدراسة إلى تحقيق مجموعة من الأهداف:

\begin{enumerate}[label=\textbf{\arabic*.}]
\item التعرف على المفاهيم الأساسية المتعلقة بوكالات الأسفار والمنصات الإلكترونية للسفر والسياحة.
\item تحليل طبيعة المنافسة القائمة بين وكالات الأسفار والمنصات الإلكترونية وأبعادها المختلفة.
\item تحديد وتحليل أهم التحديات التي تواجهها وكالات الأسفار في العصر الرقمي.
\item رصد التغيرات في سلوك المستهلك السياحي وتأثيرها على وكالات الأسفار.
\item اقتراح حلول واستراتيجيات عملية تساعد وكالات الأسفار على التكيف والتطور في البيئة الرقمية.
\end{enumerate}

\vspace{0.8cm}

\section*{منهج الدراسة}
\addcontentsline{toc}{section}{منهج الدراسة}

اعتمدت هذه الدراسة على المنهج الوصفي التحليلي، وذلك لملاءمته لطبيعة الموضوع وأهدافه. حيث تم توظيف هذا المنهج في وصف وتحليل الظاهرة المدروسة من خلال جمع المعلومات والبيانات من مصادر متنوعة تشمل: الكتب والمراجع العلمية المتخصصة، الدراسات والأبحاث السابقة، التقارير والإحصائيات الصادرة عن المنظمات الدولية المتخصصة في السياحة والسفر، بالإضافة إلى المصادر الإلكترونية الموثوقة.

وقد تم اختيار هذا المنهج لعدة أسباب:

\begin{enumerate}[label=\textbf{\arabic*.}]
\item \textbf{طبيعة الموضوع:} إن دراسة التحديات والمنافسة في قطاع السفر تتطلب وصفاً دقيقاً للواقع الحالي وتحليلاً معمقاً للعوامل المؤثرة، وهو ما يوفره المنهج الوصفي التحليلي.
\item \textbf{تعدد مصادر البيانات:} يسمح هذا المنهج بالجمع بين مصادر متنوعة من البيانات الكمية والنوعية، مما يُثري التحليل ويعزز مصداقية النتائج.
\item \textbf{إمكانية التعميم:} يتيح المنهج الوصفي التحليلي استخلاص خلاصات وتوصيات قابلة للتطبيق في سياقات متعددة.
\end{enumerate}

وتشمل أدوات جمع البيانات المستخدمة في هذه الدراسة:

\begin{itemize}[label=\textbf{--}]
\item المسح المكتبي والتوثيقي للأدبيات العلمية المتعلقة بالموضوع.
\item تحليل التقارير الإحصائية الصادرة عن المنظمات الدولية (منظمة السياحة العالمية، مجلس السفر والسياحة العالمي).
\item مراجعة دراسات الحالة والتجارب الدولية والعربية في مجال التحول الرقمي لوكالات الأسفار.
\item تحليل البيانات والإحصائيات المنشورة حول سوق السفر الإلكتروني وحصص المنصات الرقمية.
\end{itemize}

\vspace{0.8cm}

\section*{حدود الدراسة}
\addcontentsline{toc}{section}{حدود الدراسة}

\begin{itemize}[label=\textbf{--}]
\item \textbf{الحدود الموضوعية:} تتناول الدراسة التحديات التي تواجه وكالات الأسفار في ظل المنافسة مع المنصات الإلكترونية، مع التركيز على البعد الاستراتيجي والتنافسي.
\item \textbf{الحدود الزمنية:} يغطي البحث الفترة الممتدة من بداية الألفية الثالثة (2000) حتى الوقت الحاضر (2024)، مع التركيز بشكل خاص على التطورات الأخيرة بعد جائحة كوفيد-19.
\item \textbf{الحدود المكانية:} تتناول الدراسة الظاهرة على المستوى العالمي مع إشارات خاصة إلى السياق العربي.
\end{itemize}

\vspace{0.8cm}

\section*{صعوبات الدراسة}
\addcontentsline{toc}{section}{صعوبات الدراسة}

واجه الباحث عدة صعوبات أثناء إعداد هذه الدراسة، من أبرزها:

\begin{enumerate}[label=\textbf{\arabic*.}]
\item ندرة الدراسات الأكاديمية العربية المتعلقة بشكل مباشر بموضوع المنافسة بين وكالات الأسفار والمنصات الإلكترونية.
\item سرعة التغيرات في القطاع الرقمي مما يجعل بعض البيانات والإحصائيات تفقد صلاحيتها بسرعة.
\item صعوبة الحصول على بيانات دقيقة ومحدّثة حول السوق العربي لخدمات السفر الإلكتروني.
\item تعدد الزوايا والأبعاد المرتبطة بالموضوع مما يتطلب اختيارات صعبة في تحديد نطاق البحث.
\end{enumerate}

\vspace{0.8cm}

\section*{هيكل الدراسة}
\addcontentsline{toc}{section}{هيكل الدراسة}

تم تقسيم هذه الدراسة إلى ثلاثة فصول رئيسية، بالإضافة إلى المقدمة العامة والخاتمة:

\begin{itemize}[label=\textbf{$\blacktriangleright$}]
\item \textbf{الفصل الأول: الإطار النظري لوكالات السفر والمنصات الإلكترونية.} يتناول هذا الفصل التعريف بوكالات الأسفار من حيث النشأة والتطور والأنواع، والخدمات التي تقدمها، بالإضافة إلى التعريف بالمنصات الإلكترونية ومزاياها.

\item \textbf{الفصل الثاني: المنافسة بين وكالات الأسفار والمنصات الإلكترونية.} يعالج هذا الفصل طبيعة المنافسة بين الطرفين، وتأثير المنصات الإلكترونية على النشاط التجاري لوكالات الأسفار.

\item \textbf{الفصل الثالث: التحديات والحلول المقترحة لوكالات السفر.} يركز هذا الفصل على تحديد التحديات الرئيسية التي تواجه وكالات الأسفار، واقتراح الحلول والاستراتيجيات الملائمة للتطوير والتكيف.
\end{itemize}


% الفصل الأول
% ==============================================================================
% الفصل الأول: الإطار النظري لوكالات السفر والمنصات الإلكترونية
% ==============================================================================
\chapter{الإطار النظري لوكالات السفر والمنصات الإلكترونية}

\section*{تمهيد}

يُعتبر قطاع السياحة والسفر من أكثر القطاعات حيوية وديناميكية في الاقتصاد العالمي، حيث يشهد تطورات مستمرة ومتسارعة تمس مختلف جوانبه ومكوناته. ومن أبرز الفاعلين في هذا القطاع نجد وكالات الأسفار التي لعبت دوراً محورياً في تنظيم وتسهيل عملية السفر والسياحة على مدى عقود طويلة. غير أن ظهور المنصات الإلكترونية أحدث تحولاً جذرياً في المشهد السياحي العالمي، مما يستوجب فهماً عميقاً لكلا الطرفين.

يهدف هذا الفصل إلى تقديم إطار نظري شامل حول وكالات الأسفار والمنصات الإلكترونية، وذلك من خلال ثلاثة مباحث: يتناول المبحث الأول تعريف وكالات الأسفار ونشأتها وأنواعها المختلفة، بينما يستعرض المبحث الثاني الخدمات التي تقدمها هذه الوكالات، أما المبحث الثالث فيركز على المنصات الإلكترونية ومزاياها في سوق السفر والسياحة.

% ======================================================================
% المبحث الأول: تعريف وكالات الأسفار، نشأتها، وأنواعها
% ======================================================================
\section{المبحث الأول: تعريف وكالات الأسفار، نشأتها، وأنواعها}

\subsection{تعريف وكالات الأسفار}

\subsubsection{المفهوم اللغوي والاصطلاحي}

تُعرّف وكالة الأسفار لغوياً بأنها المؤسسة أو الهيئة التي تتولى تنظيم وترتيب شؤون السفر والتنقل نيابة عن المسافرين. أما اصطلاحاً، فقد تعددت التعريفات التي قدمها الباحثون والمنظمات الدولية لوكالات الأسفار، ويمكن استعراض أبرزها على النحو التالي:

يُعرّف كوبر \parencite{cooper2018} وكالة الأسفار بأنها ``مؤسسة تجارية تعمل كوسيط بين مقدمي الخدمات السياحية والمسافرين، حيث تتولى بيع وتسويق المنتجات السياحية مقابل عمولة أو رسوم خدمة''. وهذا التعريف يبرز الدور الوسيط الذي تلعبه الوكالة في سلسلة القيمة السياحية.

أما منظمة السياحة العالمية \parencite{unwto2023} فتُعرّفها بأنها ``منشأة تجارية مرخصة تقدم خدمات تتعلق بتنظيم الرحلات والسفريات وحجز تذاكر النقل والإقامة وتأشيرات الدخول والتأمين على السفر وغيرها من الخدمات المرتبطة بالسفر والسياحة''. ويتميز هذا التعريف بشموليته في تحديد نطاق الخدمات التي تقدمها الوكالة.

من جهته، يُعرّف ميدلتون وفايال \parencite{middleton2009} وكالة الأسفار بأنها ``قناة توزيع رئيسية في صناعة السياحة، تربط بين العرض والطلب السياحيين من خلال تقديم مجموعة متنوعة من المنتجات والخدمات السياحية للمستهلكين النهائيين''. ويركز هذا التعريف على البعد التسويقي والتوزيعي لنشاط الوكالة.

وفي السياق العربي، يُعرّف الشمري \parencite{alshammari2018} وكالة الأسفار بأنها ``شركة تجارية متخصصة في تقديم خدمات السفر والسياحة، تعمل كحلقة وصل بين المسافر ومقدمي الخدمات السياحية، وتسعى إلى تلبية احتياجات ورغبات العملاء في مجال السفر والترفيه''.

ومن خلال استقراء التعريفات السابقة، يمكن استخلاص مجموعة من الخصائص المشتركة التي تميز وكالات الأسفار:

\begin{itemize}[label=\textbf{--}]
\item أنها مؤسسة تجارية ذات طابع خدمي.
\item تعمل كوسيط بين مقدمي الخدمات السياحية والمسافرين.
\item تقدم مجموعة متنوعة من الخدمات المتعلقة بالسفر والسياحة.
\item تحصل على مقابل مادي (عمولة أو رسوم) نظير خدماتها.
\item تخضع لتنظيم وترخيص من الجهات المختصة.
\end{itemize}

\subsubsection{التعريف القانوني لوكالات الأسفار}

تحرص معظم الدول على تنظيم نشاط وكالات الأسفار من خلال إطار قانوني وتشريعي يحدد شروط ممارسة هذا النشاط وحقوق والتزامات الأطراف المعنية. ففي الجزائر مثلاً، يُنظّم القانون رقم 99-06 المتعلق بالسياحة نشاط وكالات الأسفار والسياحة، حيث يُعرّفها بأنها ``كل شخص طبيعي أو معنوي يمارس بصفة دائمة نشاطاً يتعلق بتنظيم الرحلات والأسفار وبيع تذاكر النقل وحجز أماكن الإقامة وتقديم خدمات سياحية أخرى''.

وتشترط معظم التشريعات لممارسة نشاط وكالة الأسفار الحصول على ترخيص أو اعتماد من الجهة المختصة (عادة وزارة السياحة)، وتقديم ضمان مالي، وتوفر مؤهلات مهنية لدى المسيّرين، والتأمين على المسؤولية المهنية. وتهدف هذه الشروط إلى حماية المستهلك وضمان جودة الخدمات المقدمة.

\subsection{نشأة وتطور وكالات الأسفار}

\subsubsection{الجذور التاريخية لصناعة السفر}

تعود جذور صناعة السفر المنظم إلى حقب تاريخية بعيدة، حيث عرفت الحضارات القديمة أشكالاً مختلفة من الأسفار والرحلات. فقد كان الإغريق والرومان يسافرون لأغراض تجارية ودينية وعلاجية وترفيهية. كما عرف العالم الإسلامي رحلات استكشافية مهمة مثل رحلات ابن بطوطة في القرن الرابع عشر الميلادي. غير أن صناعة السفر بمفهومها الحديث لم تبدأ في التبلور إلا في القرن التاسع عشر.

\subsubsection{مرحلة التأسيس (القرن التاسع عشر)}

يُعتبر البريطاني توماس كوك (Thomas Cook) رائد صناعة السفر المنظم في العصر الحديث. ففي عام 1841م، نظّم كوك أول رحلة جماعية بالقطار من مدينة ليستر إلى مدينة لوبورو في إنجلترا لحضور تجمع مناهض للكحول، وشارك فيها نحو 570 شخصاً. وعلى الرغم من أن هذه الرحلة لم تكن تجارية بالمعنى الدقيق، إلا أنها شكّلت البذرة الأولى لنشاط وكالات الأسفار.

في عام 1845م، أنشأ كوك أول وكالة سفر تجارية في التاريخ، وبدأ في تنظيم رحلات سياحية بشكل احترافي. وفي عام 1855م، نظّم أول رحلة دولية من إنجلترا إلى فرنسا بمناسبة المعرض العالمي في باريس. ثم توسعت أعمال كوك لتشمل رحلات إلى مصر وفلسطين والولايات المتحدة الأمريكية. كما ابتكر كوك نظام القسائم الفندقية (Hotel Vouchers) الذي يُعدّ من أوائل أدوات الدفع المسبق في صناعة السياحة \parencite{cooper2018}.

\subsubsection{مرحلة النمو والتوسع (النصف الأول من القرن العشرين)}

شهدت بداية القرن العشرين توسعاً ملحوظاً في نشاط وكالات الأسفار، مدفوعاً بعدة عوامل من أبرزها: تطور وسائل النقل (السفن البخارية، السكك الحديدية، وبداية الطيران التجاري)، وتحسن المستوى المعيشي في الدول الصناعية، وتزايد الوعي بأهمية السياحة والترفيه. وقد ظهرت في هذه المرحلة شركات سفر كبرى مثل أمريكان إكسبريس (American Express) التي بدأت نشاطها في مجال السفر عام 1915م.

غير أن الحربين العالميتين الأولى والثانية شكّلتا عائقاً كبيراً أمام نمو هذا القطاع، حيث توقفت معظم الأنشطة السياحية خلال فترتي الحرب. ومع ذلك، أسهمت الحروب بشكل غير مباشر في تطوير البنية التحتية للنقل، خاصة الطيران، مما مهّد الطريق للطفرة السياحية في النصف الثاني من القرن العشرين.

\subsubsection{مرحلة الازدهار (1950-1990)}

تُعدّ الفترة الممتدة من خمسينيات إلى تسعينيات القرن العشرين العصر الذهبي لوكالات الأسفار. فقد شهدت هذه المرحلة ظهور السياحة الجماهيرية (Mass Tourism) بفضل تطور الطيران التجاري وانخفاض تكاليفه النسبية، وتحسن مستويات الدخل في الدول المتقدمة، وتزايد أوقات الفراغ نتيجة تقنين ساعات العمل والإجازات المدفوعة.

في هذه المرحلة، أصبحت وكالات الأسفار الوسيط الأساسي والإلزامي تقريباً بين شركات الطيران والفنادق من جهة والمسافرين من جهة أخرى. وقد تطورت أنظمة الحجز المحوسبة (Computer Reservation Systems - CRS) في السبعينيات والثمانينيات، مثل نظام سيبر (Sabre) ونظام أماديوس (Amadeus) ونظام غاليليو (Galileo)، مما ساهم في تحسين كفاءة عمل الوكالات وسرعة تقديم الخدمات \parencite{kracht2010}.

وقد شهدت هذه الفترة أيضاً ظهور مفهوم الرحلات الشاملة (Package Tours) التي تجمع بين النقل والإقامة والبرنامج السياحي في باقة واحدة بسعر موحد، وهو ما أسهم في تعزيز مكانة وكالات الأسفار كمنظم رئيسي للرحلات.

\subsubsection{مرحلة التحول الرقمي (من التسعينيات إلى اليوم)}

مع انتشار شبكة الإنترنت في التسعينيات، بدأت ملامح تحول جذري في صناعة السفر والسياحة. ففي عام 1996م، أُنشئت منصة إكسبيديا (Expedia) كواحدة من أوائل وكالات السفر عبر الإنترنت (Online Travel Agencies - OTAs). وتوالى بعدها ظهور منصات أخرى مثل بوكينغ دوت كوم (Booking.com) عام 1996م، وتريب أدفايزر (TripAdvisor) عام 2000م، وإير بي إن بي (Airbnb) عام 2008م.

وقد أدى هذا التحول إلى إعادة تشكيل سلسلة التوزيع السياحي بالكامل، حيث أصبح بإمكان المسافرين الوصول مباشرة إلى مقدمي الخدمات دون الحاجة إلى وسيط تقليدي. كما قامت شركات الطيران بتخفيض أو إلغاء العمولات المدفوعة لوكالات الأسفار، مما شكّل ضربة اقتصادية كبيرة لهذه الأخيرة \parencite{standing2014}.

ويوضح الجدول \ref{tab:evolution} أهم المحطات التاريخية في تطور وكالات الأسفار:

\begin{table}[H]
\centering
\caption{المحطات التاريخية الرئيسية في تطور وكالات الأسفار}
\label{tab:evolution}
\begin{tabular}{|r|r|}
\hline
\textbf{السنة} & \textbf{الحدث} \\
\hline
1841 & أول رحلة جماعية منظمة بواسطة توماس كوك \\
\hline
1845 & إنشاء أول وكالة سفر تجارية \\
\hline
1915 & دخول أمريكان إكسبريس مجال السفر \\
\hline
1946 & تأسيس الاتحاد الدولي للنقل الجوي (إياتا) \\
\hline
1960 & بداية عصر السياحة الجماهيرية \\
\hline
1976 & إطلاق نظام الحجز المحوسب سيبر \\
\hline
1987 & إطلاق نظام أماديوس للحجز \\
\hline
1996 & ظهور أولى وكالات السفر عبر الإنترنت \\
\hline
2000 & إطلاق تريب أدفايزر \\
\hline
2008 & إطلاق إير بي إن بي \\
\hline
2020 & تأثير جائحة كورونا على قطاع السفر \\
\hline
\end{tabular}
\end{table}


\subsection{أنواع وكالات الأسفار}

تتعدد أنواع وكالات الأسفار وتتنوع وفقاً لعدة معايير تصنيف، يمكن استعراض أبرزها على النحو التالي:

\subsubsection{التصنيف حسب طبيعة النشاط}

\textbf{أ. وكالات السفر بالتجزئة (Retail Travel Agencies):}

وهي الوكالات التي تتعامل مباشرة مع المستهلك النهائي (المسافر)، وتقوم ببيع المنتجات والخدمات السياحية التي يقدمها منظمو الرحلات وشركات الطيران والفنادق وغيرهم من مقدمي الخدمات. وتُعدّ هذه الوكالات الأكثر انتشاراً وعدداً في السوق السياحي، وتتميز بقربها من العملاء وقدرتها على تقديم خدمة شخصية ومباشرة.

تعمل وكالات التجزئة عادة بنظام العمولة، حيث تحصل على نسبة مئوية من قيمة الخدمات المباعة. وتتراوح هذه النسبة عادة بين 5\% و15\% حسب نوع الخدمة والاتفاقيات المبرمة مع مقدمي الخدمات. كما تحصل بعض الوكالات على رسوم خدمة ثابتة من العملاء مقابل عمليات البحث والحجز \parencite{middleton2009}.

\textbf{ب. منظمو الرحلات (Tour Operators):}

يُعرف منظم الرحلات بأنه المؤسسة التي تقوم بتجميع وتنسيق مختلف عناصر المنتج السياحي (النقل، الإقامة، الزيارات، الأنشطة) في باقة متكاملة تُباع بسعر شامل. ويختلف منظم الرحلات عن وكالة التجزئة في كونه يشتري الخدمات بالجملة من مقدمي الخدمات ثم يعيد بيعها بعد تجميعها في منتج سياحي متكامل.

يتحمل منظم الرحلات مخاطر مالية أكبر من وكالة التجزئة، حيث يلتزم بشراء كميات كبيرة من المقاعد والغرف الفندقية مسبقاً، بغض النظر عما إذا كان سيتمكن من بيعها جميعاً. وفي المقابل، يحقق هوامش ربح أعلى بفضل الشراء بالجملة والأسعار التفضيلية التي يحصل عليها.

\textbf{ج. وكالات السفر بالجملة (Wholesale Travel Agencies):}

تعمل هذه الوكالات كوسيط بين مقدمي الخدمات السياحية ووكالات التجزئة، حيث تشتري كميات كبيرة من المنتجات السياحية بأسعار تفضيلية وتعيد بيعها لوكالات التجزئة. ولا تتعامل عادة مع المستهلك النهائي بشكل مباشر.

\subsubsection{التصنيف حسب التخصص}

\textbf{أ. وكالات الأسفار العامة:}

تقدم هذه الوكالات مجموعة واسعة ومتنوعة من الخدمات السياحية دون التخصص في نوع معين. وتستهدف شرائح واسعة من المسافرين بمختلف احتياجاتهم وميزانياتهم.

\textbf{ب. وكالات الأسفار المتخصصة:}

تركز هذه الوكالات على نوع محدد من السفر أو على شريحة معينة من العملاء. ومن أمثلة التخصصات:

\begin{itemize}[label=\textbf{--}]
\item \textbf{سياحة الأعمال:} تتخصص في تنظيم سفرات رجال الأعمال والمؤتمرات والمعارض.
\item \textbf{السياحة الدينية:} تتخصص في تنظيم رحلات الحج والعمرة وزيارة الأماكن المقدسة.
\item \textbf{سياحة المغامرات:} تتخصص في تنظيم رحلات المغامرة والأنشطة الرياضية في الطبيعة.
\item \textbf{السياحة الفاخرة:} تستهدف الشريحة الراقية من المسافرين وتقدم خدمات عالية الجودة.
\item \textbf{سياحة العلاج:} تتخصص في تنظيم رحلات العلاج والاستشفاء.
\end{itemize}

\subsubsection{التصنيف حسب نمط التشغيل}

\textbf{أ. الوكالات المستقلة:}

وهي وكالات مملوكة ومدارة بشكل مستقل، تعمل بصورة فردية دون الارتباط بشبكة أو سلسلة. تتميز بمرونة أكبر في اتخاذ القرارات لكنها قد تعاني من محدودية الموارد والقدرة التفاوضية.

\textbf{ب. سلاسل وكالات الأسفار:}

وهي مجموعة وكالات تعمل تحت علامة تجارية واحدة وتتبع نظام إدارة موحد. تستفيد من وفورات الحجم والقدرة التفاوضية الأكبر مع مقدمي الخدمات، ومن الدعم التسويقي والتكنولوجي المشترك.

\textbf{ج. وكالات الامتياز (Franchise):}

تعمل وفق نظام الامتياز التجاري، حيث تحصل الوكالة على حق استخدام العلامة التجارية والنظام التشغيلي مقابل رسوم ونسبة من الأرباح. تجمع بين استقلالية الملكية ومزايا الانتماء إلى شبكة كبيرة.

\subsubsection{التصنيف حسب الوسيلة}

\textbf{أ. الوكالات التقليدية (Brick-and-Mortar):}

تمارس نشاطها من خلال مكاتب ومحلات تجارية فعلية يتوجه إليها العملاء شخصياً. وتتميز بالتواصل المباشر والشخصي مع العملاء، لكنها تتطلب تكاليف تشغيلية مرتفعة (إيجار، موظفين، تجهيزات).

\textbf{ب. وكالات السفر عبر الإنترنت (Online Travel Agencies - OTAs):}

تمارس نشاطها بالكامل عبر الإنترنت من خلال مواقع وتطبيقات إلكترونية. وتتميز بانخفاض التكاليف التشغيلية والقدرة على الوصول إلى قاعدة عملاء واسعة. ومن أبرز الأمثلة: بوكينغ وإكسبيديا وتريفاغو.

\textbf{ج. الوكالات الهجينة (Click-and-Mortar):}

تجمع بين التواجد المادي الفعلي والتواجد الرقمي عبر الإنترنت، مما يتيح لها الاستفادة من مزايا كلا النموذجين. وتتجه العديد من الوكالات التقليدية نحو هذا النموذج كاستراتيجية للتكيف مع التطورات الرقمية.

ويوضح الجدول \ref{fig:types} تصنيف أنواع وكالات الأسفار وفقاً للمعايير المختلفة:

\begin{table}[H]
\centering
\caption{تصنيف أنواع وكالات الأسفار وفقاً للمعايير المختلفة}
\label{fig:types}
\begin{tabular}{|r|r|r|}
\hline
\textbf{معيار التصنيف} & \textbf{النوع الأول} & \textbf{النوع الثاني} \\
\hline
حسب النشاط & وكالات التجزئة & وكالات الجملة \\
\hline
حسب التخصص & وكالات عامة & وكالات متخصصة \\
\hline
حسب الوسيلة & وكالات تقليدية & وكالات إلكترونية \\
\hline
حسب نمط التشغيل & وكالات مستقلة & سلاسل وكالات / امتياز \\
\hline
\end{tabular}
\end{table}


% ======================================================================
% المبحث الثاني: الخدمات التي تقدمها وكالات الأسفار
% ======================================================================
\section{المبحث الثاني: الخدمات التي تقدمها وكالات الأسفار}

تقدم وكالات الأسفار مجموعة واسعة ومتنوعة من الخدمات التي تلبي مختلف احتياجات المسافرين. وتتراوح هذه الخدمات بين الخدمات الأساسية التي تمثل جوهر نشاط الوكالة، والخدمات المساعدة والتكميلية التي تضيف قيمة للعميل وتعزز تجربة السفر. ويمكن تصنيف هذه الخدمات على النحو التالي:

\subsection{خدمات النقل والحجز}

\subsubsection{حجز تذاكر الطيران}

تُعدّ خدمة حجز تذاكر الطيران من أقدم وأهم الخدمات التي تقدمها وكالات الأسفار. وتشمل هذه الخدمة البحث عن أفضل الرحلات والأسعار، وإجراء الحجوزات، وإصدار التذاكر، وإدارة التغييرات والإلغاءات. وتستخدم الوكالات أنظمة الحجز العالمية (GDS) مثل أماديوس وسيبر وغاليليو للوصول إلى قواعد بيانات شركات الطيران وإجراء الحجوزات بشكل فوري.

وقد كانت وكالات الأسفار تاريخياً القناة الرئيسية لتوزيع تذاكر الطيران، حيث كانت تحصل على عمولات تتراوح بين 7\% و10\% من قيمة التذكرة. غير أن هذا النموذج تعرض لضغوط كبيرة منذ أواخر التسعينيات، حيث قامت معظم شركات الطيران بتخفيض العمولات تدريجياً ثم إلغائها في كثير من الحالات \parencite{iata2023}، مما اضطر الوكالات إلى التحول نحو فرض رسوم خدمة على العملاء.

\subsubsection{حجز وسائل النقل الأخرى}

بالإضافة إلى تذاكر الطيران، تقدم وكالات الأسفار خدمات حجز وسائل النقل الأخرى التي تشمل:

\begin{itemize}[label=\textbf{--}]
\item حجز تذاكر القطارات والحافلات السياحية.
\item تأجير السيارات في الوجهات السياحية.
\item حجز الرحلات البحرية والعبّارات.
\item تنظيم خدمات النقل من وإلى المطارات.
\item حجز سيارات خاصة مع سائق للتنقلات المحلية.
\end{itemize}

\subsection{خدمات الإقامة}

تتولى وكالات الأسفار حجز مختلف أنواع الإقامة للمسافرين، بما في ذلك الفنادق بمختلف فئاتها (من النجمة الواحدة إلى الخمس نجوم)، والشقق المفروشة، والمنتجعات السياحية، وبيوت الشباب، والفيلات السياحية. وتمتلك الوكالات عادة اتفاقيات مع سلاسل فندقية وفنادق مستقلة تحصل بموجبها على أسعار تفضيلية يمكنها تمريرها للعملاء.

وتتميز الوكالات في هذا المجال بقدرتها على تقديم نصائح موثوقة بشأن اختيار مكان الإقامة الأنسب بناءً على معرفتها المباشرة بالفنادق والوجهات، وهو ما يُعرف بالـ``المعرفة الميدانية'' التي لا تتوفر عادة لدى المنصات الإلكترونية بنفس العمق والدقة.

\subsection{تنظيم الرحلات والبرامج السياحية}

تُعتبر خدمة تنظيم الرحلات والبرامج السياحية من أبرز الخدمات المميزة لوكالات الأسفار، وتشمل:

\subsubsection{الرحلات الشاملة (الباقات السياحية)}

يقوم منظمو الرحلات بتصميم باقات سياحية متكاملة تجمع بين مختلف عناصر الرحلة في منتج واحد بسعر شامل. وتتضمن هذه الباقات عادةً: تذاكر الطيران ذهاباً وإياباً، والإقامة الفندقية، والنقل المحلي، والجولات السياحية المصحوبة بمرشد، والوجبات (كلياً أو جزئياً)، والتأمين على السفر.

وتوفر الباقات السياحية للمسافر مزايا عديدة أبرزها: سهولة التخطيط وتوفير الوقت والجهد، والحصول على سعر إجمالي أقل مما لو تم حجز كل عنصر على حدة، والاستفادة من خبرة المنظم في اختيار أفضل الخيارات، والحماية القانونية في حالات الإلغاء أو المشاكل أثناء الرحلة \parencite{cooper2018}.

\subsubsection{الرحلات المُصممة حسب الطلب}

تقدم بعض الوكالات خدمة تصميم رحلات مخصصة وفقاً لرغبات واحتياجات العميل الخاصة. وتتطلب هذه الخدمة مهارة ومعرفة عالية من موظفي الوكالة، حيث يتم بناء البرنامج من الصفر بناءً على تفضيلات العميل من حيث الوجهة والمدة والميزانية والأنشطة المفضلة.

\subsection{الخدمات الإدارية والاستشارية}

\subsubsection{التأشيرات والوثائق}

تقدم وكالات الأسفار خدمات مهمة تتعلق بالإجراءات الإدارية للسفر، من أبرزها:

\begin{itemize}[label=\textbf{--}]
\item المساعدة في استخراج التأشيرات للدول التي تتطلب ذلك.
\item تقديم المعلومات حول متطلبات الدخول لمختلف الدول.
\item المساعدة في إعداد ملفات طلب التأشيرة.
\item متابعة طلبات التأشيرات مع السفارات والقنصليات.
\item تقديم النصح بشأن جوازات السفر ومدة صلاحيتها.
\end{itemize}

\subsubsection{التأمين على السفر}

توفر الوكالات لعملائها مختلف أنواع التأمين المرتبط بالسفر، بما في ذلك التأمين الصحي، وتأمين إلغاء الرحلة، وتأمين الأمتعة، والتأمين ضد الحوادث. وتتعامل الوكالات مع شركات تأمين متخصصة وتقوم ببيع منتجاتها التأمينية كجزء من الخدمة الشاملة.

\subsubsection{الاستشارات والنصائح}

تقدم وكالات الأسفار خدمة استشارية ذات قيمة عالية تشمل: تقديم نصائح حول أفضل الوجهات والأوقات المناسبة للسفر، والمعلومات حول المناخ والعادات والتقاليد المحلية في الوجهات المختلفة، والتوصيات بشأن المطاعم والأماكن السياحية والأنشطة الترفيهية، والإرشادات الأمنية والصحية.

وتمثل هذه الخدمة الاستشارية إحدى أهم نقاط القوة التي تتميز بها وكالات الأسفار التقليدية مقارنة بالمنصات الإلكترونية، حيث تعتمد على الخبرة الشخصية والمعرفة المتراكمة لموظفي الوكالة \parencite{zeithaml2018}.

\subsection{خدمات سياحة الأعمال}

تخصصت العديد من وكالات الأسفار في خدمة قطاع الأعمال والمؤسسات، وتشمل خدماتها في هذا المجال:

\begin{itemize}[label=\textbf{--}]
\item إدارة سفرات رجال الأعمال والموظفين.
\item تنظيم المؤتمرات والندوات والمعارض.
\item تنظيم الحوافز السياحية للشركات (Incentive Travel).
\item إعداد تقارير مفصلة عن نفقات السفر.
\item التفاوض على اتفاقيات أسعار خاصة مع مقدمي الخدمات.
\end{itemize}

وتُعدّ سياحة الأعمال من أهم مصادر الدخل لوكالات الأسفار، حيث تتميز بحجم إنفاق مرتفع ومعدل تكرار عالٍ.

\subsection{خدمات الحج والعمرة}

في العالم العربي والإسلامي، تمثل خدمات الحج والعمرة جزءاً مهماً من نشاط وكالات الأسفار. وتشمل هذه الخدمات: تنظيم رحلات الحج والعمرة بمختلف فئاتها، وحجز الفنادق القريبة من الحرمين الشريفين، وتوفير خدمات النقل والإعاشة، وتقديم الإرشاد الديني والمناسكي، والمساعدة في استخراج تأشيرات الحج والعمرة.

ويوضح الجدول \ref{tab:services} ملخصاً لأهم الخدمات التي تقدمها وكالات الأسفار:

\begin{table}[H]
\centering
\caption{ملخص الخدمات الرئيسية لوكالات الأسفار}
\label{tab:services}
\begin{tabular}{|r|r|}
\hline
\textbf{فئة الخدمة} & \textbf{الخدمات الفرعية} \\
\hline
النقل والحجز & تذاكر طيران، قطارات، تأجير سيارات \\
\hline
الإقامة & فنادق، منتجعات، شقق مفروشة \\
\hline
تنظيم الرحلات & باقات شاملة، رحلات مخصصة \\
\hline
خدمات إدارية & تأشيرات، تأمين، وثائق \\
\hline
استشارات & نصائح سفر، معلومات وجهات \\
\hline
أعمال & مؤتمرات، حوافز، إدارة سفر \\
\hline
دينية & حج، عمرة، زيارات دينية \\
\hline
\end{tabular}
\end{table}


% ======================================================================
% المبحث الثالث: المنصات الإلكترونية ومزاياها
% ======================================================================
\section{المبحث الثالث: المنصات الإلكترونية ومزاياها}

\subsection{تعريف المنصات الإلكترونية للسفر}

\subsubsection{المفهوم والتعريف}

المنصات الإلكترونية للسفر هي مواقع ويب وتطبيقات رقمية متخصصة تتيح للمستخدمين البحث عن خدمات السفر والسياحة ومقارنتها وحجزها ودفع ثمنها عبر الإنترنت. وقد عرّف بوهاليس \parencite{buhalis2020} هذه المنصات بأنها ``أنظمة تكنولوجية متكاملة تستخدم الإنترنت كقناة رئيسية للتواصل مع المستهلكين وتقديم خدمات السفر والسياحة بشكل رقمي بالكامل''.

ويمكن تعريف المنصات الإلكترونية للسفر بشكل أكثر شمولية بأنها: بيئات رقمية تفاعلية تجمع بين مقدمي خدمات السفر والمسافرين في فضاء إلكتروني واحد، وتوفر أدوات للبحث والمقارنة والحجز والدفع والتقييم، مع تقديم تجربة مستخدم سلسة ومتكاملة على مدار الساعة ومن أي مكان.

\subsubsection{أنواع المنصات الإلكترونية للسفر}

تتنوع المنصات الإلكترونية للسفر من حيث نموذج العمل والخدمات المقدمة، ويمكن تصنيفها إلى عدة أنواع رئيسية:

\textbf{أ. وكالات السفر عبر الإنترنت (OTAs):}

وهي منصات تعمل كوسيط رقمي بين مقدمي الخدمات والمسافرين، وتحقق إيراداتها من العمولات أو الفرق بين سعر الشراء وسعر البيع. ومن أبرز هذه المنصات:

\begin{itemize}[label=\textbf{--}]
\item \textbf{بوكينغ دوت كوم (Booking.com):} تأسست عام 1996 في هولندا، وتُعدّ أكبر منصة لحجز الفنادق في العالم، بأكثر من 28 مليون وحدة إقامة مُدرجة في أكثر من 220 دولة ومنطقة. تعمل بنظام العمولة حيث تحصل على نسبة تتراوح بين 15\% و25\% من قيمة الحجز \parencite{booking2023}.

\item \textbf{إكسبيديا (Expedia):} تأسست عام 1996 كفرع لشركة مايكروسوفت، وتُقدم خدمات شاملة تشمل حجز الرحلات والفنادق وتأجير السيارات والأنشطة السياحية. تمتلك مجموعة إكسبيديا عدة علامات تجارية منها هوتيلز دوت كوم وأوربيتز وتريفاغو.

\item \textbf{تريب دوت كوم (Trip.com):} منصة صينية عملاقة تخدم أكثر من 400 مليون مستخدم حول العالم، وتُعدّ من أكبر منصات السفر في آسيا.
\end{itemize}

\textbf{ب. محركات البحث عن السفر (Meta-search Engines):}

وهي منصات لا تبيع خدمات السفر مباشرة، بل تقوم بتجميع ومقارنة الأسعار من مصادر متعددة (وكالات سفر إلكترونية ومواقع شركات الطيران والفنادق)، ثم توجّه المستخدم إلى الموقع المقدم للخدمة لإتمام الحجز. ومن أبرز هذه المحركات:

\begin{itemize}[label=\textbf{--}]
\item \textbf{سكاي سكانر (Skyscanner):} متخصص في مقارنة أسعار تذاكر الطيران، مع توفره أيضاً على خدمات مقارنة الفنادق وتأجير السيارات.
\item \textbf{غوغل فلايتس (Google Flights):} خدمة من غوغل لمقارنة أسعار الرحلات الجوية.
\item \textbf{تريفاغو (Trivago):} متخصص في مقارنة أسعار الفنادق من مصادر متعددة.
\item \textbf{كاياك (Kayak):} يقارن أسعار الرحلات والفنادق وتأجير السيارات.
\end{itemize}

\textbf{ج. منصات الاقتصاد التشاركي:}

وهي منصات تربط بين أصحاب الممتلكات (منازل، شقق، غرف) والمسافرين الباحثين عن إقامة، وتمثل نموذجاً جديداً في صناعة السياحة يُعرف بالاقتصاد التشاركي. ومن أبرزها:

\begin{itemize}[label=\textbf{--}]
\item \textbf{إير بي إن بي (Airbnb):} تأسست عام 2008 وأصبحت من أكبر مقدمي خدمات الإقامة في العالم بأكثر من 7 ملايين مسكن مُدرج في أكثر من 220 دولة.
\item \textbf{في آر بي أو (VRBO):} متخصصة في تأجير المنازل والفيلات لقضاء العطلات.
\end{itemize}

\textbf{د. منصات التقييم والمراجعات:}

وهي منصات تتيح للمسافرين مشاركة تجاربهم وتقييماتهم للفنادق والمطاعم والوجهات السياحية، وتؤثر بشكل كبير في قرارات المسافرين. ومن أبرزها:

\begin{itemize}[label=\textbf{--}]
\item \textbf{تريب أدفايزر (TripAdvisor):} أكبر منصة للمراجعات السياحية في العالم بأكثر من مليار تقييم ومراجعة.
\item \textbf{يلب (Yelp):} تتخصص في تقييم المطاعم والخدمات المحلية.
\end{itemize}

\subsection{مزايا المنصات الإلكترونية}

تتمتع المنصات الإلكترونية للسفر بمجموعة واسعة من المزايا التي جعلتها تستقطب أعداداً متزايدة من المسافرين على حساب وكالات الأسفار التقليدية. ويمكن تصنيف هذه المزايا في عدة فئات:

\subsubsection{مزايا مرتبطة بسهولة الاستخدام والوصول}

\textbf{أ. التوفر على مدار الساعة:}

تتيح المنصات الإلكترونية للمسافرين إجراء عمليات البحث والحجز في أي وقت يشاؤون، دون التقيد بمواعيد العمل الرسمية. وهذه الميزة ذات أهمية خاصة في عالم معولم يتعامل فيه المسافرون مع مناطق زمنية مختلفة. فبينما تعمل وكالات الأسفار التقليدية عادة خلال ساعات محددة (غالباً من الثامنة صباحاً إلى الخامسة مساءً)، فإن المنصات الإلكترونية متاحة 24 ساعة في اليوم و7 أيام في الأسبوع و365 يوماً في السنة \parencite{xiang2015}.

\textbf{ب. الوصول من أي مكان:}

يمكن للمسافر الوصول إلى المنصات الإلكترونية من أي مكان في العالم عبر جهاز كمبيوتر أو هاتف ذكي أو جهاز لوحي متصل بالإنترنت. وقد ساهم انتشار الهواتف الذكية في تعزيز هذه الميزة بشكل كبير، حيث أصبح أكثر من 70\% من عمليات البحث عن السفر تتم عبر الأجهزة المحمولة \parencite{google2022}.

\textbf{ج. سهولة واجهة المستخدم:}

تستثمر المنصات الإلكترونية الكبرى مبالغ ضخمة في تصميم واجهات مستخدم سهلة وبديهية ومريحة بصرياً، مما يجعل عملية البحث والحجز تجربة سلسة وممتعة. وتتضمن هذه الواجهات عادة أدوات بحث متقدمة وفلاتر دقيقة وخرائط تفاعلية وصور عالية الجودة.

\subsubsection{مزايا مرتبطة بالمعلومات والشفافية}

\textbf{أ. وفرة المعلومات:}

توفر المنصات الإلكترونية كماً هائلاً من المعلومات المفصلة حول الرحلات والفنادق والوجهات السياحية. وتشمل هذه المعلومات: أوصاف تفصيلية للخدمات والمرافق، وصور فوتوغرافية عالية الدقة، ومقاطع فيديو، وخرائط تفاعلية، ومعلومات عن الموقع والمسافات. وتساعد هذه المعلومات المسافر على اتخاذ قرار مستنير دون الحاجة إلى زيارة وكالة سفر.

\textbf{ب. تقييمات ومراجعات المستخدمين:}

تُعدّ تقييمات ومراجعات المستخدمين السابقين من أهم المزايا التي تقدمها المنصات الإلكترونية. فمنصة مثل تريب أدفايزر تضم أكثر من مليار مراجعة وتقييم من مسافرين حقيقيين، مما يوفر للمسافر المحتمل رؤية واقعية وغير متحيزة عن جودة الخدمات. وتشير الدراسات إلى أن أكثر من 80\% من المسافرين يقرؤون التقييمات عبر الإنترنت قبل اتخاذ قرار الحجز \parencite{amaro2015}.

\textbf{ج. الشفافية في الأسعار:}

تتيح المنصات الإلكترونية للمسافرين مقارنة الأسعار من مصادر متعددة بسهولة وسرعة. فمحركات البحث عن السفر مثل سكاي سكانر وكاياك تعرض أسعار نفس الرحلة أو الفندق من عشرات المصادر المختلفة، مما يمكّن المسافر من اختيار أفضل سعر. وقد أسهمت هذه الشفافية في تكثيف المنافسة السعرية في سوق السفر.

\subsubsection{مزايا مرتبطة بالتكلفة}

\textbf{أ. أسعار تنافسية:}

بفضل انخفاض تكاليفها التشغيلية مقارنة بالوكالات التقليدية (عدم الحاجة إلى مكاتب ومحلات تجارية ومقابل أقل للموظفين)، تستطيع المنصات الإلكترونية تقديم أسعار أكثر تنافسية للمسافرين. كما أن المنافسة الشديدة بين المنصات تدفعها إلى تقديم عروض وخصومات مستمرة. وتشير الدراسات إلى أن الأسعار عبر الإنترنت تكون أقل بنسبة 10\% إلى 30\% مقارنة بالأسعار المقدمة في وكالات الأسفار التقليدية في كثير من الحالات \parencite{statista2023}.

\textbf{ب. عروض وخصومات حصرية:}

تقدم المنصات الإلكترونية بشكل مستمر عروضاً ترويجية وخصومات خاصة لجذب العملاء والاحتفاظ بهم. وتشمل هذه العروض: خصومات الحجز المبكر، وعروض اللحظة الأخيرة، وبرامج الولاء والمكافآت، وعروض خاصة لمستخدمي التطبيق، وباقات مخفضة عند حجز عدة خدمات معاً.

\textbf{ج. عدم وجود رسوم خدمة:}

على عكس العديد من وكالات الأسفار التقليدية التي أصبحت تفرض رسوم خدمة على العملاء (بعد تخفيض أو إلغاء العمولات من شركات الطيران)، فإن معظم المنصات الإلكترونية لا تفرض رسوماً إضافية على المسافرين، وتحقق إيراداتها بشكل أساسي من العمولات التي تحصل عليها من مقدمي الخدمات.

\subsubsection{مزايا مرتبطة بالتكنولوجيا والابتكار}

\textbf{أ. التخصيص والتوصيات الذكية:}

تستخدم المنصات الإلكترونية تقنيات الذكاء الاصطناعي وتحليل البيانات الضخمة لتقديم توصيات مخصصة لكل مستخدم بناءً على سلوكه السابق وتفضيلاته. فعلى سبيل المثال، تقوم منصة بوكينغ بتحليل أنماط البحث والحجز السابقة للمستخدم لتقديم اقتراحات تتوافق مع اهتماماته، مما يحسّن تجربة المستخدم ويزيد من احتمالية إتمام الحجز \parencite{law2014}.

\textbf{ب. تطبيقات الهاتف المحمول:}

توفر المنصات الإلكترونية الكبرى تطبيقات متطورة للهواتف الذكية تتيح للمسافرين إجراء الحجوزات والاطلاع على تفاصيل رحلاتهم والحصول على إشعارات فورية وبطاقات صعود رقمية وغيرها من الخدمات أثناء التنقل. وقد أصبحت التطبيقات قناة أساسية للتفاعل مع المسافرين، حيث تمثل أكثر من 50\% من حجوزات بعض المنصات.

\textbf{ج. المرونة في الإدارة:}

تتيح المنصات الإلكترونية للمسافرين إدارة حجوزاتهم بمرونة عالية، بما في ذلك تعديل التواريخ وتغيير الخيارات وإلغاء الحجوزات، وذلك بضغطات بسيطة على الشاشة دون الحاجة إلى الاتصال أو زيارة مكتب. كما توفر أنظمة إشعارات تُبلّغ المسافر بأي تغييرات أو تحديثات تتعلق برحلته.

ويوضح الجدول \ref{fig:advantages} مقارنة بين أبرز مزايا المنصات الإلكترونية ووكالات الأسفار التقليدية:

\begin{table}[H]
\centering
\caption{مقارنة بين مزايا المنصات الإلكترونية ووكالات الأسفار التقليدية (من 10)}
\label{fig:advantages}
\begin{tabular}{|r|r|r|}
\hline
\textbf{المعيار} & \textbf{المنصات الإلكترونية} & \textbf{وكالات الأسفار التقليدية} \\
\hline
السعر & 9/10 & 6/10 \\
\hline
السهولة & 9/10 & 5/10 \\
\hline
التوفر & 10/10 & 4/10 \\
\hline
المعلومات & 9/10 & 7/10 \\
\hline
التخصيص & 8/10 & 7/10 \\
\hline
الخدمة الشخصية & 4/10 & 9/10 \\
\hline
\end{tabular}
\end{table}

ويوضح الشكل \ref{fig:comparison_chart} تمثيلاً بيانياً لهذه المقارنة، حيث يتضح تفوق المنصات الإلكترونية في معظم المعايير باستثناء الخدمة الشخصية التي تتفوق فيها وكالات الأسفار بشكل واضح:

\begin{figure}[H]
\centering
\begin{tikzpicture}
\begin{axis}[
    ybar,
    bar width=10pt,
    width=0.85\textwidth,
    height=7cm,
    ylabel={الدرجة (من 10)},
    xtick={1,2,3,4,5,6},
    xticklabels={السعر, السهولة, التوفر, المعلومات, التخصيص, الخدمة الشخصية},
    x tick label style={font=\small, align=center},
    ymin=0, ymax=11,
    ytick={0,2,4,6,8,10},
    legend style={at={(0.5,-0.18)}, anchor=north, legend columns=2, font=\small},
    nodes near coords,
    every node near coord/.append style={font=\scriptsize},
    enlarge x limits=0.12,
]
\addplot[fill=blue!60, draw=blue!70] coordinates {(1,9) (2,9) (3,10) (4,9) (5,8) (6,4)};
\addplot[fill=orange!50, draw=orange!60] coordinates {(1,6) (2,5) (3,4) (4,7) (5,7) (6,9)};
\legend{المنصات الإلكترونية, وكالات الأسفار التقليدية}
\end{axis}
\end{tikzpicture}
\caption{تمثيل بياني لمقارنة المزايا بين المنصات الإلكترونية ووكالات الأسفار}
\label{fig:comparison_chart}
\end{figure}

\subsection{نماذج الأعمال للمنصات الإلكترونية}

تعتمد المنصات الإلكترونية للسفر على نماذج أعمال متنوعة لتحقيق الإيرادات، ويمكن تلخيص أبرزها فيما يلي:

\subsubsection{نموذج العمولة (Commission Model)}

يُعدّ هذا النموذج الأكثر شيوعاً في صناعة السفر عبر الإنترنت. وبموجبه، تحصل المنصة على نسبة مئوية من قيمة كل حجز يتم عبرها. وتتراوح نسبة العمولة عادة بين 15\% و25\% لحجوزات الفنادق، وبين 3\% و5\% لتذاكر الطيران. وتتبع هذا النموذج منصات كبرى مثل بوكينغ دوت كوم وإكسبيديا.

\subsubsection{نموذج التاجر (Merchant Model)}

في هذا النموذج، تشتري المنصة غرفاً فندقية أو مقاعد طيران بأسعار الجملة، ثم تعيد بيعها للمسافرين بسعر أعلى مع إضافة هامش ربحها. ويتميز هذا النموذج بتحقيق هوامش ربح أعلى لكنه ينطوي على مخاطر مالية أكبر.

\subsubsection{نموذج الإعلانات (Advertising Model)}

تعتمد بعض المنصات، خاصة محركات البحث عن السفر، على الإيرادات الإعلانية كمصدر رئيسي للدخل. حيث تدفع شركات الطيران والفنادق ووكالات السفر الإلكترونية مقابل الظهور في نتائج البحث أو الحصول على مواقع بارزة في الموقع. وتتبع هذا النموذج منصات مثل تريب أدفايزر وسكاي سكانر.

\subsubsection{نموذج الاشتراك (Subscription Model)}

ظهر هذا النموذج حديثاً في بعض المنصات التي تقدم خدمات متميزة مقابل اشتراك شهري أو سنوي. ويحصل المشتركون على مزايا حصرية مثل خصومات إضافية وخدمة عملاء مميزة وضمانات أفضل.

\subsection{الاتجاهات الحديثة في المنصات الإلكترونية}

تتطور المنصات الإلكترونية للسفر بشكل مستمر مع تطور التكنولوجيا وتغير توقعات المستهلكين. ومن أبرز الاتجاهات الحديثة:

\begin{itemize}[label=\textbf{--}]
\item \textbf{الذكاء الاصطناعي والروبوتات المحادثة:} تستخدم المنصات بشكل متزايد تقنيات الذكاء الاصطناعي لتقديم خدمة عملاء آلية من خلال روبوتات المحادثة (Chatbots) التي يمكنها الإجابة على استفسارات المسافرين وتقديم المساعدة على مدار الساعة.

\item \textbf{الواقع الافتراضي والمعزز:} بدأت بعض المنصات في استخدام تقنيات الواقع الافتراضي لتمكين المسافرين من ``زيارة'' الفنادق والوجهات السياحية افتراضياً قبل الحجز، مما يساعد في اتخاذ قرارات أفضل.

\item \textbf{التخصيص الفائق:} تتجه المنصات نحو مستويات أعمق من التخصيص باستخدام تحليل البيانات الضخمة والتعلم الآلي، لتقديم تجارب مصممة خصيصاً لكل مستخدم.

\item \textbf{الاستدامة والسياحة المسؤولة:} تولي المنصات اهتماماً متزايداً بموضوع الاستدامة البيئية، حيث بدأت في عرض معلومات عن البصمة الكربونية للرحلات وتشجيع الخيارات الأكثر استدامة.

\item \textbf{المدفوعات الرقمية المتقدمة:} توسيع خيارات الدفع لتشمل المحافظ الإلكترونية والعملات المشفرة وأنظمة الدفع المحلية في مختلف الأسواق.
\end{itemize}


\subsection{البنية التحتية التكنولوجية للمنصات الإلكترونية}

لفهم أعمق لتفوق المنصات الإلكترونية، من المفيد تحليل البنية التحتية التكنولوجية التي تقوم عليها هذه المنصات والتي تمثل أحد أهم عوامل نجاحها.

\subsubsection{الحوسبة السحابية والبنية التحتية}

تعتمد المنصات الإلكترونية الكبرى على بنية تحتية تكنولوجية ضخمة ومعقدة تقوم على تقنيات الحوسبة السحابية. فمنصة مثل بوكينغ دوت كوم تعالج ملايين عمليات البحث والحجز يومياً، مما يتطلب قدرات حوسبية هائلة وأنظمة ذات موثوقية عالية. وتستخدم هذه المنصات خدمات الحوسبة السحابية من مزودين عالميين مثل أمازون ويب سيرفيسز (AWS) ومايكروسوفت أزور وغوغل كلاود، مما يتيح لها توسيع أو تقليص قدراتها الحوسبية تلقائياً حسب حجم الطلب.

وتتميز هذه البنية التحتية بعدة خصائص أساسية:

\begin{itemize}[label=\textbf{--}]
\item \textbf{التوافرية العالية:} أنظمة مكررة في مراكز بيانات متعددة حول العالم لضمان استمرارية الخدمة.
\item \textbf{المرونة:} القدرة على التوسع التلقائي لاستيعاب فترات الذروة (مواسم الإجازات، العروض الترويجية).
\item \textbf{السرعة:} أوقات استجابة لا تتجاوز أجزاء من الثانية لتحسين تجربة المستخدم.
\item \textbf{الأمان:} طبقات متعددة من الحماية للبيانات الشخصية والمالية.
\end{itemize}

\subsubsection{أنظمة التسعير الديناميكي}

من أبرز الابتكارات التكنولوجية للمنصات الإلكترونية أنظمة التسعير الديناميكي (Dynamic Pricing)، وهي أنظمة خوارزمية معقدة تقوم بتعديل الأسعار تلقائياً وفي الوقت الحقيقي بناءً على مجموعة من العوامل، منها \parencite{laudon2020}:

\begin{itemize}[label=\textbf{--}]
\item مستوى العرض والطلب الحالي.
\item الموسم والتوقيت (أوقات الذروة مقابل أوقات الهدوء).
\item سلوك المستخدم وتاريخ تصفحه.
\item أسعار المنافسين.
\item المدة المتبقية قبل تاريخ الخدمة.
\item حجم الحجوزات المؤكدة مقابل السعة المتاحة.
\end{itemize}

وتتيح هذه الأنظمة للمنصات تعظيم إيراداتها مع الحفاظ على تنافسية أسعارها، وهو ما يصعب على وكالات الأسفار التقليدية تحقيقه بالأساليب اليدوية.

\subsubsection{تقنيات تحليل البيانات الضخمة}

تجمع المنصات الإلكترونية كميات هائلة من البيانات عن سلوك المستخدمين واتجاهات السوق وأداء مقدمي الخدمات. وتستخدم تقنيات تحليل البيانات الضخمة (Big Data Analytics) لاستخراج رؤى قيّمة من هذه البيانات:

\begin{itemize}[label=\textbf{--}]
\item \textbf{تحليل سلوك المستخدم:} فهم أنماط البحث والحجز والتفضيلات الفردية.
\item \textbf{التنبؤ بالطلب:} توقع حجم الطلب على وجهات ومنتجات معينة في فترات محددة.
\item \textbf{تحسين العروض:} تحديد العروض الأكثر جاذبية لكل شريحة من العملاء.
\item \textbf{كشف الاحتيال:} تحديد المعاملات المشبوهة وحماية المنصة والمستخدمين.
\item \textbf{تحليل المنافسين:} رصد ومتابعة أسعار واستراتيجيات المنافسين في الوقت الحقيقي.
\end{itemize}


\subsection{وضع وكالات الأسفار في العالم العربي}

\subsubsection{السياق العام}

يتميز قطاع وكالات الأسفار في العالم العربي بخصوصيات تميزه عن نظيره في الدول الغربية. فعلى الرغم من أن التحول الرقمي يسير بوتيرة أبطأ نسبياً مقارنة بالأسواق المتقدمة، إلا أن التحديات التي تواجه الوكالات العربية لا تقل حدة. ويمكن تلخيص أبرز خصوصيات هذا السياق \parencite{albalushi2019}:

\begin{itemize}[label=\textbf{--}]
\item أهمية خاصة لقطاع الحج والعمرة الذي يمثل جزءاً كبيراً من نشاط الوكالات في الدول الإسلامية.
\item تأثير العوامل الثقافية والاجتماعية على تفضيلات المسافرين (الثقة في التعامل الشخصي، أهمية العلاقات).
\item تفاوت كبير في مستوى التطور الرقمي بين بلدان الخليج العربي والبلدان الأخرى.
\item دور الحكومات في تنظيم وتوجيه القطاع، خاصة في ما يتعلق بالسياحة الدينية.
\item تنامي نسبة الشباب في المجتمعات العربية وتأثير ذلك على أنماط الاستهلاك السياحي.
\end{itemize}

\subsubsection{التحديات الخاصة في السوق العربي}

تواجه وكالات الأسفار في المنطقة العربية تحديات إضافية خاصة بالسياق المحلي:

\begin{itemize}[label=\textbf{--}]
\item \textbf{ضعف البنية التحتية الرقمية:} في بعض البلدان العربية، لا تزال البنية التحتية للاتصالات والإنترنت دون المستوى المطلوب لدعم التحول الرقمي الكامل.
\item \textbf{محدودية أنظمة الدفع الإلكتروني:} لا يزال انتشار بطاقات الائتمان والمحافظ الإلكترونية محدوداً في بعض الأسواق العربية مقارنة بالدول المتقدمة.
\item \textbf{الحواجز اللغوية:} محدودية المحتوى العربي الجيد على المنصات الإلكترونية العالمية.
\item \textbf{الأوضاع السياسية والأمنية:} تأثير عدم الاستقرار في بعض مناطق العالم العربي على حركة السياحة والسفر.
\item \textbf{البيروقراطية الإدارية:} تعقيد الإجراءات المتعلقة بالتراخيص والتأشيرات في بعض الدول.
\end{itemize}

\subsubsection{الفرص المتاحة}

في المقابل، تتوفر فرص مهمة لوكالات الأسفار في العالم العربي:

\begin{itemize}[label=\textbf{--}]
\item \textbf{النمو السياحي:} تشهد عدة دول عربية (السعودية، الإمارات، المغرب، مصر) استثمارات ضخمة في قطاع السياحة.
\item \textbf{رؤية 2030 السعودية:} تستهدف المملكة العربية السعودية استقطاب 100 مليون سائح بحلول 2030، مما يخلق فرصاً هائلة.
\item \textbf{تنامي الطبقة المتوسطة:} يؤدي تحسن المستوى المعيشي في بعض الدول العربية إلى زيادة الإنفاق على السفر والسياحة.
\item \textbf{السياحة البينية العربية:} تمثل السياحة بين الدول العربية سوقاً متنامياً يمكن لوكالات الأسفار استغلاله.
\item \textbf{الحج والعمرة:} سوق مستقر ومتنامٍ يتطلب خبرة متخصصة لا تتوفر بسهولة عبر المنصات الإلكترونية.
\end{itemize}

ويوضح الجدول \ref{tab:arab_market} بعض المؤشرات السياحية في الدول العربية الرئيسية:

\begin{table}[H]
\centering
\caption{مؤشرات سياحية مختارة في الدول العربية الرئيسية (2023)}
\label{tab:arab_market}
\begin{tabular}{|r|r|r|r|}
\hline
\textbf{الدولة} & \textbf{عدد السياح (مليون)} & \textbf{إيرادات السياحة (مليار \$)} & \textbf{نسبة الحجز الإلكتروني (\%)} \\
\hline
الإمارات & 21.9 & 38.5 & 55 \\
\hline
السعودية & 27.4 & 36.2 & 40 \\
\hline
المغرب & 14.5 & 10.2 & 35 \\
\hline
مصر & 14.9 & 13.6 & 30 \\
\hline
تونس & 9.4 & 2.8 & 25 \\
\hline
الأردن & 5.8 & 6.1 & 30 \\
\hline
\end{tabular}
\end{table}


\subsection{الدراسات السابقة المتعلقة بالموضوع}

تناولت العديد من الدراسات الأكاديمية موضوع التحول الرقمي في صناعة السفر وتأثيره على وكالات الأسفار التقليدية. ويمكن استعراض أبرز هذه الدراسات:

\subsubsection{الدراسات الأجنبية}

\textbf{1. دراسة ستاندينغ وتانغ-تافي وبويار (2014):}

هدفت هذه الدراسة إلى تحليل تأثير الإنترنت على وكالات السفر التقليدية في أستراليا. وتوصلت إلى أن الإنترنت أحدث تحولاً جذرياً في سلسلة التوزيع السياحي، وأن وكالات الأسفار التي لم تتكيف مع التطورات الرقمية واجهت تراجعاً حاداً في أعمالها. كما أكدت الدراسة أن التخصص والخدمة الشخصية يمثلان فرصة للتمايز \parencite{standing2014}.

\textbf{2. دراسة بوهاليس (2020):}

ركزت هذه الدراسة على التحول الرقمي في صناعة السياحة بشكل عام، وأبرزت كيف أن التكنولوجيات الذكية (الذكاء الاصطناعي، إنترنت الأشياء، البلوك تشين) تُعيد تشكيل القطاع بالكامل. وخلصت إلى أن الابتكار التكنولوجي المستمر سيزيد من الضغط على الوسطاء التقليديين، لكنه سيخلق أيضاً فرصاً جديدة لمن يستطيع التكيف \parencite{buhalis2020}.

\textbf{3. دراسة أمارو وديوغو (2015):}

درست العوامل المؤثرة في نية شراء السفر عبر الإنترنت، وتوصلت إلى أن الثقة وسهولة الاستخدام والفائدة المدركة هي العوامل الأكثر تأثيراً في قرار المسافر بالحجز عبر الإنترنت. كما أشارت إلى أن الخبرة السابقة في التعاملات الإلكترونية تلعب دوراً مهماً في تعزيز التوجه نحو الحجز الرقمي \parencite{amaro2015}.

\subsubsection{الدراسات العربية}

\textbf{1. دراسة الحماد (2021):}

تناولت التحول الرقمي في قطاع السياحة العربي، وأبرزت الفجوة الرقمية الكبيرة بين وكالات الأسفار في المنطقة العربية ونظيراتها في الدول المتقدمة. وأوصت بضرورة تبني استراتيجيات تحول رقمي شاملة مع مراعاة الخصوصيات الثقافية والاقتصادية للمنطقة \parencite{alhammad2021}.

\textbf{2. دراسة القحطاني (2020):}

ركزت على تطبيقات التجارة الإلكترونية في قطاع السياحة بالمملكة العربية السعودية. وأظهرت أن معدل تبني التجارة الإلكترونية في وكالات الأسفار السعودية لا يزال دون المستوى المأمول، رغم الإمكانات الكبيرة المتاحة. وأوصت بتعزيز الدعم الحكومي لتسريع التحول الرقمي في القطاع \parencite{alqahtani2020}.

\textbf{3. دراسة البلوشي (2019):}

تناولت التحديات والفرص التي تواجه صناعة السياحة في العالم العربي بشكل عام. وأبرزت أهمية التكيف مع التغيرات التكنولوجية والسلوكية للحفاظ على تنافسية القطاع السياحي العربي \parencite{albalushi2019}.

\subsubsection{موقع الدراسة الحالية من الدراسات السابقة}

تتميز الدراسة الحالية عن الدراسات السابقة في عدة جوانب:

\begin{enumerate}[label=\textbf{\arabic*.}]
\item تركز بشكل خاص على التحديات التي تواجه وكالات الأسفار في مواجهة المنصات الإلكترونية، وليس على التحول الرقمي بشكل عام.
\item تقدم تحليلاً شاملاً يجمع بين الأبعاد التكنولوجية والاقتصادية والسلوكية والتنظيمية.
\item تقترح حلولاً واستراتيجيات عملية مبنية على تحليل واقعي للتحديات.
\item تأخذ بعين الاعتبار تأثير جائحة كوفيد-19 على ديناميكيات المنافسة في القطاع.
\end{enumerate}


\section*{خلاصة الفصل الأول}

من خلال ما تم عرضه في هذا الفصل، يتضح أن وكالات الأسفار قد مرت بمراحل تطور عديدة منذ تأسيسها في منتصف القرن التاسع عشر، حيث عرفت فترة ذهبية امتدت لعقود طويلة كانت خلالها الوسيط الأساسي في سلسلة التوزيع السياحي. غير أن ظهور المنصات الإلكترونية للسفر في أواخر التسعينيات وبداية الألفية الثالثة شكّل منعطفاً حاسماً في تاريخ هذا القطاع.

وقد تبيّن أن المنصات الإلكترونية تتمتع بمزايا تنافسية عديدة تشمل: التوفر الدائم، وسهولة الاستخدام، والشفافية في الأسعار، ووفرة المعلومات والتقييمات، والأسعار التنافسية، واستخدام التكنولوجيا المتقدمة في تحسين تجربة المستخدم. وهي مزايا أربكت الوكالات التقليدية وفرضت عليها تحديات غير مسبوقة.

كما تبيّن أن وكالات الأسفار في العالم العربي تواجه تحديات خاصة مرتبطة بالسياق المحلي، لكنها تتوفر أيضاً على فرص مهمة مرتبطة بنمو القطاع السياحي والخصوصيات الثقافية للمنطقة. وقد أسهمت الدراسات السابقة في إثراء فهمنا لهذه الديناميكيات، وتسعى الدراسة الحالية إلى تقديم إضافة نوعية في هذا السياق.

وفي الفصل الموالي، سيتم تناول طبيعة المنافسة القائمة بين وكالات الأسفار والمنصات الإلكترونية، وتحليل تأثير هذه المنصات على النشاط التجاري للوكالات.


% الفصل الثاني
% ==============================================================================
% الفصل الثاني: المنافسة بين وكالات الأسفار والمنصات الإلكترونية
% ==============================================================================
\chapter{المنافسة بين وكالات الأسفار والمنصات الإلكترونية}

\section*{تمهيد}

تشهد صناعة السفر والسياحة منافسة حادة ومتعددة الأبعاد بين وكالات الأسفار التقليدية والمنصات الإلكترونية. وتتميز هذه المنافسة بطبيعتها غير المتكافئة في كثير من الجوانب، حيث تمتلك المنصات الإلكترونية موارد تكنولوجية ومالية ضخمة تمنحها تفوقاً واضحاً في العديد من المجالات. وفي المقابل، تحتفظ وكالات الأسفار التقليدية ببعض نقاط القوة التي لا تزال تمثل قيمة حقيقية لشرائح من المسافرين.

يتناول هذا الفصل طبيعة المنافسة القائمة بين وكالات الأسفار والمنصات الإلكترونية من خلال مبحثين: يحلل المبحث الأول طبيعة هذه المنافسة وأبعادها وآلياتها، بينما يدرس المبحث الثاني تأثير المنصات الإلكترونية على النشاط التجاري لوكالات الأسفار.


% ======================================================================
% المبحث الأول: طبيعة المنافسة
% ======================================================================
\section{المبحث الأول: طبيعة المنافسة}

\subsection{الإطار النظري للمنافسة في قطاع السياحة}

\subsubsection{مفهوم المنافسة وأنواعها}

تُعرّف المنافسة في سياق الأعمال بأنها التنافس بين الشركات والمؤسسات للاستحواذ على أكبر حصة ممكنة من السوق وتحقيق التفوق على المنافسين. وقد قدّم بورتر \parencite{porter2008} إطاراً نظرياً شاملاً لتحليل المنافسة في الصناعات من خلال نموذجه المعروف بـ``القوى التنافسية الخمس''، والذي يمكن تطبيقه على قطاع السفر والسياحة.

وتتعدد أنواع المنافسة التي تواجهها وكالات الأسفار في ظل التحول الرقمي:

\begin{itemize}[label=\textbf{--}]
\item \textbf{المنافسة المباشرة:} وهي المنافسة بين وكالات الأسفار فيما بينها، حيث تتنافس على نفس الشريحة من العملاء وتقدم خدمات متشابهة.
\item \textbf{المنافسة غير المباشرة:} وهي المنافسة مع المنصات الإلكترونية التي تقدم بدائل رقمية لخدمات الوكالات التقليدية.
\item \textbf{المنافسة البديلة:} وتتمثل في قدرة المسافرين على تنظيم رحلاتهم بأنفسهم دون الحاجة إلى أي وسيط (وكالة تقليدية أو منصة إلكترونية).
\end{itemize}

\subsubsection{تحليل القوى التنافسية في قطاع السفر}

يمكن تطبيق نموذج بورتر للقوى الخمس على صناعة السفر لفهم طبيعة المنافسة وكثافتها:

\textbf{أ. تهديد الداخلين الجدد:}

أدى التحول الرقمي إلى تخفيض كبير في حواجز الدخول إلى صناعة السفر. فبينما كان إنشاء وكالة سفر تقليدية يتطلب استثمارات كبيرة في المقر والتجهيزات والتراخيص والموارد البشرية المتخصصة، فإن إنشاء منصة إلكترونية أصبح أسهل بكثير وبتكاليف أقل نسبياً. وقد أدى ذلك إلى تزايد عدد المنافسين في السوق وتكثيف المنافسة \parencite{kracht2010}.

غير أنه يجب التنبيه إلى أن النجاح في سوق منصات السفر الإلكترونية يتطلب استثمارات تكنولوجية وتسويقية ضخمة للمنافسة مع اللاعبين الكبار المهيمنين، مما يحد عملياً من عدد الداخلين الجدد القادرين على المنافسة بفعالية.

\textbf{ب. القوة التفاوضية للموردين:}

يتمثل الموردون في هذا السياق في مقدمي الخدمات السياحية (شركات الطيران، الفنادق، شركات النقل). وقد تغيرت العلاقة بين الموردين وقنوات التوزيع بشكل جذري مع التحول الرقمي:

\begin{itemize}[label=\textbf{--}]
\item أصبح بإمكان شركات الطيران والفنادق البيع مباشرة للمستهلكين عبر مواقعها الإلكترونية، مما قلل من اعتمادها على الوسطاء.
\item قامت شركات الطيران بتخفيض أو إلغاء العمولات المدفوعة لوكالات الأسفار.
\item في المقابل، أصبحت الفنادق الصغيرة والمتوسطة أكثر اعتماداً على المنصات الإلكترونية الكبرى لتسويق خدماتها.
\end{itemize}

\textbf{ج. القوة التفاوضية للمشترين (المسافرين):}

ازدادت القوة التفاوضية للمسافرين بشكل كبير بفضل التحول الرقمي. فأصبح المسافر يمتلك:

\begin{itemize}[label=\textbf{--}]
\item وصولاً سهلاً إلى المعلومات والأسعار من مصادر متعددة.
\item القدرة على مقارنة العروض بسرعة وسهولة.
\item حرية الاختيار بين عدد كبير من البدائل.
\item منصات لتبادل الخبرات والتقييمات مع مسافرين آخرين.
\end{itemize}

وقد أدت هذه التطورات إلى تحويل ميزان القوة لصالح المستهلك، مما فرض ضغوطاً إضافية على وكالات الأسفار والمنصات الإلكترونية على حد سواء.

\textbf{د. تهديد المنتجات والخدمات البديلة:}

تواجه وكالات الأسفار تهديداً من عدة بدائل:

\begin{itemize}[label=\textbf{--}]
\item المنصات الإلكترونية كبديل رقمي للخدمة الوسيطة.
\item المبيعات المباشرة من شركات الطيران والفنادق عبر مواقعها.
\item التخطيط الذاتي للرحلات باستخدام أدوات الإنترنت المجانية.
\item شبكات التواصل الاجتماعي كمصدر للمعلومات والتوصيات السياحية.
\end{itemize}

\textbf{هـ. شدة المنافسة بين المتنافسين الحاليين:}

تتسم المنافسة في سوق خدمات السفر بالشدة العالية، ويعود ذلك إلى عدة عوامل:

\begin{itemize}[label=\textbf{--}]
\item تعدد المتنافسين من وكالات تقليدية ومنصات إلكترونية ومقدمي خدمات مباشرين.
\item التجانس النسبي في المنتجات والخدمات المقدمة.
\item سهولة تحويل العملاء بين مقدمي الخدمات (انخفاض تكاليف التبديل).
\item الضغط المستمر على الأسعار بسبب الشفافية السعرية.
\end{itemize}

ويلخص الجدول \ref{fig:porter} تطبيق نموذج بورتر على قطاع السفر:

\begin{table}[H]
\centering
\caption{نموذج بورتر للقوى الخمس مطبقاً على قطاع السفر}
\label{fig:porter}
\begin{tabular}{|r|r|}
\hline
\textbf{القوة التنافسية} & \textbf{التطبيق على قطاع السفر} \\
\hline
\textbf{شدة المنافسة بين المتنافسين الحاليين} & منافسة شديدة بين الوكالات التقليدية والمنصات الإلكترونية \\
\hline
\textbf{تهديد الداخلين الجدد} & سهولة إنشاء منصات رقمية جديدة بتكاليف منخفضة \\
\hline
\textbf{تهديد المنتجات البديلة} & الحجز المباشر عبر مواقع شركات الطيران والفنادق \\
\hline
\textbf{القوة التفاوضية للموردين} & شركات الطيران والفنادق تتحكم في الأسعار والعمولات \\
\hline
\textbf{القوة التفاوضية للمشترين} & المسافرون يملكون خيارات متعددة ومعلومات وافرة \\
\hline
\end{tabular}
\end{table}

ويوضح الشكل \ref{fig:porter_diagram} تمثيلاً بيانياً لنموذج بورتر مطبقاً على قطاع السفر، حيث تتفاعل القوى الخمس لتشكيل بيئة تنافسية شديدة:

\begin{figure}[H]
\centering
\begin{tikzpicture}[
    box/.style={rectangle, draw=blue!60, rounded corners=4pt, fill=blue!8, 
                text width=4.2cm, minimum height=1.5cm, align=center, font=\small\bfseries},
    cbox/.style={rectangle, draw=red!60, rounded corners=4pt, fill=red!8,
                text width=4.8cm, minimum height=1.8cm, align=center, font=\small\bfseries},
    arr/.style={->, very thick, >=stealth, draw=gray!60}
]
\node[cbox] (c) at (0,0) {شدة المنافسة بين\\المتنافسين الحاليين};
\node[box] (t) at (0,3.8) {تهديد\\الداخلين الجدد};
\node[box] (b) at (0,-3.8) {تهديد المنتجات\\البديلة};
\node[box] (l) at (-5.2,0) {القوة التفاوضية\\للموردين};
\node[box] (r) at (5.2,0) {القوة التفاوضية\\للمشترين};
\draw[arr] (t) -- (c);
\draw[arr] (b) -- (c);
\draw[arr] (l) -- (c);
\draw[arr] (r) -- (c);
\end{tikzpicture}
\caption{نموذج بورتر للقوى التنافسية الخمس مطبقاً على قطاع السفر}
\label{fig:porter_diagram}
\end{figure}


\subsection{أبعاد المنافسة بين وكالات الأسفار والمنصات الإلكترونية}

\subsubsection{البعد التكنولوجي}

يُعدّ البعد التكنولوجي من أهم أبعاد المنافسة بين وكالات الأسفار والمنصات الإلكترونية، وربما الأكثر تأثيراً. فالمنصات الإلكترونية تتمتع بتفوق تكنولوجي واضح يتجلى في عدة جوانب:

\textbf{أولاً: البنية التحتية التكنولوجية.} تستثمر المنصات الإلكترونية الكبرى مبالغ ضخمة في تطوير بنيتها التحتية التكنولوجية. فمجموعة بوكينغ هولدينغز، على سبيل المثال، أنفقت أكثر من 5 مليارات دولار على التكنولوجيا والتطوير في عام 2022 وحده. وتشمل هذه الاستثمارات: مراكز بيانات متقدمة، وأنظمة حوسبة سحابية، ومنصات معالجة بيانات ضخمة، وأنظمة أمن سيبراني متطورة.

في المقابل، تعاني معظم وكالات الأسفار التقليدية من محدودية استثماراتها التكنولوجية. فالعديد منها لا يزال يعتمد على أنظمة قديمة ولا يمتلك مواقع إلكترونية متطورة أو تطبيقات للهاتف المحمول. وحتى تلك التي تمتلك تواجداً رقمياً، فإن مستوى تطور منصاتها يبقى بعيداً عن مستوى المنصات العالمية \parencite{alhammad2021}.

\textbf{ثانياً: الذكاء الاصطناعي وتحليل البيانات.} تستخدم المنصات الإلكترونية تقنيات متقدمة في الذكاء الاصطناعي والتعلم الآلي لتحسين تجربة المستخدم وزيادة معدلات التحويل. ومن تطبيقات هذه التقنيات:

\begin{itemize}[label=\textbf{--}]
\item أنظمة التوصيات الذكية التي تقترح وجهات وفنادق بناءً على سلوك المستخدم.
\item التسعير الديناميكي الذي يعدّل الأسعار تلقائياً بناءً على العرض والطلب.
\item روبوتات المحادثة (Chatbots) لخدمة العملاء الآلية.
\item أنظمة كشف الاحتيال وحماية المعاملات المالية.
\item تحليل المشاعر والتقييمات لفهم رضا العملاء.
\end{itemize}

\textbf{ثالثاً: تجربة المستخدم الرقمية.} تولي المنصات الإلكترونية اهتماماً بالغاً بتصميم تجربة المستخدم (UX)، حيث توظف فرقاً متخصصة في التصميم وإجراء البحث المستمر لتحسين واجهات الاستخدام وتبسيط رحلة العميل من البحث إلى الحجز إلى ما بعد الرحلة.

\subsubsection{البعد التسويقي}

تتمتع المنصات الإلكترونية بقدرات تسويقية ضخمة تفوق بكثير ما تمتلكه وكالات الأسفار التقليدية:

\textbf{أ. ميزانيات التسويق الرقمي:}

تُنفق المنصات الإلكترونية الكبرى مبالغ هائلة على التسويق الرقمي. فمجموعة بوكينغ هولدينغز أنفقت أكثر من 6 مليارات دولار على التسويق عبر الإنترنت في عام 2022، معظمها على إعلانات غوغل وحملات التسويق الرقمي. وتُعدّ هذه المنصات من أكبر المعلنين على محركات البحث عالمياً \parencite{phocuswright2022}.

في المقابل، تمتلك وكالات الأسفار التقليدية ميزانيات تسويقية محدودة جداً مقارنة بالمنصات العالمية، مما يجعلها غير قادرة على المنافسة في مجال التسويق الرقمي.

\textbf{ب. تحسين محركات البحث (SEO):}

تستثمر المنصات الإلكترونية بكثافة في تحسين ظهورها في نتائج محركات البحث. وعند البحث عن أي وجهة أو فندق أو رحلة طيران على غوغل، تهيمن المنصات الإلكترونية الكبرى على النتائج الأولى، مما يجعلها النقطة الأولى التي يصل إليها المسافر عند البحث عبر الإنترنت.

\textbf{ج. التسويق عبر وسائل التواصل الاجتماعي:}

تمتلك المنصات الإلكترونية حضوراً قوياً على منصات التواصل الاجتماعي، حيث تستخدمها للتفاعل مع العملاء ومشاركة المحتوى الملهم وإطلاق الحملات الترويجية. كما تستفيد من التسويق عبر المؤثرين (Influencer Marketing) للوصول إلى شرائح جديدة من الجمهور.

\textbf{د. برامج الولاء:}

تمتلك معظم المنصات الإلكترونية الكبرى برامج ولاء متطورة تهدف إلى الاحتفاظ بالعملاء وتشجيع تكرار الحجز. ومن أمثلة هذه البرامج: برنامج Genius من بوكينغ، وبرنامج One Key من إكسبيديا. وتقدم هذه البرامج مزايا مثل خصومات حصرية، وترقيات مجانية، وخدمات ذات أولوية.

\subsubsection{البعد السعري}

تُعتبر المنافسة السعرية من أشد أبعاد المنافسة تأثيراً على وكالات الأسفار التقليدية. وتتمتع المنصات الإلكترونية بمزايا سعرية عديدة تعود إلى:

\begin{enumerate}[label=\textbf{\arabic*.}]
\item \textbf{وفورات الحجم:} بفضل حجم أعمالها الهائل، تحصل المنصات الإلكترونية على أسعار تفاوضية أفضل من مقدمي الخدمات. فمنصة بوكينغ، التي تسجل أكثر من مليون ليلة فندقية يومياً، تتمتع بقدرة تفاوضية لا يمكن لأي وكالة تقليدية مضاهاتها.

\item \textbf{انخفاض التكاليف التشغيلية:} لا تحتاج المنصات الإلكترونية إلى شبكة مكاتب ومحلات تجارية، مما يوفر تكاليف الإيجار والتجهيز والصيانة. كما أن أتمتة العمليات تقلل من الحاجة إلى عدد كبير من الموظفين.

\item \textbf{التسعير الديناميكي:} تستخدم المنصات تقنيات التسعير الديناميكي التي تعدّل الأسعار تلقائياً بناءً على العرض والطلب وسلوك المستخدم وعوامل أخرى، مما يمكّنها من تحقيق أقصى إيرادات مع الحفاظ على تنافسية الأسعار.

\item \textbf{ضمانات أفضل الأسعار:} تقدم بعض المنصات ضمانات مطابقة السعر (Price Match Guarantee)، حيث تتعهد بتقديم أفضل سعر متاح وتعويض الفرق إذا وجد العميل سعراً أقل في مكان آخر.
\end{enumerate}

\subsubsection{البعد الجغرافي}

تمتلك المنصات الإلكترونية نطاقاً جغرافياً عالمياً يفوق بمراحل ما تستطيع وكالة الأسفار التقليدية تقديمه:

\begin{itemize}[label=\textbf{--}]
\item \textbf{التغطية العالمية:} تغطي المنصات الإلكترونية الكبرى آلاف الوجهات في جميع أنحاء العالم. فمنصة بوكينغ تقدم خيارات إقامة في أكثر من 220 دولة ومنطقة.
\item \textbf{عدم التقيد بالموقع:} يمكن للمسافر الوصول إلى المنصة والحجز من أي مكان في العالم، بينما تتقيد وكالة الأسفار التقليدية بموقعها الجغرافي.
\item \textbf{تعدد اللغات والعملات:} تدعم المنصات الإلكترونية عشرات اللغات والعملات، مما يسهل استخدامها من قبل مسافرين من مختلف الجنسيات.
\end{itemize}

\subsection{نقاط قوة وكالات الأسفار في المنافسة}

على الرغم من التفوق الواضح للمنصات الإلكترونية في العديد من الأبعاد التنافسية، تحتفظ وكالات الأسفار التقليدية بنقاط قوة مهمة يمكن أن تشكل أساساً لاستراتيجية تنافسية فعالة:

\subsubsection{الخدمة الشخصية والعلاقة الإنسانية}

تُعدّ الخدمة الشخصية والتواصل الإنساني المباشر من أبرز نقاط القوة التي تتميز بها وكالات الأسفار التقليدية. فالعديد من المسافرين يفضلون التعامل مع شخص حقيقي يمكنه فهم احتياجاتهم بعمق وتقديم نصائح مخصصة بناءً على خبرة شخصية ومعرفة متراكمة. وتبرز أهمية هذه الميزة بشكل خاص في الحالات التالية:

\begin{itemize}[label=\textbf{--}]
\item الرحلات المعقدة التي تشمل وجهات متعددة وعناصر كثيرة.
\item السفر إلى وجهات غير مألوفة أو تتطلب ترتيبات خاصة.
\item المسافرون كبار السن أو ذوو الاحتياجات الخاصة.
\item رحلات شهر العسل والمناسبات الخاصة.
\item سفرات الأعمال التي تتطلب مرونة وتنسيقاً عالياً.
\end{itemize}

\subsubsection{الخبرة والمعرفة المتخصصة}

يمتلك مستشارو السفر في الوكالات التقليدية معرفة ميدانية وخبرة متراكمة لا يمكن لخوارزميات الذكاء الاصطناعي محاكاتها بالكامل. فهم يعرفون الفنادق والوجهات من تجربة شخصية، ويمكنهم تقديم توصيات دقيقة بناءً على معرفة واقعية بجودة الخدمات والمرافق. كما يمكنهم التعامل مع المواقف الاستثنائية وتقديم حلول إبداعية لا تستطيع الأنظمة الآلية توفيرها \parencite{zeithaml2018}.

\subsubsection{الأمان والثقة}

يفضل العديد من المسافرين التعامل مع وكالة أسفار معروفة وموثوقة في مجتمعهم المحلي، خاصة عند إجراء حجوزات مالية كبيرة. فالتعامل وجهاً لوجه مع موظف في مكتب فعلي يمنح العميل شعوراً أكبر بالأمان والثقة مقارنة بالتعامل مع منصة إلكترونية. كما أن وجود مقر فعلي يمكن الرجوع إليه في حالات المشاكل أو النزاعات يُعدّ عامل اطمئنان مهماً لكثير من العملاء.

\subsubsection{الحماية القانونية}

توفر وكالات الأسفار المرخصة حماية قانونية أكبر للمسافرين في العديد من الدول. فالتشريعات المنظمة لنشاط الوكالات تفرض عليها التزامات تجاه العملاء في حالات الإلغاء أو التغيير أو المشاكل أثناء الرحلة. كما أن الضمانات المالية المطلوبة قانونياً توفر حماية لأموال العملاء في حالات الإفلاس.

\subsubsection{خدمة ما بعد البيع}

تتميز وكالات الأسفار بقدرتها على تقديم خدمة ما بعد البيع شخصية وفعالة. ففي حالات المشاكل أثناء الرحلة (تأخير رحلات، إلغاء حجوزات، مشاكل في الفندق)، يمكن للمسافر الاتصال بوكالته التي تتولى التنسيق مع مقدمي الخدمات لحل المشكلة. وهذه الخدمة تمثل قيمة حقيقية خاصة في المواقف الطارئة والأزمات.

ويوضح الجدول \ref{tab:comparison} مقارنة شاملة بين نقاط القوة والضعف لكل من وكالات الأسفار والمنصات الإلكترونية:

\begin{table}[H]
\centering
\caption{مقارنة بين نقاط القوة والضعف}
\label{tab:comparison}
\begin{tabular}{|r|r|r|}
\hline
\textbf{المعيار} & \textbf{وكالات الأسفار} & \textbf{المنصات الإلكترونية} \\
\hline
الخدمة الشخصية & قوية جداً & ضعيفة \\
\hline
التوفر الزمني & محدود & دائم (24/7) \\
\hline
النطاق الجغرافي & محلي/إقليمي & عالمي \\
\hline
الأسعار & متوسطة & تنافسية \\
\hline
المعلومات & خبرة شخصية & بيانات ضخمة \\
\hline
التكنولوجيا & محدودة & متقدمة جداً \\
\hline
الثقة & عالية محلياً & متفاوتة \\
\hline
المرونة & عالية & محدودة بالنظام \\
\hline
\end{tabular}
\end{table}


% ======================================================================
% المبحث الثاني: تأثير المنصات الإلكترونية على وكالات الأسفار
% ======================================================================
\section{المبحث الثاني: تأثير المنصات الإلكترونية على وكالات الأسفار}

\subsection{التأثير على الحصة السوقية}

\subsubsection{تراجع الحصة السوقية لوكالات الأسفار}

أدى نمو المنصات الإلكترونية إلى تآكل مستمر في الحصة السوقية لوكالات الأسفار التقليدية. وتشير الإحصائيات والتقارير الدولية إلى حجم هذا التحول:

وفقاً لتقرير فوكسرايت \parencite{phocuswright2022}، ارتفعت حصة الحجوزات عبر الإنترنت من إجمالي حجوزات السفر العالمية من أقل من 5\% في عام 2000 إلى أكثر من 65\% في عام 2022. وفي بعض الأسواق المتقدمة رقمياً مثل الولايات المتحدة والمملكة المتحدة والدول الاسكندنافية، تجاوزت النسبة 75\%.

أما في الأسواق العربية، فإن نسبة الحجز عبر الإنترنت أقل نسبياً لكنها في تزايد مستمر. وتشير التقديرات إلى أن نسبة الحجز الإلكتروني في منطقة الشرق الأوسط وشمال إفريقيا تجاوزت 40\% في عام 2023، مع توقعات بارتفاعها بشكل كبير في السنوات القادمة \parencite{euromonitor2023}.

\subsubsection{تراجع عدد وكالات الأسفار}

انعكس تراجع الحصة السوقية على عدد وكالات الأسفار العاملة في العديد من الدول. فقد شهدت أسواق مثل الولايات المتحدة وأوروبا انخفاضاً كبيراً في عدد الوكالات:

\begin{itemize}[label=\textbf{--}]
\item في الولايات المتحدة، انخفض عدد وكالات الأسفار من أكثر من 34,000 وكالة في عام 2000 إلى نحو 15,000 وكالة في عام 2022.
\item في المملكة المتحدة، أغلقت شركة توماس كوك العريقة أبوابها نهائياً في عام 2019 بعد 178 عاماً من العمل، في رمزية بالغة الدلالة على حجم التحولات في القطاع.
\item في فرنسا، تراجع عدد وكالات الأسفار بنحو 30\% خلال العشرية الأخيرة.
\end{itemize}

وقد تسارع هذا التراجع بشكل كبير بعد جائحة كوفيد-19 في عام 2020، التي شكّلت ضربة قاصمة للعديد من الوكالات التي لم تستطع تحمل التوقف شبه الكامل لحركة السفر لأشهر طويلة.

\subsubsection{إعادة التوزيع السوقي}

لم يقتصر تأثير المنصات الإلكترونية على تقليص حصة وكالات الأسفار فحسب، بل أعاد رسم خريطة التوزيع السوقي بالكامل. فقد أصبحت شركات قليلة عملاقة تهيمن على السوق العالمي لحجز السفر عبر الإنترنت:

\begin{itemize}[label=\textbf{--}]
\item مجموعة بوكينغ هولدينغز (تشمل بوكينغ دوت كوم وكاياك وأجودا): إيرادات تجاوزت 17 مليار دولار في 2022.
\item مجموعة إكسبيديا (تشمل إكسبيديا وهوتيلز دوت كوم وفي آر بي أو): إيرادات تجاوزت 12 مليار دولار في 2022.
\item مجموعة تريب دوت كوم: إيرادات تجاوزت 3 مليارات دولار في 2022.
\end{itemize}

ويوضح الجدول \ref{fig:marketshare} تطور توزيع حصص السوق بين القنوات المختلفة:

\begin{table}[H]
\centering
\caption{تطور توزيع حصص سوق حجز السفر بالنسبة المئوية (2000-2023)}
\label{fig:marketshare}
\begin{tabular}{|r|r|r|r|r|r|r|}
\hline
\textbf{القناة} & \textbf{2000} & \textbf{2005} & \textbf{2010} & \textbf{2015} & \textbf{2020} & \textbf{2023} \\
\hline
وكالات الأسفار التقليدية & 60\% & 45\% & 30\% & 22\% & 18\% & 15\% \\
\hline
مبيعات مباشرة & 35\% & 30\% & 25\% & 20\% & 22\% & 20\% \\
\hline
المنصات الإلكترونية & 5\% & 25\% & 45\% & 58\% & 60\% & 65\% \\
\hline
\end{tabular}
\end{table}

ويوضح الشكل \ref{fig:marketshare_chart} بشكل بياني حجم التحول الذي شهده سوق حجز السفر، حيث يتضح بجلاء الصعود الكبير للمنصات الإلكترونية على حساب وكالات الأسفار التقليدية:

\begin{figure}[H]
\centering
\begin{tikzpicture}
\begin{axis}[
    ybar stacked,
    bar width=18pt,
    width=0.85\textwidth,
    height=8cm,
    ylabel={النسبة المئوية (\%)},
    xtick=data,
    xticklabels={2000, 2005, 2010, 2015, 2020, 2023},
    ymin=0, ymax=105,
    legend style={at={(0.5,-0.18)}, anchor=north, legend columns=3, font=\small},
    enlarge x limits=0.12,
]
\addplot[fill=blue!60, draw=blue!70] coordinates {(1,60) (2,45) (3,30) (4,22) (5,18) (6,15)};
\addplot[fill=red!50, draw=red!60] coordinates {(1,5) (2,25) (3,45) (4,58) (5,60) (6,65)};
\addplot[fill=green!40!black, draw=green!50!black] coordinates {(1,35) (2,30) (3,25) (4,20) (5,22) (6,20)};
\legend{وكالات الأسفار, المنصات الإلكترونية, مبيعات مباشرة}
\end{axis}
\end{tikzpicture}
\caption{التطور البياني لتوزيع حصص سوق حجز السفر (2000--2023)}
\label{fig:marketshare_chart}
\end{figure}


\subsection{التأثير على نموذج الأعمال}

\subsubsection{انهيار نظام العمولات التقليدي}

لعقود طويلة، اعتمدت وكالات الأسفار بشكل أساسي على نظام العمولات كمصدر رئيسي للدخل، حيث كانت تحصل على نسبة مئوية من كل عملية بيع تتم عبرها. غير أن هذا النظام تعرض لضربات متتالية بدءاً من أواخر التسعينيات:

\begin{enumerate}[label=\textbf{\arabic*.}]
\item في عام 1995، بدأت شركات الطيران الأمريكية في تخفيض عمولات وكالات الأسفار من 10\% إلى 8\%.
\item في عام 1999، تم تخفيض العمولات مرة أخرى إلى 5\%.
\item في عام 2002، ألغت معظم شركات الطيران الأمريكية العمولات بالكامل.
\item تبعت ذلك شركات الطيران الأوروبية والعالمية بتخفيضات مماثلة.
\item اليوم، لا تدفع معظم شركات الطيران أي عمولة لوكالات الأسفار التقليدية.
\end{enumerate}

وقد اضطر هذا التحول وكالات الأسفار إلى البحث عن مصادر دخل بديلة، أبرزها فرض رسوم خدمة على العملاء، وهو ما أثار استياء بعض المسافرين ودفعهم نحو الحجز المباشر عبر الإنترنت \parencite{iata2023}.

\subsubsection{الضغط على هوامش الربح}

أدى تزايد المنافسة وتراجع العمولات إلى ضغط شديد على هوامش ربح وكالات الأسفار:

\begin{itemize}[label=\textbf{--}]
\item انخفاض إجمالي العمولات المحصّلة بسبب إلغاء عمولات شركات الطيران.
\item ضغط الأسعار الناتج عن المنافسة مع المنصات الإلكترونية.
\item ارتفاع تكاليف التشغيل (إيجارات، رواتب، تكنولوجيا).
\item صعوبة فرض رسوم خدمة عالية في ظل وجود بدائل مجانية عبر الإنترنت.
\end{itemize}

وتشير التقديرات إلى أن متوسط هامش الربح الصافي لوكالات الأسفار التقليدية انخفض من نحو 8-10\% في التسعينيات إلى 2-4\% في السنوات الأخيرة \parencite{mckinsey2022}، مما يجعل استمرارية العديد من الوكالات الصغيرة أمراً صعباً.

\subsubsection{تغير في تركيبة الإيرادات}

في مواجهة تراجع العمولات، اضطرت وكالات الأسفار إلى تنويع مصادر إيراداتها:

\begin{itemize}[label=\textbf{--}]
\item \textbf{رسوم الخدمة:} فرض رسوم على العملاء مقابل خدمات البحث والحجز والاستشارة.
\item \textbf{الخدمات الإضافية:} التوسع في بيع خدمات ذات هامش ربح أعلى مثل التأمين والتأشيرات والرحلات المخصصة.
\item \textbf{التركيز على الأسواق المتخصصة:} التحول نحو أسواق ذات قيمة مضافة عالية مثل سياحة الفخامة وسياحة الأعمال.
\item \textbf{برامج الحوافز:} الحصول على مكافآت وحوافز من منظمي الرحلات وشركات السياحة مقابل تحقيق أهداف مبيعات محددة.
\end{itemize}


\subsection{التأثير على سلوك المستهلك}

\subsubsection{تحول في أنماط البحث والحجز}

أحدثت المنصات الإلكترونية تحولاً جذرياً في الطريقة التي يبحث بها المسافرون عن خدمات السفر ويحجزونها. وتشير الدراسات إلى أن المسافر المعاصر يمر عادة بالمراحل التالية قبل إتمام الحجز \parencite{google2022}:

\begin{enumerate}[label=\textbf{\arabic*.}]
\item \textbf{مرحلة الإلهام:} يستلهم المسافر أفكار الرحلات من شبكات التواصل الاجتماعي (إنستغرام، يوتيوب، تيك توك) ومن مواقع المحتوى السياحي.
\item \textbf{مرحلة البحث:} يبحث عن معلومات مفصلة عبر محركات البحث والمنصات المتخصصة.
\item \textbf{مرحلة المقارنة:} يقارن الأسعار والخيارات عبر محركات المقارنة ومواقع التقييم.
\item \textbf{مرحلة الحجز:} يختار المنصة التي تقدم أفضل سعر وشروط ويتم الحجز إلكترونياً.
\item \textbf{مرحلة المشاركة:} يشارك تجربته عبر التقييمات ومنشورات التواصل الاجتماعي.
\end{enumerate}

ويُلاحظ أن وكالات الأسفار التقليدية غائبة بشكل كبير عن معظم هذه المراحل، مما يعني أنها تفقد الاتصال مع المسافر في أهم مراحل عملية اتخاذ القرار.

\subsubsection{ظاهرة البحث في الوكالة والحجز عبر الإنترنت}

من الظواهر اللافتة التي أفرزتها المنافسة بين الوكالات والمنصات الإلكترونية ظاهرة ``ROBO'' (Research Offline, Book Online)، أي البحث في الوكالة والحجز عبر الإنترنت. حيث يتوجه بعض المسافرين إلى وكالات الأسفار للاستفادة من خبرتها واستشارتها المجانية في اختيار الوجهة والبرنامج، ثم يقومون بالحجز عبر الإنترنت بسعر أقل. وتُعدّ هذه الظاهرة مصدر إحباط كبير لوكالات الأسفار التي تستثمر وقتاً وجهداً في تقديم الاستشارة دون تحقيق عائد.

\subsubsection{تغير في توقعات المستهلك}

رفعت المنصات الإلكترونية سقف توقعات المستهلك السياحي بشكل كبير. فأصبح المسافر المعاصر يتوقع:

\begin{itemize}[label=\textbf{--}]
\item استجابة فورية لاستفساراته في أي وقت.
\item شفافية كاملة في الأسعار والشروط.
\item إمكانية المقارنة بين خيارات متعددة بسهولة.
\item تجربة حجز سلسة وسريعة.
\item مرونة عالية في التعديل والإلغاء.
\item تأكيد فوري للحجوزات.
\end{itemize}

وتجد وكالات الأسفار التقليدية صعوبة في تلبية هذه التوقعات المرتفعة بنماذجها التشغيلية الحالية، مما يدفع المزيد من المسافرين نحو المنصات الإلكترونية \parencite{bennett2012}.


\subsection{التأثير على سوق العمل في قطاع السفر}

أثر التحول الرقمي في صناعة السفر بشكل كبير على سوق العمل في هذا القطاع:

\subsubsection{تراجع في عدد الوظائف التقليدية}

أدى إغلاق العديد من وكالات الأسفار وتقليص حجم العمالة في الوكالات المتبقية إلى فقدان أعداد كبيرة من وظائف مستشاري السفر التقليديين. ففي الولايات المتحدة وحدها، انخفض عدد العاملين في وكالات الأسفار من أكثر من 124,000 موظف في عام 2000 إلى نحو 65,000 في عام 2022.

\subsubsection{تغير في المهارات المطلوبة}

تغيرت المهارات المطلوبة للعمل في قطاع السفر بشكل جوهري. فبينما كانت المهارات التقليدية (معرفة الوجهات، إجادة أنظمة الحجز) كافية في السابق، أصبح المطلوب اليوم مجموعة أوسع من المهارات تشمل:

\begin{itemize}[label=\textbf{--}]
\item مهارات رقمية متقدمة (التسويق الرقمي، إدارة المواقع، تحليل البيانات).
\item مهارات استشارية عالية المستوى لتقديم قيمة مضافة لا توفرها المنصات الإلكترونية.
\item مهارات تواصل متميزة وقدرة على بناء علاقات طويلة الأمد مع العملاء.
\item معرفة عميقة بأسواق وقطاعات متخصصة.
\end{itemize}

\subsubsection{ظهور أدوار ووظائف جديدة}

في المقابل، أوجد التحول الرقمي أنواعاً جديدة من الوظائف في قطاع السفر، مثل:

\begin{itemize}[label=\textbf{--}]
\item مديرو المحتوى الرقمي السياحي.
\item متخصصو تجربة المستخدم في منصات السفر.
\item محللو بيانات السفر والسياحة.
\item مديرو التسويق الرقمي السياحي.
\item متخصصو تحسين محركات البحث في قطاع السفر.
\end{itemize}


\subsection{تأثير جائحة كوفيد-19}

شكّلت جائحة كوفيد-19 التي اندلعت في أوائل عام 2020 نقطة تحول حاسمة في العلاقة بين وكالات الأسفار والمنصات الإلكترونية. فقد كان لها تأثيرات مزدوجة ومتناقضة في بعض جوانبها:

\subsubsection{التأثير السلبي على وكالات الأسفار}

\begin{itemize}[label=\textbf{--}]
\item توقف شبه كامل لنشاط السفر لأشهر طويلة.
\item إغلاق عدد كبير من الوكالات بشكل نهائي بسبب عدم القدرة على تحمل التكاليف.
\item فقدان أعداد كبيرة من الموظفين ذوي الخبرة الذين انتقلوا إلى قطاعات أخرى.
\item تسريع التحول نحو الحجز الإلكتروني لدى شرائح جديدة من المسافرين.
\end{itemize}

\subsubsection{الفرص التي أتاحتها الجائحة}

في المقابل، أظهرت الجائحة قيمة وكالات الأسفار في بعض الجوانب:

\begin{itemize}[label=\textbf{--}]
\item أدركت شريحة من المسافرين أهمية التعامل مع وكالة سفر عند حدوث أزمات (إلغاء رحلات، استرداد أموال).
\item أثبتت الوكالات قدرتها على التعامل مع المواقف المعقدة وتقديم الدعم الشخصي في أوقات الأزمات.
\item ازداد الوعي بأهمية التأمين على السفر والحماية من المخاطر.
\item برز دور الوكالات في تقديم معلومات محدّثة ودقيقة حول متطلبات السفر المتغيرة (فحوصات، تطعيمات، حجر صحي).
\end{itemize}

وتشير الدراسات إلى أن شريحة من المسافرين عادت إلى التعامل مع وكالات الأسفار بعد تجارب سلبية مع المنصات الإلكترونية خلال فترة الجائحة، لا سيما فيما يتعلق بصعوبة التواصل مع خدمة العملاء واسترداد الأموال \parencite{wttc2023}.


\subsection{التأثير على سلسلة القيمة في صناعة السفر}

أحدثت المنصات الإلكترونية تحولاً جذرياً في سلسلة القيمة (Value Chain) في صناعة السفر والسياحة. ولفهم عمق هذا التأثير، يمكن تحليل التغيرات التي طرأت على كل حلقة في هذه السلسلة.

\subsubsection{إعادة هيكلة قنوات التوزيع}

كانت سلسلة التوزيع التقليدية في صناعة السفر تسير وفق مسار خطي واضح: مقدم الخدمة (شركة طيران أو فندق) $\rightarrow$ منظم الرحلات $\rightarrow$ وكالة التجزئة $\rightarrow$ المسافر. غير أن التحول الرقمي أدى إلى ظهور نماذج توزيع جديدة ومتعددة، مما أفقد الوكالات التقليدية موقعها المحوري في السلسلة \parencite{kracht2010}.

فاليوم، يمكن للمسافر الحجز من خلال عدة قنوات:

\begin{enumerate}[label=\textbf{\arabic*.}]
\item مباشرة من مقدم الخدمة عبر موقعه الإلكتروني.
\item عبر منصة إلكترونية (OTA) مثل بوكينغ أو إكسبيديا.
\item عبر محرك بحث سفر (Meta-search) ثم التوجه لأحد المصادر.
\item عبر وسائل التواصل الاجتماعي التي أصبحت تتيح الحجز المباشر.
\item عبر تطبيقات الهاتف المحمول المتخصصة.
\item عبر المساعدين الرقميين الصوتيين.
\item والخيار التقليدي: عبر وكالة أسفار مادية.
\end{enumerate}

وهذا التعدد في قنوات التوزيع يعني أن وكالة الأسفار لم تعد القناة الوحيدة أو حتى القناة الرئيسية للوصول إلى خدمات السفر.

\subsubsection{ظاهرة إلغاء الوساطة (Disintermediation)}

من أبرز التأثيرات الهيكلية للتحول الرقمي في صناعة السفر ظاهرة ``إلغاء الوساطة'' (Disintermediation)، أي الاستغناء عن الوسطاء التقليديين. وقد تجلت هذه الظاهرة في عدة مظاهر:

\begin{itemize}[label=\textbf{--}]
\item لجوء شركات الطيران إلى البيع المباشر عبر مواقعها الإلكترونية، حيث أصبحت الحجوزات المباشرة تمثل أكثر من 50\% من مبيعات بعض شركات الطيران.
\item توجه سلاسل الفنادق الكبرى نحو تشجيع الحجز المباشر من خلال تقديم أسعار وضمانات حصرية على مواقعها.
\item يقوم المسافرون أنفسهم بدور ``مستشار السفر'' من خلال البحث والتخطيط الذاتي مستفيدين من الأدوات المتاحة مجاناً على الإنترنت.
\end{itemize}

\subsubsection{ظاهرة إعادة الوساطة (Reintermediation)}

في المقابل، أدى التحول الرقمي أيضاً إلى ظاهرة ``إعادة الوساطة'' عبر ظهور وسطاء رقميين جدد (المنصات الإلكترونية). فبدلاً من اختفاء الوساطة بالكامل، تم استبدال الوسيط التقليدي (وكالة الأسفار) بوسيط رقمي (منصة إلكترونية) في كثير من الحالات. وبذلك لم تختفِ الوساطة بل تغيّر شكلها وطبيعتها.

ويطرح هذا التحول سؤالاً جوهرياً: هل ستنجح وكالات الأسفار التقليدية في إعادة تعريف دورها في سلسلة القيمة الجديدة، أم ستُزاح نهائياً لصالح الوسطاء الرقميين؟


\subsection{تحليل معمّق لسلوك المستهلك السياحي في العصر الرقمي}

يُعدّ فهم سلوك المستهلك السياحي في العصر الرقمي أمراً ضرورياً لتحليل ديناميكيات المنافسة في قطاع السفر. وقد شهد هذا السلوك تحولات عميقة يمكن تحليلها من عدة زوايا.

\subsubsection{نموذج رحلة العميل الرقمي (Digital Customer Journey)}

يمر المسافر الرقمي المعاصر بعدة مراحل قبل وأثناء وبعد الرحلة، تختلف عن المسار التقليدي الذي كان يمر حتماً عبر وكالة الأسفار:

\textbf{المرحلة الأولى - الحلم والإلهام:} يبدأ المسافر بالتفكير في وجهة الرحلة متأثراً بعدة مصادر: منشورات الأصدقاء على وسائل التواصل الاجتماعي (45\% من المسافرين)، محتوى المؤثرين على إنستغرام ويوتيوب (25\%)، مقالات المدونات السياحية (15\%)، الإعلانات الرقمية (10\%)، ومصادر أخرى (5\%).

\textbf{المرحلة الثانية - البحث والتخطيط:} يجري المسافر بحثاً مكثفاً عبر الإنترنت يشمل: محركات البحث (غوغل)، مواقع التقييم (تريب أدفايزر)، المنصات الإلكترونية (بوكينغ، إكسبيديا)، ومنتديات السفر. وتشير الدراسات إلى أن المسافر المتوسط يزور 38 موقعاً إلكترونياً مختلفاً قبل إتمام حجزه \parencite{google2022}.

\textbf{المرحلة الثالثة - المقارنة والقرار:} يقارن المسافر بين الخيارات المتاحة من حيث السعر والجودة والملاءمة. وتلعب محركات البحث عن السفر دوراً حاسماً في هذه المرحلة.

\textbf{المرحلة الرابعة - الحجز والدفع:} يتم الحجز عادة عبر القناة الأرخص أو الأكثر موثوقية من وجهة نظر المسافر.

\textbf{المرحلة الخامسة - التجربة والمشاركة:} يشارك المسافر تجربته عبر التقييمات والصور والمنشورات على وسائل التواصل الاجتماعي، مما يؤثر على قرارات مسافرين آخرين.

\subsubsection{عوامل اختيار قناة الحجز}

تتعدد العوامل التي تؤثر في اختيار المسافر لقناة الحجز (وكالة تقليدية أو منصة إلكترونية أو حجز مباشر)، ويمكن تصنيفها كالتالي:

\begin{table}[H]
\centering
\caption{عوامل اختيار قناة الحجز ودرجة تأثيرها}
\label{tab:booking_factors}
\begin{tabular}{|r|r|r|}
\hline
\textbf{العامل} & \textbf{يفضل المنصة} & \textbf{يفضل الوكالة} \\
\hline
السعر & 78\% & 22\% \\
\hline
السرعة والسهولة & 85\% & 15\% \\
\hline
التوفر الزمني & 90\% & 10\% \\
\hline
الثقة والأمان & 45\% & 55\% \\
\hline
تعقيد الرحلة & 30\% & 70\% \\
\hline
الرحلات الجماعية & 25\% & 75\% \\
\hline
الاستشارة المتخصصة & 20\% & 80\% \\
\hline
خدمة ما بعد البيع & 35\% & 65\% \\
\hline
\end{tabular}
\end{table}

\subsubsection{الفجوة بين الأجيال في سلوك الحجز}

يتباين سلوك الحجز بشكل كبير بين الأجيال المختلفة. ويمكن رصد أنماط واضحة تعكس هذا التباين:

\textbf{الجيل الصامت وجيل الطفرة (فوق 60 عاماً):} يميل هذا الجيل بشكل أكبر إلى التعامل مع وكالات الأسفار التقليدية، حيث يقدّر العلاقة الشخصية والاستشارة المباشرة. نسبة استخدام المنصات الإلكترونية لا تتجاوز 25\%.

\textbf{جيل إكس (40-60 عاماً):} يستخدم مزيجاً من القنوات التقليدية والرقمية. يلجأ إلى المنصات الإلكترونية للرحلات البسيطة وإلى الوكالات للرحلات المعقدة. نسبة الحجز الإلكتروني حوالي 55\%.

\textbf{جيل الألفية (25-40 عاماً):} يفضل بشكل واضح الحجز عبر الإنترنت ويستخدم الهاتف المحمول كأداة رئيسية. نسبة الحجز الإلكتروني تتجاوز 80\%.

\textbf{جيل زد (أقل من 25 عاماً):} شبه كامل الاعتماد على الأدوات الرقمية. يتأثر بشكل كبير بوسائل التواصل الاجتماعي. نسبة الحجز الإلكتروني تتجاوز 90\%.


\subsection{السيناريوهات المستقبلية للعلاقة بين الوكالات والمنصات}

بناءً على التحليل السابق، يمكن رسم عدة سيناريوهات مستقبلية محتملة:

\subsubsection{السيناريو الأول: الزوال التدريجي}

في هذا السيناريو، تستمر وكالات الأسفار التقليدية في التراجع حتى تختفي بشكل كامل تقريباً، كما حدث مع العديد من القطاعات الأخرى التي عصفت بها الرقمنة (مثل متاجر الموسيقى التقليدية ومكاتب الصرافة). ويفترض هذا السيناريو استمرار التطور التكنولوجي بنفس الوتيرة وعدم قدرة الوكالات على التكيف.

\subsubsection{السيناريو الثاني: التحول والتكيف}

في هذا السيناريو، تنجح شريحة من وكالات الأسفار في التحول الرقمي والتكيف مع البيئة الجديدة من خلال تبني استراتيجيات التمايز والتخصص. وتبقى الوكالات فاعلة في أسواق محددة (السفر الفاخر، سياحة الأعمال، الرحلات المعقدة)، لكن حصتها الإجمالية من السوق تبقى محدودة.

\subsubsection{السيناريو الثالث: النموذج الهجين}

يُعتبر هذا السيناريو الأكثر ترجيحاً. حيث تتطور وكالات الأسفار الناجحة نحو نموذج هجين يجمع بين الخدمة الشخصية والتواجد الرقمي القوي. وتصبح الوكالة بمثابة ``مستشار سفر متعدد القنوات'' يتفاعل مع العميل عبر المكتب الفعلي والموقع الإلكتروني والتطبيق ووسائل التواصل الاجتماعي حسب تفضيلاته.


\section*{خلاصة الفصل الثاني}

يتضح من خلال ما تم عرضه في هذا الفصل أن المنافسة بين وكالات الأسفار والمنصات الإلكترونية هي منافسة متعددة الأبعاد وغير متكافئة في كثير من الجوانب. فالمنصات الإلكترونية تتفوق بوضوح في الأبعاد التكنولوجية والتسويقية والسعرية والجغرافية، بينما تحتفظ وكالات الأسفار بتفوق في الخدمة الشخصية والمعرفة المتخصصة والثقة المحلية.

وقد أدى التحول الرقمي إلى إعادة هيكلة سلسلة القيمة في صناعة السفر، مع ظاهرتي إلغاء الوساطة وإعادة الوساطة بأشكال جديدة. كما تغيّر سلوك المستهلك السياحي بشكل جذري، مع فجوة واضحة بين الأجيال في أنماط البحث والحجز.

وبشكل عام، أدت هذه المنافسة إلى تأثيرات عميقة على وكالات الأسفار تمثلت في: تراجع حصتها السوقية، وانهيار نظام العمولات، وانخفاض هوامش الربح، وتغير سلوك المستهلك، وتراجع عدد الوكالات والوظائف في القطاع. كما أسهمت جائحة كوفيد-19 في تسريع هذه التحولات مع إتاحة بعض الفرص لإعادة إبراز قيمة الخدمة الشخصية.

وفي الفصل الثالث، سيتم التركيز على التحديات الرئيسية التي تواجه وكالات الأسفار والحلول والاستراتيجيات المقترحة للتكيف مع البيئة التنافسية الجديدة.


% الفصل الثالث
% ==============================================================================
% الفصل الثالث: التحديات والحلول المقترحة لوكالات السفر
% ==============================================================================
\chapter{التحديات والحلول المقترحة لوكالات السفر}

\section*{تمهيد}

في ظل التحولات العميقة التي يشهدها قطاع السياحة والسفر نتيجة الثورة الرقمية وهيمنة المنصات الإلكترونية، تواجه وكالات الأسفار التقليدية مجموعة واسعة من التحديات التي تهدد وجودها واستمراريتها. غير أن هذه التحديات لا تعني بالضرورة نهاية وكالات الأسفار، بل يمكن أن تشكل فرصة للتحول والتطور إذا ما تم التعامل معها بشكل استراتيجي ومدروس.

يتناول هذا الفصل التحديات الرئيسية التي تواجه وكالات الأسفار في العصر الرقمي (المبحث الأول)، ثم يعرض مجموعة من الحلول والاستراتيجيات المقترحة للتطوير والتكيف (المبحث الثاني).


% ======================================================================
% المبحث الأول: التحديات التي تواجه وكالات السفر
% ======================================================================
\section{المبحث الأول: التحديات التي تواجه وكالات السفر}

تتنوع التحديات التي تواجه وكالات الأسفار في العصر الرقمي وتتشابك فيما بينها، مما يجعل مواجهتها أكثر تعقيداً. ويمكن تصنيف هذه التحديات في عدة فئات رئيسية:

\subsection{التحديات التكنولوجية}

\subsubsection{الفجوة الرقمية}

تُعدّ الفجوة الرقمية من أخطر التحديات التي تواجه وكالات الأسفار التقليدية. فبينما تستثمر المنصات الإلكترونية مليارات الدولارات سنوياً في التكنولوجيا والابتكار، تعاني معظم وكالات الأسفار من تأخر كبير في تبني التقنيات الحديثة. ويتجلى هذا التأخر في عدة مظاهر:

\begin{enumerate}[label=\textbf{\arabic*.}]
\item \textbf{غياب التواجد الرقمي الفعّال:} لا تزال نسبة كبيرة من وكالات الأسفار في العالم العربي لا تمتلك مواقع إلكترونية احترافية أو تطبيقات للهاتف المحمول. وحتى تلك التي تمتلك مواقع إلكترونية، فإن مستوى هذه المواقع من حيث التصميم والوظائف والتجربة يبقى بعيداً عن المعايير الدولية.

\item \textbf{عدم تبني أنظمة الحجز المتقدمة:} لا تزال بعض الوكالات تعتمد على أساليب حجز تقليدية وبطيئة، ولا تستفيد بالكامل من إمكانات أنظمة الحجز العالمية الحديثة.

\item \textbf{غياب أنظمة إدارة علاقات العملاء (CRM):} لا تمتلك معظم الوكالات الصغيرة والمتوسطة أنظمة رقمية لإدارة علاقات العملاء وتتبع تفضيلاتهم وسجل تعاملاتهم.

\item \textbf{ضعف استخدام تحليل البيانات:} لا تستفيد الوكالات من البيانات المتاحة لديها في فهم سلوك العملاء وتحسين الخدمات وتوجيه القرارات التسويقية.
\end{enumerate}

وتعود أسباب هذه الفجوة الرقمية إلى عدة عوامل منها: محدودية الموارد المالية المتاحة للاستثمار في التكنولوجيا، ونقص الكفاءات التقنية داخل الوكالات، ومقاومة التغيير لدى بعض المسيّرين التقليديين، وغياب الوعي بأهمية التحول الرقمي \parencite{alhammad2021}.

\subsubsection{سرعة التطور التكنولوجي}

تتطور التكنولوجيا بوتيرة متسارعة يصعب على وكالات الأسفار مواكبتها. فكل عام يشهد ظهور تقنيات وأدوات جديدة تُعيد تشكيل طريقة تقديم خدمات السفر. ومن أبرز التطورات التكنولوجية التي تشكل تحدياً للوكالات:

\begin{itemize}[label=\textbf{--}]
\item \textbf{الذكاء الاصطناعي:} أصبح الذكاء الاصطناعي يقدم خدمات استشارية وتوصيات مخصصة قد تُغني عن الحاجة إلى مستشار سفر بشري في كثير من الحالات.
\item \textbf{المساعدون الرقميون الصوتيون:} أصبح بإمكان المسافرين البحث عن رحلات وحجزها عبر المساعدين الصوتيين مثل أليكسا وغوغل أسيستنت.
\item \textbf{تقنية البلوك تشين:} قد تؤدي هذه التقنية إلى إلغاء الحاجة إلى الوسطاء في معاملات السفر من خلال العقود الذكية.
\item \textbf{الواقع الافتراضي والمعزز:} تتيح هذه التقنيات للمسافرين استكشاف الوجهات والفنادق افتراضياً قبل الحجز.
\end{itemize}

\subsubsection{تحديات الأمن السيبراني}

مع تزايد التوجه نحو الرقمنة، تواجه وكالات الأسفار تحديات متنامية في مجال الأمن السيبراني. فالتعامل مع بيانات العملاء الحساسة (معلومات جوازات السفر، بيانات البطاقات المصرفية) يتطلب مستويات عالية من الحماية والأمان التي قد لا تكون متوفرة لدى الوكالات الصغيرة والمتوسطة.


\subsection{التحديات الاقتصادية والمالية}

\subsubsection{تراجع مصادر الدخل}

كما تم تفصيله في الفصل السابق، أدى إلغاء العمولات من شركات الطيران وتراجع حجم المبيعات إلى انخفاض كبير في إيرادات وكالات الأسفار. وتُعدّ مشكلة تراجع مصادر الدخل من أكثر التحديات إلحاحاً، حيث تهدد الاستدامة المالية للعديد من الوكالات.

ويمكن تلخيص أبرز مظاهر هذا التحدي:

\begin{itemize}[label=\textbf{--}]
\item إلغاء أو تخفيض عمولات شركات الطيران.
\item المنافسة السعرية الشديدة التي تحدّ من إمكانية فرض رسوم خدمة مرتفعة.
\item تحوّل حصة متزايدة من الحجوزات نحو المنصات الإلكترونية.
\item تراجع الطلب على بعض الخدمات التقليدية مثل إصدار التذاكر.
\end{itemize}

\subsubsection{ارتفاع تكاليف التشغيل}

تواجه وكالات الأسفار ارتفاعاً مستمراً في تكاليف التشغيل يشمل:

\begin{itemize}[label=\textbf{--}]
\item \textbf{تكاليف المقر:} إيجارات المحلات التجارية في المواقع الاستراتيجية في تزايد مستمر.
\item \textbf{تكاليف العمالة:} رواتب الموظفين المؤهلين والتكوين المستمر.
\item \textbf{تكاليف التكنولوجيا:} اشتراكات أنظمة الحجز العالمية ورسوم الترخيص والصيانة.
\item \textbf{تكاليف التسويق:} الحاجة المتزايدة للاستثمار في التسويق الرقمي للحفاظ على الظهور.
\item \textbf{تكاليف الامتثال التنظيمي:} متطلبات الترخيص والضمانات المالية والتأمين.
\end{itemize}

\subsubsection{صعوبة الحصول على التمويل}

تواجه وكالات الأسفار، خاصة الصغيرة والمتوسطة منها، صعوبات في الحصول على التمويل اللازم للتطوير والتحديث. فالمؤسسات المالية تنظر بحذر إلى قطاع وكالات الأسفار بسبب التحديات التي يواجهها والمخاطر المرتبطة به. كما أن عدم امتلاك معظم الوكالات لأصول مادية كبيرة يُضعف قدرتها على تقديم ضمانات للحصول على قروض.


\subsection{التحديات المرتبطة بسلوك المستهلك}

\subsubsection{استقلالية المسافر الرقمي}

أصبح المسافر المعاصر أكثر استقلالية ووعياً من أي وقت مضى. فبفضل الإنترنت ووسائل التواصل الاجتماعي، أصبح لديه وصول سهل إلى كم هائل من المعلومات والأدوات التي كانت حكراً في السابق على المتخصصين في صناعة السفر. ويتميز المسافر الرقمي المعاصر بعدة سمات تشكل تحدياً لوكالات الأسفار:

\begin{itemize}[label=\textbf{--}]
\item \textbf{القدرة على البحث الذاتي:} يمتلك المهارات والأدوات اللازمة للبحث عن المعلومات ومقارنة الخيارات بنفسه.
\item \textbf{الثقة في المراجعات الإلكترونية:} يثق بتقييمات المسافرين الآخرين أكثر من توصيات مستشار السفر في بعض الأحيان.
\item \textbf{تفضيل التخصيص:} يرغب في تصميم رحلته بنفسه وفقاً لتفضيلاته الخاصة.
\item \textbf{الحساسية للسعر:} يبحث عن أفضل الأسعار ولا يقبل دفع مبالغ إضافية دون قيمة مضافة واضحة.
\item \textbf{التوقعات العالية:} يتوقع خدمة سريعة وسلسة ومتاحة على مدار الساعة.
\end{itemize}

\subsubsection{تأثير الأجيال الجديدة}

تُشكّل الأجيال الجديدة (جيل الألفية Generation Y وجيل Z) تحدياً خاصاً لوكالات الأسفار التقليدية. فهذه الأجيال نشأت في بيئة رقمية وتعتبر الإنترنت والتكنولوجيا جزءاً طبيعياً من حياتها اليومية. وتتميز هذه الأجيال بعدة خصائص في ما يتعلق بالسفر \parencite{xiang2015}:

\begin{itemize}[label=\textbf{--}]
\item تفضيل الحجز عبر الهاتف المحمول على أي قناة أخرى.
\item البحث عن تجارب سفر فريدة وأصيلة بدلاً من الرحلات التقليدية.
\item التأثر بشكل كبير بوسائل التواصل الاجتماعي والمؤثرين.
\item قلة الولاء للعلامات التجارية والاستعداد لتجربة خيارات جديدة.
\item تفضيل الخدمات الرقمية ذاتية الخدمة على التفاعل البشري المباشر.
\end{itemize}

وبما أن هذه الأجيال تمثل شريحة متنامية من المسافرين (يُشكّل جيل الألفية وجيل Z أكثر من 50\% من المسافرين عالمياً)، فإن عدم قدرة وكالات الأسفار على استقطابها يُنذر بتآكل مستمر في قاعدة عملائها.

\subsubsection{اقتصاد التجربة والمشاركة}

شهد العقد الأخير تحولاً في تفضيلات المستهلكين من ``امتلاك الأشياء'' إلى ``عيش التجارب''. وفي سياق السفر، يتجلى هذا التحول في:

\begin{itemize}[label=\textbf{--}]
\item تفضيل الإقامة في منازل محلية (عبر إير بي إن بي) على الفنادق التقليدية.
\item البحث عن تجارب ثقافية وتفاعلية مع المجتمعات المحلية.
\item الرغبة في المغامرة والاستكشاف بعيداً عن البرامج السياحية المعلّبة.
\item مشاركة التجارب على وسائل التواصل الاجتماعي كجزء أساسي من تجربة السفر.
\end{itemize}

وهذا التحول يطرح تساؤلات حول مدى ملاءمة النموذج التقليدي لوكالات الأسفار (القائم على الباقات الجاهزة والبرامج المحددة) لتوقعات الجيل الجديد من المسافرين.


\subsection{التحديات التنظيمية والبيئية}

\subsubsection{عدم تكافؤ الإطار التنظيمي}

من المفارقات أن وكالات الأسفار التقليدية تخضع لإطار تنظيمي صارم يتضمن شروط ترخيص وضمانات مالية والتزامات قانونية، بينما تعمل العديد من المنصات الإلكترونية في فراغ تنظيمي نسبي، خاصة عبر الحدود. وهذا التفاوت يضع الوكالات في وضع تنافسي غير عادل.

فعلى سبيل المثال، تُلزم التشريعات في كثير من الدول وكالات الأسفار بتقديم ضمانات مالية لحماية أموال العملاء، وبالتأمين على المسؤولية المهنية، وبالالتزام بمعايير صارمة في تقديم الخدمات. بينما لا تخضع المنصات الإلكترونية العابرة للحدود لنفس المتطلبات في كثير من الأحيان.

\subsubsection{ضعف الدعم المؤسسي}

تعاني وكالات الأسفار في العديد من الدول، خاصة في العالم العربي، من ضعف الدعم المؤسسي الموجه خصيصاً لمساعدتها على التكيف مع التحول الرقمي. فالسياسات الحكومية والبرامج الداعمة لا تأخذ بالحسبان دائماً التحديات الخاصة بهذا القطاع \parencite{albalushi2019}.

\subsubsection{المنافسة من داخل القطاع}

لا تقتصر التحديات على المنافسة مع المنصات الإلكترونية، بل تواجه الوكالات أيضاً منافسة داخلية متزايدة:

\begin{itemize}[label=\textbf{--}]
\item البيع المباشر من شركات الطيران والفنادق عبر مواقعها الإلكترونية.
\item دخول لاعبين جدد من خارج القطاع (شركات تكنولوجيا، بنوك، شركات اتصالات) إلى سوق خدمات السفر.
\item المنافسة غير المشروعة من وكالات غير مرخصة تعمل عبر وسائل التواصل الاجتماعي.
\end{itemize}

\subsection{ملخص التحديات}

يوضح الجدول \ref{tab:challenges} ملخصاً لأبرز التحديات التي تواجه وكالات الأسفار ودرجة تأثيرها:

\begin{table}[H]
\centering
\caption{ملخص التحديات الرئيسية ودرجة تأثيرها}
\label{tab:challenges}
\begin{tabular}{|r|r|r|}
\hline
\textbf{الفئة} & \textbf{التحدي} & \textbf{درجة التأثير} \\
\hline
\multirow{3}{*}{تكنولوجية} & الفجوة الرقمية & مرتفعة جداً \\
\cline{2-3}
 & سرعة التطور التكنولوجي & مرتفعة \\
\cline{2-3}
 & الأمن السيبراني & متوسطة \\
\hline
\multirow{3}{*}{اقتصادية} & تراجع مصادر الدخل & مرتفعة جداً \\
\cline{2-3}
 & ارتفاع تكاليف التشغيل & مرتفعة \\
\cline{2-3}
 & صعوبة التمويل & مرتفعة \\
\hline
\multirow{3}{*}{سلوك المستهلك} & استقلالية المسافر الرقمي & مرتفعة جداً \\
\cline{2-3}
 & تأثير الأجيال الجديدة & مرتفعة \\
\cline{2-3}
 & اقتصاد التجربة & متوسطة \\
\hline
\multirow{2}{*}{تنظيمية} & عدم تكافؤ التنظيم & متوسطة \\
\cline{2-3}
 & ضعف الدعم المؤسسي & متوسطة \\
\hline
\end{tabular}
\end{table}


% ======================================================================
% المبحث الثاني: حلول واستراتيجيات التطوير
% ======================================================================
\section{المبحث الثاني: حلول واستراتيجيات التطوير}

في مواجهة التحديات المتعددة التي تم استعراضها في المبحث السابق، تحتاج وكالات الأسفار إلى تبني حزمة متكاملة من الحلول والاستراتيجيات التي تمكنها من التكيف مع البيئة الرقمية الجديدة والحفاظ على تنافسيتها. ويمكن تصنيف هذه الحلول في عدة محاور استراتيجية:

\subsection{استراتيجية التحول الرقمي}

يُعدّ التحول الرقمي الاستراتيجية الأكثر إلحاحاً وأهمية لوكالات الأسفار في العصر الحالي. ولا يعني التحول الرقمي مجرد إنشاء موقع إلكتروني أو صفحة على فيسبوك، بل يتطلب إعادة التفكير الجذري في نموذج العمل بأكمله وتوظيف التكنولوجيا في جميع جوانب النشاط.

\subsubsection{بناء منصة رقمية متكاملة}

يجب على وكالات الأسفار الاستثمار في بناء تواجد رقمي قوي يشمل:

\textbf{أ. موقع إلكتروني احترافي:}

يجب أن يتضمن الموقع الإلكتروني للوكالة الحد الأدنى من المكونات التالية:

\begin{itemize}[label=\textbf{--}]
\item تصميم عصري ومتجاوب يعمل بشكل مثالي على جميع الأجهزة (حاسوب، هاتف، جهاز لوحي).
\item نظام بحث وحجز متكامل يتيح للعملاء البحث عن الخدمات وحجزها إلكترونياً.
\item محتوى غني ومحدّث عن الوجهات والعروض والخدمات.
\item نظام دفع إلكتروني آمن يدعم وسائل الدفع المختلفة.
\item قسم للمدونة والمحتوى السياحي لتحسين الظهور في محركات البحث.
\item نظام دردشة مباشرة للتواصل الفوري مع العملاء.
\end{itemize}

\textbf{ب. تطبيق للهاتف المحمول:}

نظراً لأن أكثر من 70\% من عمليات البحث عن السفر تتم عبر الأجهزة المحمولة، يُعدّ امتلاك تطبيق متطور للهاتف المحمول أمراً ضرورياً. ويجب أن يوفر التطبيق تجربة مستخدم سلسة وسريعة مع إمكانات الحجز والدفع والإشعارات الفورية.

\textbf{ج. التكامل مع أنظمة الحجز العالمية:}

يجب على الوكالات الارتباط بأنظمة الحجز العالمية (GDS) المتطورة واستخدام واجهات برمجة التطبيقات (APIs) للحصول على بيانات فورية عن الرحلات والفنادق والأسعار وعرضها على منصتها الرقمية.

\subsubsection{تبني أنظمة إدارة علاقات العملاء (CRM)}

يُعدّ نظام إدارة علاقات العملاء أداة أساسية لأي وكالة سفر تسعى للتنافس في العصر الرقمي. ويتيح هذا النظام:

\begin{itemize}[label=\textbf{--}]
\item تسجيل وتتبع جميع تفاعلات العملاء مع الوكالة.
\item بناء قاعدة بيانات شاملة عن تفضيلات العملاء وسجل رحلاتهم.
\item تخصيص العروض والتوصيات بناءً على بيانات العميل.
\item أتمتة عمليات المتابعة والتسويق (إرسال رسائل تذكير، عروض مناسبات، استطلاعات رضا).
\item تحليل سلوك العملاء وتحديد الفرص التجارية.
\item تحسين خدمة ما بعد البيع من خلال تتبع الشكاوى والملاحظات.
\end{itemize}

\subsubsection{الاستفادة من الذكاء الاصطناعي}

يمكن لوكالات الأسفار، حتى الصغيرة منها، الاستفادة من أدوات الذكاء الاصطناعي المتاحة في السوق لتحسين خدماتها:

\begin{itemize}[label=\textbf{--}]
\item استخدام روبوتات المحادثة (Chatbots) للرد على استفسارات العملاء الشائعة على مدار الساعة.
\item توظيف أدوات تحليل البيانات لفهم اتجاهات السوق وسلوك العملاء.
\item استخدام أنظمة التوصيات الذكية لاقتراح رحلات مخصصة للعملاء.
\item الاستفادة من أدوات التسعير الديناميكي لتحسين استراتيجية التسعير.
\end{itemize}


\subsection{استراتيجية التمايز والتخصص}

في ظل المنافسة الشديدة مع المنصات الإلكترونية، يُعدّ التمايز والتخصص من أنجح الاستراتيجيات التي يمكن أن تتبناها وكالات الأسفار. فبدلاً من محاولة منافسة المنصات العملاقة في مجالات تتفوق فيها (السعر، الحجم، التكنولوجيا)، يمكن للوكالات التركيز على مجالات تتمتع فيها بميزة تنافسية حقيقية.

\subsubsection{التخصص في أسواق محددة}

يمكن لوكالات الأسفار أن تتخصص في أسواق أو شرائح محددة تتطلب معرفة عميقة وخدمة شخصية لا تستطيع المنصات الإلكترونية توفيرها بنفس المستوى. ومن أمثلة مجالات التخصص \parencite{christensen2016}:

\begin{itemize}[label=\textbf{--}]
\item \textbf{السياحة الفاخرة:} سوق يُقدَّر بأكثر من 900 مليار دولار عالمياً، يبحث عملاؤه عن تجارب استثنائية وخدمة شخصية لا تتوفر عبر المنصات الإلكترونية.
\item \textbf{سياحة المغامرات والاستكشاف:} تتطلب معرفة متخصصة وتنسيقاً عالياً لا يمكن أتمتته بالكامل.
\item \textbf{السياحة الصحية والعلاجية:} تحتاج إلى تنسيق مع مؤسسات طبية وترتيبات خاصة.
\item \textbf{سياحة الأعمال والمؤتمرات:} تتطلب مرونة عالية وتنسيقاً مع أطراف متعددة.
\item \textbf{السياحة الدينية (الحج والعمرة):} تحتاج إلى خبرة خاصة وترتيبات محددة.
\item \textbf{سياحة الزفاف وشهر العسل:} يبحث عملاؤها عن تنظيم مثالي وتجربة لا تُنسى.
\item \textbf{السفر لذوي الاحتياجات الخاصة:} يتطلب ترتيبات وتنسيقاً خاصاً.
\end{itemize}

\subsubsection{تقديم تجارب فريدة}

يمكن لوكالات الأسفار التمايز من خلال تقديم تجارب سفر فريدة لا تتوفر على المنصات الإلكترونية. وتشمل هذه التجارب:

\begin{itemize}[label=\textbf{--}]
\item تصميم رحلات حصرية تضم أنشطة وزيارات غير متاحة للعموم.
\item توفير مرشدين سياحيين محليين متخصصين ذوي معرفة عميقة.
\item تنظيم تجارب ثقافية وتفاعلية مع المجتمعات المحلية.
\item تقديم باقات موضوعية (رحلات طعام، رحلات تصوير، جولات تاريخية متعمقة).
\item بناء شراكات حصرية مع مقدمي خدمات استثنائيين في الوجهات المختلفة.
\end{itemize}

\subsubsection{بناء العلامة التجارية الشخصية}

في عالم رقمي يزداد تجرداً من الطابع الإنساني، يمكن لوكالات الأسفار بناء علامة تجارية قوية تقوم على:

\begin{itemize}[label=\textbf{--}]
\item الخبرة والمصداقية في مجال تخصصها.
\item القصص والتجارب الشخصية لمستشاري السفر.
\item العلاقات الإنسانية والتواصل الشخصي مع العملاء.
\item القيم والمبادئ (الاستدامة، المسؤولية الاجتماعية، الأصالة).
\end{itemize}


\subsection{استراتيجية التسويق الرقمي}

يُعدّ التسويق الرقمي ركيزة أساسية لأي استراتيجية تطوير لوكالات الأسفار في العصر الحالي. ويجب أن تتضمن استراتيجية التسويق الرقمي عدة محاور:

\subsubsection{تحسين محركات البحث (SEO)}

يجب على وكالات الأسفار الاستثمار في تحسين ظهور مواقعها في نتائج محركات البحث، وذلك من خلال:

\begin{itemize}[label=\textbf{--}]
\item إنتاج محتوى عالي الجودة ومفيد حول الوجهات والنصائح السياحية.
\item تحسين الهيكل التقني للموقع (سرعة التحميل، التوافق مع الأجهزة المحمولة).
\item بناء روابط خلفية من مواقع ذات سلطة في مجال السياحة.
\item استهداف كلمات مفتاحية طويلة ومتخصصة (Long-tail Keywords) بدلاً من المنافسة على الكلمات العامة التي تهيمن عليها المنصات الكبرى.
\end{itemize}

\subsubsection{التسويق عبر وسائل التواصل الاجتماعي}

تمثل وسائل التواصل الاجتماعي فرصة كبيرة لوكالات الأسفار للتواصل مع العملاء الحاليين والمحتملين:

\begin{itemize}[label=\textbf{--}]
\item \textbf{إنستغرام:} منصة مثالية لمشاركة صور وفيديوهات ملهمة عن الوجهات والرحلات.
\item \textbf{فيسبوك:} مناسبة لبناء مجتمع وتبادل التجارب ونشر العروض.
\item \textbf{يوتيوب:} إنتاج محتوى فيديو عن الوجهات ونصائح السفر.
\item \textbf{تيك توك:} استهداف الأجيال الشابة بمحتوى قصير وجذاب.
\item \textbf{لينكد إن:} للتسويق في قطاع سياحة الأعمال.
\end{itemize}

\subsubsection{التسويق بالمحتوى}

يُعدّ التسويق بالمحتوى من أكثر استراتيجيات التسويق الرقمي فعالية لوكالات الأسفار. ويتضمن إنتاج محتوى ذي قيمة يجذب الجمهور المستهدف ويبني الثقة ويُرسّخ مكانة الوكالة كمرجع في مجالها. ويمكن أن يشمل هذا المحتوى:

\begin{itemize}[label=\textbf{--}]
\item مقالات ودلائل سفر مفصلة عن الوجهات.
\item نصائح وإرشادات عملية للمسافرين.
\item قصص وتجارب سفر ملهمة.
\item فيديوهات وبودكاست عن عالم السفر.
\item رسائل إخبارية دورية بعروض ومعلومات مفيدة.
\end{itemize}

\subsubsection{التسويق عبر البريد الإلكتروني}

يظل البريد الإلكتروني من أكثر أدوات التسويق فعالية من حيث العائد على الاستثمار. ويمكن لوكالات الأسفار استخدامه بفعالية من خلال:

\begin{itemize}[label=\textbf{--}]
\item بناء قائمة بريدية مؤهلة من العملاء الحاليين والمحتملين.
\item إرسال عروض مخصصة بناءً على تفضيلات كل عميل.
\item رسائل تذكير بمواسم السفر والعطل والمناسبات.
\item نشرات إخبارية دورية بمحتوى مفيد وملهم.
\item حملات استعادة العملاء غير النشطين.
\end{itemize}


\subsection{استراتيجية تحسين تجربة العميل}

في عالم أصبحت فيه تجربة العميل عاملاً حاسماً في النجاح التنافسي، يجب على وكالات الأسفار الارتقاء بمستوى الخدمة المقدمة إلى أعلى المعايير.

\subsubsection{تطوير الخدمة الاستشارية}

بدلاً من الاكتفاء بدور الوسيط في بيع الخدمات، يجب على وكالات الأسفار التحول نحو نموذج ``مستشار السفر'' الذي يقدم قيمة مضافة حقيقية للعميل:

\begin{itemize}[label=\textbf{--}]
\item تقديم استشارات معمّقة مبنية على خبرة شخصية ومعرفة متخصصة.
\item بناء علاقات طويلة الأمد مع العملاء قائمة على الثقة والمصداقية.
\item توفير خدمة شاملة تغطي جميع جوانب الرحلة من التخطيط إلى العودة.
\item المتابعة المستمرة مع العميل قبل وأثناء وبعد الرحلة.
\end{itemize}

\subsubsection{تحسين خدمة ما بعد البيع}

تمثل خدمة ما بعد البيع فرصة كبيرة للتمايز عن المنصات الإلكترونية:

\begin{itemize}[label=\textbf{--}]
\item توفير خط تواصل مباشر مع العميل أثناء رحلته.
\item التدخل الفوري لحل أي مشاكل أو طوارئ.
\item متابعة العميل بعد عودته للاطمئنان على رضاه.
\item جمع الملاحظات والاستفادة منها في تحسين الخدمات.
\end{itemize}

\subsubsection{إنشاء برامج ولاء العملاء}

يمكن لوكالات الأسفار إنشاء برامج ولاء تكافئ العملاء المتكررين وتشجعهم على البقاء:

\begin{itemize}[label=\textbf{--}]
\item نظام نقاط يتم استبدالها بخصومات أو خدمات مجانية.
\item عروض حصرية للعملاء الدائمين.
\item معاملة تفضيلية في أوقات الذروة.
\item هدايا ومفاجآت في المناسبات الخاصة (أعياد ميلاد، ذكرى سنوية).
\end{itemize}


\subsection{استراتيجية الشراكات والتحالفات}

في مواجهة المنصات الإلكترونية العملاقة، قد يكون التعاون بين وكالات الأسفار وتشكيل تحالفات استراتيجية أحد الحلول الفعالة.

\subsubsection{الشراكات بين الوكالات}

يمكن لوكالات الأسفار الصغيرة والمتوسطة تشكيل تحالفات وشبكات تعاونية للاستفادة من:

\begin{itemize}[label=\textbf{--}]
\item القوة التفاوضية المشتركة مع مقدمي الخدمات.
\item مشاركة تكاليف التكنولوجيا والتسويق.
\item تبادل الخبرات وأفضل الممارسات.
\item توسيع نطاق الخدمات الجغرافي.
\item بناء منصة إلكترونية مشتركة.
\end{itemize}

\subsubsection{الشراكات مع مقدمي الخدمات}

يمكن لوكالات الأسفار بناء شراكات استراتيجية مع مقدمي خدمات محددين للحصول على:

\begin{itemize}[label=\textbf{--}]
\item أسعار حصرية وشروط تفضيلية.
\item منتجات وتجارب غير متاحة عبر المنصات الإلكترونية.
\item دعم تسويقي ومادي مشترك.
\item برامج تدريب وتأهيل للموظفين.
\end{itemize}

\subsubsection{التعاون مع المنصات الإلكترونية}

بدلاً من اعتبار المنصات الإلكترونية عدواً يجب محاربته، يمكن النظر إليها كشريك محتمل:

\begin{itemize}[label=\textbf{--}]
\item استخدام المنصات كقناة إضافية للتوزيع.
\item الاستفادة من الأدوات التكنولوجية التي توفرها المنصات للوكالات.
\item التركيز على الخدمات ذات القيمة المضافة التي لا تقدمها المنصات.
\item تبني نموذج هجين يجمع بين التواجد الفعلي والرقمي.
\end{itemize}


\subsection{استراتيجية تطوير الموارد البشرية}

يُعدّ العنصر البشري الأصل الأهم لدى وكالات الأسفار وأكبر ميزة تنافسية لها مقارنة بالمنصات الإلكترونية. لذلك يجب الاستثمار في تطوير الكفاءات البشرية.

\subsubsection{التكوين والتدريب المستمر}

يجب على وكالات الأسفار الاستثمار في تطوير مهارات موظفيها في مجالات:

\begin{itemize}[label=\textbf{--}]
\item المهارات الرقمية (التسويق الإلكتروني، استخدام الأدوات التكنولوجية).
\item مهارات الاستشارة والبيع المتقدم.
\item المعرفة المتعمقة بالوجهات والمنتجات السياحية.
\item مهارات التواصل وإدارة علاقات العملاء.
\item اللغات الأجنبية والثقافات المختلفة.
\end{itemize}

\subsubsection{استقطاب الكفاءات الجديدة}

في إطار التحول الرقمي، تحتاج وكالات الأسفار إلى استقطاب كفاءات جديدة في مجالات التكنولوجيا والتسويق الرقمي وتحليل البيانات، بالإضافة إلى المهارات التقليدية في مجال السفر والسياحة.

\subsubsection{تحسين بيئة العمل}

لجذب الكفاءات والاحتفاظ بها، يجب على وكالات الأسفار تحسين بيئة العمل من خلال:

\begin{itemize}[label=\textbf{--}]
\item تقديم رواتب ومزايا تنافسية.
\item توفير فرص التطور الوظيفي والتكوين المستمر.
\item إتاحة رحلات تعريفية للموظفين لزيارة الوجهات وتقييم الفنادق.
\item بناء ثقافة مؤسسية إيجابية تشجع الابتكار والمبادرة.
\end{itemize}


\subsection{استراتيجية التنويع وتوسيع مصادر الدخل}

لتعزيز استدامتها المالية، يجب على وكالات الأسفار تنويع مصادر إيراداتها:

\subsubsection{خدمات ذات قيمة مضافة عالية}

\begin{itemize}[label=\textbf{--}]
\item الاستشارات المدفوعة (رسوم تخطيط رحلات مخصصة).
\item خدمات الكونسيرج (Concierge Services) للمسافرين.
\item تنظيم فعاليات وأنشطة حصرية في الوجهات.
\item خدمات إدارة السفر المتكاملة للشركات.
\end{itemize}

\subsubsection{مصادر دخل جديدة}

\begin{itemize}[label=\textbf{--}]
\item بيع التأمينات والخدمات التكميلية بهوامش ربح أعلى.
\item تحصيل رسوم عضوية لبرامج الولاء المميزة.
\item تقديم خدمات التأشيرات والوثائق بأسعار تنافسية.
\item إنتاج محتوى سياحي مموّل أو مدفوع.
\item تقديم خدمات التدريب والاستشارات لوكالات أخرى.
\end{itemize}

ويلخص الجدول \ref{fig:strategies} الاستراتيجيات المقترحة لتطوير وكالات الأسفار:

\begin{table}[H]
\centering
\caption{الاستراتيجيات المقترحة لتطوير وكالات الأسفار}
\label{fig:strategies}
\begin{tabular}{|r|r|}
\hline
\textbf{الاستراتيجية} & \textbf{الهدف الرئيسي} \\
\hline
\textbf{1. التحول الرقمي} & بناء تواجد رقمي قوي ومنافس \\
\hline
\textbf{2. التمايز والتخصص} & التركيز على أسواق دقيقة ذات قيمة مضافة عالية \\
\hline
\textbf{3. التسويق الرقمي} & الوصول إلى العملاء المحتملين عبر القنوات الرقمية \\
\hline
\textbf{4. تحسين تجربة العميل} & تقديم خدمة شخصية متفوقة ومتكاملة \\
\hline
\textbf{5. الشراكات والتحالفات} & تعزيز القدرة التنافسية من خلال التعاون \\
\hline
\textbf{6. تطوير الموارد البشرية} & بناء كفاءات رقمية واستشارية متقدمة \\
\hline
\textbf{7. تنويع مصادر الدخل} & تقليل الاعتماد على العمولات التقليدية \\
\hline
\end{tabular}
\end{table}

ويوضح الشكل \ref{fig:framework} إطار العمل المقترح للتحول الاستراتيجي لوكالات الأسفار من خلال ست مراحل متتابعة ومتكاملة، مع وجود حلقة تغذية راجعة تضمن التحسين المستمر:

\begin{figure}[H]
\centering
\begin{tikzpicture}[
    phase/.style={rectangle, draw=blue!50, rounded corners=5pt,
                  text width=9cm, minimum height=1cm, align=center, font=\small\bfseries},
    arr/.style={->, thick, >=stealth, draw=blue!40}
]
\node[phase, fill=blue!15] (p1) at (0,0) {المرحلة 1: التشخيص والتقييم (1--3 أشهر)};
\node[phase, fill=blue!12] (p2) at (0,-1.7) {المرحلة 2: بناء الرؤية والاستراتيجية (1--2 شهر)};
\node[phase, fill=green!12] (p3) at (0,-3.4) {المرحلة 3: البناء التكنولوجي (3--6 أشهر)};
\node[phase, fill=green!10] (p4) at (0,-5.1) {المرحلة 4: تطوير الكفاءات (مستمرة)};
\node[phase, fill=orange!15] (p5) at (0,-6.8) {المرحلة 5: الإطلاق والتنفيذ (مستمرة)};
\node[phase, fill=orange!10] (p6) at (0,-8.5) {المرحلة 6: التقييم والتحسين المستمر (دورية)};
\draw[arr] (p1) -- (p2);
\draw[arr] (p2) -- (p3);
\draw[arr] (p3) -- (p4);
\draw[arr] (p4) -- (p5);
\draw[arr] (p5) -- (p6);
\draw[arr, dashed, draw=red!50] (p6.east) -- ++(1.5,0) |- (p1.east);
\end{tikzpicture}
\caption{إطار العمل المقترح للتحول الاستراتيجي لوكالات الأسفار (6 مراحل)}
\label{fig:framework}
\end{figure}


\subsection{نماذج ناجحة في التكيف}

من المفيد الاطلاع على تجارب بعض وكالات الأسفار التي نجحت في التكيف مع التحول الرقمي والحفاظ على تنافسيتها:

\subsubsection{نموذج الوكالة الهجينة}

نجحت بعض الوكالات في الجمع بين التواجد المادي والرقمي بشكل متكامل. حيث يستخدم العميل الموقع الإلكتروني أو التطبيق للبحث الأولي والاطلاع على العروض، ثم يتواصل مع مستشار سفر متخصص (حضورياً أو عبر الفيديو) للحصول على استشارة شخصية ومتعمقة. ويتم إتمام الحجز إلكترونياً مع إمكانية الدعم البشري في أي مرحلة.

\subsubsection{نموذج التخصص العميق}

اختارت بعض الوكالات التخصص في أسواق دقيقة وأصبحت مرجعاً في مجالها. فوكالات متخصصة في سياحة المغامرات أو السياحة الفاخرة أو رحلات المجموعات استطاعت تحقيق نمو مستمر رغم التحديات، بفضل المعرفة العميقة والعلاقات القوية مع الموردين والقدرة على تقديم تجارب فريدة.

\subsubsection{نموذج مستشار السفر المستقل}

ظهر نموذج جديد يتمثل في مستشاري السفر المستقلين الذين يعملون من المنزل تحت مظلة شبكة وكالات أكبر. يتيح هذا النموذج مرونة عالية وتكاليف تشغيلية منخفضة مع الاستفادة من البنية التحتية والدعم الذي توفره الشبكة الأم.

\subsubsection{نموذج الوكالة المتخصصة في السوق المحلي}

في بعض الأسواق العربية والإفريقية، نجحت وكالات أسفار في بناء موقع تنافسي قوي من خلال التركيز على فهم السياق المحلي والثقافي. فهذه الوكالات تستفيد من معرفتها العميقة بالعادات والتقاليد المحلية، ومن علاقاتها القوية مع الجهات الحكومية والقطاع الخاص المحلي، لتقديم خدمات لا تستطيع المنصات الإلكترونية مجاراتها فيها. ومن أبرز الأمثلة: وكالات الحج والعمرة في المغرب العربي التي تقدم خدمات شاملة تراعي الخصوصية الثقافية والدينية للمسافرين.


\subsection{إطار عمل مقترح للتحول الاستراتيجي}

بناءً على ما تم عرضه من تحديات واستراتيجيات، يمكن اقتراح إطار عمل متكامل للتحول الاستراتيجي لوكالات الأسفار التقليدية، يتضمن ست مراحل رئيسية:

\subsubsection{المرحلة الأولى: التشخيص والتقييم (1-3 أشهر)}

تبدأ عملية التحول بإجراء تشخيص شامل للوضع الحالي للوكالة يشمل:

\begin{enumerate}[label=\textbf{\arabic*.}]
\item تحليل SWOT لتحديد نقاط القوة والضعف والفرص والتهديدات.
\item تقييم الجاهزية الرقمية للوكالة من حيث البنية التحتية والكفاءات والثقافة المؤسسية.
\item تحليل قاعدة العملاء الحالية وتصنيفها حسب الربحية والولاء والإمكانات.
\item دراسة البيئة التنافسية المحلية وتحديد الفجوات السوقية.
\item تقييم الموارد المالية والبشرية المتاحة للتحول.
\end{enumerate}

\subsubsection{المرحلة الثانية: بناء الرؤية والاستراتيجية (1-2 شهر)}

\begin{enumerate}[label=\textbf{\arabic*.}]
\item تحديد رؤية واضحة للوكالة في أفق 3-5 سنوات.
\item اختيار القطاعات السوقية المستهدفة والتخصصات المراد تطويرها.
\item تحديد عرض القيمة الفريد (Unique Value Proposition) الذي يميز الوكالة.
\item وضع أهداف كمية ونوعية قابلة للقياس والمتابعة.
\item إعداد ميزانية التحول وتحديد مصادر التمويل.
\end{enumerate}

\subsubsection{المرحلة الثالثة: البناء التكنولوجي (3-6 أشهر)}

\begin{enumerate}[label=\textbf{\arabic*.}]
\item تطوير أو تحديث الموقع الإلكتروني ليكون متجاوباً ومحسّناً لمحركات البحث.
\item تبني نظام إدارة علاقات العملاء (CRM) مناسب لحجم الوكالة.
\item ربط الأنظمة الداخلية بأنظمة الحجز العالمية (GDS) وواجهات برمجة التطبيقات (APIs).
\item إنشاء حضور فعال على وسائل التواصل الاجتماعي.
\item تأمين البنية التحتية الرقمية وحماية بيانات العملاء.
\end{enumerate}

\subsubsection{المرحلة الرابعة: تطوير الكفاءات (مستمرة)}

\begin{enumerate}[label=\textbf{\arabic*.}]
\item تدريب فريق العمل على الأدوات الرقمية الجديدة.
\item تطوير مهارات الاستشارة المتخصصة والبيع الاحترافي.
\item استقطاب كفاءات في التسويق الرقمي والتكنولوجيا إذا لزم الأمر.
\item بناء ثقافة مؤسسية تشجع على التعلم المستمر والابتكار.
\end{enumerate}

\subsubsection{المرحلة الخامسة: الإطلاق والتنفيذ (مستمرة)}

\begin{enumerate}[label=\textbf{\arabic*.}]
\item إطلاق الحملات التسويقية الرقمية المستهدفة.
\item تفعيل قنوات التواصل الجديدة مع العملاء.
\item تطبيق المنتجات والخدمات الجديدة بشكل تدريجي.
\item مراقبة الأداء وقياس المؤشرات الرئيسية بشكل مستمر.
\end{enumerate}

\subsubsection{المرحلة السادسة: التقييم والتحسين المستمر (دورية)}

\begin{enumerate}[label=\textbf{\arabic*.}]
\item مراجعة دورية (شهرية وربع سنوية) لمؤشرات الأداء.
\item جمع ملاحظات العملاء واستغلالها في التحسين المستمر.
\item متابعة التطورات التكنولوجية والاتجاهات الجديدة في السوق.
\item تعديل الاستراتيجية بناءً على النتائج والمستجدات.
\end{enumerate}


\subsection{قياس الأداء ومؤشرات النجاح}

لضمان نجاح عملية التحول وتقييم فعالية الاستراتيجيات المتبناة، يجب على وكالات الأسفار وضع مجموعة شاملة من مؤشرات الأداء الرئيسية (KPIs) يمكن تصنيفها كما يلي:

\begin{table}[H]
\centering
\caption{مؤشرات الأداء الرئيسية لقياس نجاح الاستراتيجيات}
\label{tab:kpis}
\begin{tabular}{|r|r|r|}
\hline
\textbf{المجال} & \textbf{المؤشر} & \textbf{الهدف} \\
\hline
المبيعات & إجمالي الإيرادات & نمو سنوي 10\%+ \\
\hline
المبيعات & هامش الربح لكل معاملة & تحسن بنسبة 15\% \\
\hline
الرقمنة & نسبة الحجوزات عبر الإنترنت & أكثر من 40\% \\
\hline
الرقمنة & عدد زوار الموقع الإلكتروني & نمو شهري 5\% \\
\hline
العملاء & معدل الاحتفاظ بالعملاء & أكثر من 60\% \\
\hline
العملاء & مؤشر صافي الترويج (NPS) & أكثر من 70 \\
\hline
التسويق & تكلفة اكتساب العميل & انخفاض بنسبة 20\% \\
\hline
التسويق & معدل التحويل الإلكتروني & أكثر من 3\% \\
\hline
\end{tabular}
\end{table}


\subsection{دور الدولة والمؤسسات في دعم التحول}

لا يمكن لوكالات الأسفار مواجهة التحديات الرقمية بمفردها، بل تحتاج إلى منظومة دعم متكاملة من الدولة والمؤسسات المختصة. ويمكن أن يتجلى هذا الدعم في عدة أشكال:

\subsubsection{الدعم التشريعي والتنظيمي}

\begin{itemize}[label=\textbf{--}]
\item سنّ قوانين تضمن المنافسة العادلة بين الوكالات التقليدية والمنصات الإلكترونية، خاصة في مجال الالتزامات الضريبية والتأمين.
\item فرض شروط شفافية على المنصات الإلكترونية العاملة في السوق المحلي.
\item تيسير الإجراءات الإدارية المتعلقة بإنشاء وتشغيل وكالات الأسفار.
\item حماية بيانات المسافرين وفرض معايير أمنية موحدة على جميع القنوات.
\end{itemize}

\subsubsection{الدعم المالي والتقني}

\begin{itemize}[label=\textbf{--}]
\item تقديم حوافز ضريبية وقروض ميسّرة للوكالات التي تستثمر في التحول الرقمي.
\item إنشاء منصات حكومية تقنية يمكن لوكالات الأسفار الاستفادة منها.
\item دعم برامج التكوين والتدريب في المجال الرقمي.
\item تمويل مشاريع البحث والتطوير في مجال السياحة الرقمية.
\end{itemize}

\subsubsection{الدور الجمعوي والمهني}

يمكن للجمعيات والاتحادات المهنية لوكالات الأسفار أن تلعب دوراً محورياً من خلال:

\begin{itemize}[label=\textbf{--}]
\item الدفاع عن مصالح القطاع أمام الجهات التشريعية والتنظيمية.
\item تنظيم برامج تكوين مشتركة تفيد جميع الأعضاء.
\item إنشاء منصات تكنولوجية مشتركة بتكاليف مقسّمة.
\item تبادل أفضل الممارسات والدراسات والمعارف بين الأعضاء.
\item تمثيل القطاع في المحافل الدولية والمعارض السياحية.
\end{itemize}


\subsection{السياحة المستدامة كفرصة استراتيجية}

تُشكّل التوجهات العالمية نحو الاستدامة فرصة استراتيجية لوكالات الأسفار لبناء ميزة تنافسية مستدامة. فبينما تركز معظم المنصات الإلكترونية على حجم المبيعات والأسعار المنخفضة، يمكن لوكالات الأسفار أن تتبنى نهجاً مسؤولاً يلبي الطلب المتزايد على السياحة المستدامة.

\subsubsection{البعد البيئي}

\begin{itemize}[label=\textbf{--}]
\item ترويج خيارات السفر ذات البصمة الكربونية المنخفضة.
\item اقتراح وجهات وأنشطة صديقة للبيئة.
\item التعاون مع فنادق ومقدمي خدمات ملتزمين بمعايير الاستدامة البيئية.
\item تقديم خدمة حساب وتعويض البصمة الكربونية للرحلات.
\end{itemize}

\subsubsection{البعد الاجتماعي والثقافي}

\begin{itemize}[label=\textbf{--}]
\item تصميم رحلات تدعم المجتمعات المحلية في الوجهات السياحية.
\item تعزيز السياحة الثقافية الأصيلة بعيداً عن النمطية.
\item المساهمة في الحفاظ على التراث المادي وغير المادي في الوجهات.
\item ضمان عدالة توزيع العوائد السياحية على المجتمعات المحلية.
\end{itemize}

\subsubsection{البعد الاقتصادي}

\begin{itemize}[label=\textbf{--}]
\item دعم المؤسسات الصغيرة والمتوسطة المحلية في الوجهات السياحية.
\item تشجيع السياحة الموزعة زمنياً وجغرافياً لتجنب الضغط على المناطق المكتظة.
\item المساهمة في خلق فرص عمل محلية مستدامة.
\end{itemize}

وتشير الدراسات إلى أن 73\% من المسافرين يعبّرون عن رغبتهم في السفر بشكل أكثر استدامة، كما أن 61\% منهم أكدوا أن جائحة كوفيد-19 زادت من اهتمامهم بالسياحة المستدامة \parencite{booking2022}. وهذا يمثل فرصة حقيقية لوكالات الأسفار التي تستطيع تقديم خيارات سفر مسؤولة ومدروسة، مقارنة بالمنصات التي تركز بشكل أساسي على السعر والحجم.


\section*{خلاصة الفصل الثالث}

يتضح من خلال ما تم عرضه في هذا الفصل أن التحديات التي تواجه وكالات الأسفار متعددة ومتشابكة، تشمل الأبعاد التكنولوجية والاقتصادية والمرتبطة بسلوك المستهلك والتنظيمية. غير أن هذه التحديات ليست قدراً محتوماً، بل يمكن مواجهتها من خلال حزمة متكاملة من الاستراتيجيات التي تتمحور حول: التحول الرقمي، والتمايز والتخصص، والتسويق الرقمي الفعال، وتحسين تجربة العميل، وبناء الشراكات والتحالفات، وتطوير الموارد البشرية، وتنويع مصادر الدخل.

وقد تم اقتراح إطار عمل متكامل من ست مراحل لتنفيذ التحول الاستراتيجي، يبدأ بالتشخيص والتقييم ويمر ببناء الرؤية والبناء التكنولوجي وتطوير الكفاءات والإطلاق، وينتهي بالتقييم والتحسين المستمر. كما تم التأكيد على أهمية وضع مؤشرات أداء رئيسية لقياس نجاح كل استراتيجية ومتابعة التقدم.

ولا يمكن إغفال دور الدولة والمؤسسات المهنية في دعم هذا التحول من خلال الإطار التشريعي والتنظيمي والدعم المالي والتقني. كما تمثل التوجهات العالمية نحو السياحة المستدامة فرصة استراتيجية لوكالات الأسفار لبناء ميزة تنافسية مستدامة تلبي الطلب المتزايد على السفر المسؤول.

والمفتاح الأساسي لنجاح أي استراتيجية هو القدرة على الجمع بين ما تتميز به وكالات الأسفار (الخدمة الشخصية، الخبرة، الثقة) وبين ما تفرضه البيئة الرقمية الجديدة من متطلبات (التواجد الرقمي، السرعة، الشفافية، التكنولوجيا). فالوكالات التي ستنجح في المستقبل هي تلك التي ستتمكن من تحقيق هذه المعادلة بين الإنساني والرقمي.


% الخاتمة العامة
% ==============================================================================
% الخاتمة العامة
% ==============================================================================
\chapter*{الخاتمة العامة}
\addcontentsline{toc}{chapter}{الخاتمة العامة}
\markboth{الخاتمة العامة}{الخاتمة العامة}

\vspace{1cm}

تناولت هذه الدراسة موضوعاً بالغ الأهمية والراهنية يتعلق بالتحديات التي تواجهها وكالات الأسفار التقليدية في ظل المنافسة المتزايدة مع المنصات الإلكترونية العالمية لحجز السفر. وقد سعت الدراسة إلى الإجابة على الإشكالية الرئيسية المتمثلة في: ما هي أبرز التحديات التي تواجهها وكالات الأسفار في ظل المنافسة المتزايدة مع المنصات الإلكترونية، وما هي الحلول والاستراتيجيات الممكنة للتكيف مع هذا الواقع الجديد؟

\vspace{0.8cm}

\section*{ملخص نتائج الدراسة}
\addcontentsline{toc}{section}{ملخص نتائج الدراسة}

من خلال الفصول الثلاثة التي تشكّلت منها هذه الدراسة، تم التوصل إلى مجموعة من النتائج الرئيسية:

\begin{enumerate}[label=\textbf{\arabic*.}]
\item \textbf{في ما يتعلق بالإطار النظري (الفصل الأول):} تبيّن أن وكالات الأسفار مرت بمراحل تطور عديدة منذ تأسيسها في القرن التاسع عشر، وقد عرفت عصرها الذهبي في النصف الثاني من القرن العشرين حين كانت تمثل الحلقة الأساسية في سلسلة التوزيع السياحي. كما اتضح أن المنصات الإلكترونية تتنوع في أنواعها ونماذج أعمالها، وتتمتع بمزايا تنافسية عديدة أبرزها: التوفر الدائم، والأسعار التنافسية، وسهولة الاستخدام، ووفرة المعلومات والتقييمات، واستخدام التكنولوجيا المتقدمة.

\item \textbf{في ما يتعلق بطبيعة المنافسة وتأثيراتها (الفصل الثاني):} تبيّن أن المنافسة بين وكالات الأسفار والمنصات الإلكترونية هي منافسة متعددة الأبعاد (تكنولوجي، تسويقي، سعري، جغرافي)، وغير متكافئة لصالح المنصات الإلكترونية في معظم الأبعاد. وقد أدت هذه المنافسة إلى: تراجع الحصة السوقية لوكالات الأسفار (من أكثر من 60\% إلى أقل من 15\% من إجمالي الحجوزات)، وانهيار نظام العمولات التقليدي، وانخفاض هوامش الربح، وتراجع عدد الوكالات والوظائف في القطاع.

\item \textbf{في ما يتعلق بالتحديات والحلول (الفصل الثالث):} تم تحديد مجموعة من التحديات الرئيسية تصنّف في أربع فئات: تكنولوجية (الفجوة الرقمية، سرعة التطور)، اقتصادية (تراجع الإيرادات، ارتفاع التكاليف)، مرتبطة بسلوك المستهلك (استقلالية المسافر الرقمي، تأثير الأجيال الجديدة)، وتنظيمية (عدم تكافؤ الإطار التنظيمي). كما تم اقتراح حزمة من الاستراتيجيات للتطوير تشمل: التحول الرقمي، والتمايز والتخصص، والتسويق الرقمي، وتحسين تجربة العميل، والشراكات والتحالفات، وتطوير الموارد البشرية، وتنويع مصادر الدخل.
\end{enumerate}

\vspace{0.8cm}

\section*{مناقشة الفرضيات}
\addcontentsline{toc}{section}{مناقشة الفرضيات}

في ضوء النتائج المتوصل إليها، يمكن مناقشة فرضيات الدراسة على النحو التالي:

\textbf{الفرضية الأولى:} ``تسبب المنصات الإلكترونية العالمية في انخفاض نسبة الحجوزات لدى وكالات الأسفار''.

\textbf{تم تأكيد هذه الفرضية.} فقد أظهرت البيانات والإحصائيات تراجعاً مستمراً في حصة وكالات الأسفار من إجمالي حجوزات السفر العالمية لصالح المنصات الإلكترونية. حيث انخفضت حصة الوكالات التقليدية من أكثر من 60\% في بداية الألفية إلى أقل من 15\% حالياً في بعض الأسواق. وتعود أسباب هذا الانخفاض إلى المزايا التنافسية التي تتمتع بها المنصات الإلكترونية من حيث السعر والسهولة والتوفر.

\textbf{الفرضية الثانية:} ``تواجه وكالات الأسفار تحديات أساسية بسبب ضعف الرقمنة مقارنة بالمنصات الإلكترونية''.

\textbf{تم تأكيد هذه الفرضية.} فقد اتضح أن الفجوة الرقمية تُعدّ من أخطر التحديات التي تواجه وكالات الأسفار. حيث أن غياب التواجد الرقمي الفعال وعدم تبني التقنيات الحديثة يُضعف قدرة الوكالات على المنافسة في سوق أصبح رقمياً بالدرجة الأولى. وتتجلى هذه الفجوة في مظاهر عديدة من أبرزها: غياب المواقع الإلكترونية الاحترافية، وعدم استخدام أنظمة إدارة العملاء، وضعف الحضور على وسائل التواصل الاجتماعي.

\textbf{الفرضية الثالثة:} ``أدى التغيير في سلوك المستهلك نحو الإلكتروني إلى تراجع الإقبال على وكالات الأسفار''.

\textbf{تم تأكيد هذه الفرضية.} فقد تبيّن أن تحول سلوك المستهلك السياحي نحو الرقمنة يُعدّ من العوامل الرئيسية وراء تراجع إقبال المسافرين على وكالات الأسفار التقليدية. فالمسافر المعاصر أصبح أكثر استقلالية ووعياً رقمياً، ويفضل التخطيط لرحلاته بنفسه عبر الإنترنت. كما أن تأثير الأجيال الجديدة (جيل الألفية وجيل Z)، التي تشكل الشريحة الأكبر والمتنامية من المسافرين، يُسهم في تعميق هذا التوجه.

\vspace{0.8cm}

\section*{التوصيات}
\addcontentsline{toc}{section}{التوصيات}

بناءً على النتائج المتوصل إليها، يمكن تقديم التوصيات التالية:

\subsection*{توصيات موجهة لوكالات الأسفار}

\begin{enumerate}[label=\textbf{\arabic*.}]
\item \textbf{الإسراع في التحول الرقمي:} يجب على وكالات الأسفار اعتبار التحول الرقمي أولوية قصوى، وتخصيص الموارد اللازمة لبناء تواجد رقمي قوي ومتكامل (موقع إلكتروني، تطبيق، وسائل تواصل اجتماعي).

\item \textbf{التخصص والتمايز:} بدلاً من محاولة منافسة المنصات العملاقة في جميع المجالات، يُنصح بالتركيز على أسواق وشرائح محددة يمكن تقديم قيمة مضافة حقيقية فيها.

\item \textbf{الاستثمار في الكفاءات البشرية:} تطوير مهارات الموظفين واستقطاب كفاءات جديدة في المجالات الرقمية والتسويقية.

\item \textbf{بناء شراكات وتحالفات:} التعاون مع وكالات أخرى ومقدمي خدمات لتعزيز القدرة التنافسية.

\item \textbf{تنويع مصادر الدخل:} عدم الاعتماد على مصدر دخل واحد وتطوير خدمات ذات قيمة مضافة عالية.
\end{enumerate}

\subsection*{توصيات موجهة للجهات الحكومية والتنظيمية}

\begin{enumerate}[label=\textbf{\arabic*.}]
\item \textbf{تحديث الإطار التنظيمي:} مراجعة وتحديث التشريعات المنظمة لقطاع وكالات الأسفار لتتواءم مع الواقع الرقمي الجديد.

\item \textbf{ضمان المنافسة العادلة:} إخضاع المنصات الإلكترونية العاملة في السوق المحلي لنفس المتطلبات التنظيمية المفروضة على الوكالات التقليدية.

\item \textbf{تقديم برامج دعم:} إطلاق برامج لدعم التحول الرقمي لوكالات الأسفار الصغيرة والمتوسطة (تمويل، تكوين، استشارات).

\item \textbf{تشجيع التكوين والتأهيل:} دعم برامج التكوين المهني في المجالات الرقمية لفائدة العاملين في قطاع السفر والسياحة.
\end{enumerate}

\vspace{0.8cm}

\section*{آفاق الدراسة}
\addcontentsline{toc}{section}{آفاق الدراسة}

تفتح هذه الدراسة المجال أمام عدة محاور بحثية مستقبلية:

\begin{itemize}[label=\textbf{--}]
\item إجراء دراسة ميدانية حول واقع التحول الرقمي في وكالات الأسفار في الجزائر أو في بلد عربي محدد.
\item دراسة تأثير الذكاء الاصطناعي على مستقبل وكالات الأسفار.
\item تحليل سلوك المستهلك الجزائري تجاه المنصات الإلكترونية لحجز السفر.
\item دراسة مقارنة لاستراتيجيات التكيف المعتمدة من وكالات الأسفار في دول مختلفة.
\item تقييم فعالية برامج التحول الرقمي في قطاع السفر والسياحة.
\item دراسة تأثير تقنيات الويب 3.0 والبلوك تشين على صناعة توزيع السفر.
\end{itemize}

\vspace{1.5cm}

\begin{center}
\textbf{\Large وفي الأخير، نسأل الله التوفيق والسداد، وأن يكون هذا العمل المتواضع إضافة مفيدة للمكتبة العلمية العربية في مجال السياحة والسفر.}
\end{center}


% ==================== قائمة المراجع ====================
% ==============================================================================
% قائمة المراجع
% ==============================================================================
\newpage
\addcontentsline{toc}{chapter}{قائمة المراجع}
\printbibliography[title={قائمة المراجع}]


\end{document}
