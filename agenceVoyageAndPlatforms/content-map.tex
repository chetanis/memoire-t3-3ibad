% ==============================================================================
% خريطة المحتوى: ما يتم تناوله في كل قسم من المذكرة
% ==============================================================================
\documentclass[a4paper, 14pt]{extreport}

\usepackage{fontspec}
\usepackage{polyglossia}
\setmainlanguage[numerals=maghrib]{arabic}
\setotherlanguage{english}

\setmainfont{Amiri}[
  Path = /usr/local/texlive/2025basic/texmf-dist/fonts/truetype/public/amiri/,
  Extension = .ttf,
  UprightFont = *-Regular,
  BoldFont = *-Bold,
  ItalicFont = *-Italic,
  BoldItalicFont = *-BoldItalic,
  Script = Arabic
]
\newfontfamily\arabicfont{Amiri}[
  Path = /usr/local/texlive/2025basic/texmf-dist/fonts/truetype/public/amiri/,
  Extension = .ttf,
  UprightFont = *-Regular,
  BoldFont = *-Bold,
  ItalicFont = *-Italic,
  BoldItalicFont = *-BoldItalic,
  Script = Arabic,
  Scale = 1.1
]
\newfontfamily\arabicfontsf{Amiri}[
  Path = /usr/local/texlive/2025basic/texmf-dist/fonts/truetype/public/amiri/,
  Extension = .ttf,
  UprightFont = *-Regular,
  BoldFont = *-Bold,
  ItalicFont = *-Italic,
  BoldItalicFont = *-BoldItalic,
  Script = Arabic,
  Scale = 1.1
]

\usepackage[top=2.5cm, bottom=2.5cm, right=3cm, left=2cm]{geometry}
\usepackage{setspace}
\onehalfspacing

\usepackage{longtable}
\usepackage{array}
\usepackage{booktabs}
\usepackage{enumitem}
\usepackage{xcolor}
\usepackage{hyperref}

\hypersetup{
  colorlinks=true,
  linkcolor=black,
  citecolor=blue!60!black,
  urlcolor=blue!60!black,
}

\makeatletter
\ifdefined\UseMathForPositioningText\else
  \newcommand{\UseMathForPositioningText}{}
\fi
\ifdefined\@kernel@tabular@init\else
  \newcommand{\@kernel@tabular@init}{}
\fi
\makeatother

\begin{document}

\begin{center}
{\Huge \textbf{خريطة محتويات المذكرة}}\\[0.5cm]
{\Large تحديات وكالات الأسفار في ظل المنافسة مع المنصات الإلكترونية}\\[1cm]
\end{center}

\noindent يقدم هذا الملف ملخصاً تفصيلياً لما يتم تناوله في كل قسم من أقسام المذكرة، مع أرقام الصفحات الفعلية من النسخة المجمّعة.

\vspace{1cm}

% ==============================================================================
\begin{longtable}{|r|p{8cm}|r|}
\hline
\textbf{القسم} & \textbf{المحتوى المتناول} & \textbf{الصفحات} \\
\hline
\endfirsthead
\hline
\textbf{القسم} & \textbf{المحتوى المتناول} & \textbf{الصفحات} \\
\hline
\endhead

\hline
\endfoot

% ========== الصفحات الأولية ==========
\multicolumn{3}{|r|}{\textbf{الصفحات الأولية (الترقيم الروماني)}} \\
\hline

صفحة العنوان &
اسم الجامعة، الكلية، القسم، عنوان المذكرة، اسم الطالب، المشرف، السنة الجامعية
& ص i \\
\hline

الإهداء &
إهداء الباحث لعائلته وأحبائه
& ص ii \\
\hline

الشكر والتقدير &
شكر المشرف والأساتذة والعائلة
& ص iii \\
\hline

الملخص &
ملخص الدراسة بالعربية: الهدف، الفرضيات الثلاث، هيكل الفصول الثلاثة، الكلمات المفتاحية
& ص iv \\
\hline

فهرس المحتويات &
قائمة الفصول والمباحث والأقسام الفرعية
& ص v--vii \\
\hline

قائمة الجداول &
فهرس الجداول الواردة في المذكرة
& ص viii \\
\hline

قائمة الأشكال &
فهرس الأشكال والرسوم البيانية
& ص ix \\
\hline

% ========== المقدمة العامة ==========
\multicolumn{3}{|r|}{\textbf{المقدمة العامة (الصفحات 1--6)}} \\
\hline

تقديم الموضوع &
أهمية قطاع السياحة والسفر عالمياً، الدور التاريخي لوكالات الأسفار كوسيط رئيسي، ظهور المنصات الإلكترونية (بوكينغ، إكسبيديا، سكاي سكانر، إير بي إن بي)، تهديد المنصات لوكالات الأسفار
& ص 1 \\
\hline

إشكالية الدراسة &
صياغة السؤال الرئيسي: ما أبرز التحديات التي تواجه وكالات الأسفار وما الحلول الممكنة؟ + 5 أسئلة فرعية
& ص 2 \\
\hline

فرضيات الدراسة &
الفرضية 1: المنصات تسبب انخفاض الحجوزات. الفرضية 2: ضعف الرقمنة يُضعف التنافسية. الفرضية 3: تغير سلوك المستهلك يسبب تراجع الإقبال
& ص 2 \\
\hline

أهمية الدراسة &
الأهمية العلمية (إثراء الأدبيات العربية)، العملية (رؤى وتوصيات)، الراهنية (موضوع حيوي بعد كورونا)
& ص 3 \\
\hline

أهداف الدراسة &
5 أهداف: التعريف بالمفاهيم، تحليل المنافسة، تحديد التحديات، رصد تغيرات سلوك المستهلك، اقتراح حلول
& ص 3 \\
\hline

منهج الدراسة &
المنهج الوصفي التحليلي، أسباب اختياره، أدوات جمع البيانات (مسح مكتبي، تقارير دولية، دراسات حالة)
& ص 4 \\
\hline

حدود الدراسة &
الحدود الموضوعية والزمنية (2000-2024) والمكانية (عالمي مع إشارات عربية)
& ص 5 \\
\hline

صعوبات الدراسة &
ندرة الدراسات العربية، سرعة التغير، صعوبة الحصول على بيانات عربية
& ص 5 \\
\hline

هيكل الدراسة &
عرض محتوى الفصول الثلاثة بإيجاز
& ص 6 \\
\hline

% ========== الفصل الأول ==========
\multicolumn{3}{|r|}{\textbf{الفصل الأول: الإطار النظري لوكالات السفر والمنصات الإلكترونية (ص 7--31)}} \\
\hline

تمهيد الفصل الأول &
تقديم الفصل وأهدافه والمباحث الثلاثة
& ص 7 \\
\hline

1.1.1 تعريف وكالات الأسفار &
التعريف اللغوي والاصطلاحي (كوبر، منظمة السياحة العالمية، ميدلتون، الشمري)، الخصائص المشتركة، التعريف القانوني (القانون الجزائري)
& ص 7--9 \\
\hline

2.1.1 نشأة وتطور وكالات الأسفار &
الجذور التاريخية (الحضارات القديمة)، مرحلة التأسيس (توماس كوك 1841)، مرحلة النمو (أمريكان إكسبريس)، مرحلة الازدهار (1950-1990: السياحة الجماهيرية، أنظمة CRS)، مرحلة التحول الرقمي + جدول المحطات التاريخية
& ص 9--11 \\
\hline

3.1.1 أنواع وكالات الأسفار &
التصنيف حسب النشاط (تجزئة، منظمو رحلات، جملة)، حسب التخصص (عامة، متخصصة)، حسب نمط التشغيل (مستقلة، سلاسل، امتياز)، حسب الوسيلة (تقليدية، إلكترونية، هجينة) + جدول التصنيفات
& ص 11--14 \\
\hline

1.2.1 خدمات النقل والحجز &
حجز تذاكر الطيران (أنظمة GDS، تاريخ العمولات)، حجز وسائل النقل الأخرى (قطارات، سيارات، رحلات بحرية)
& ص 14--15 \\
\hline

2.2.1 خدمات الإقامة &
أنواع الإقامة (فنادق، منتجعات، شقق)، المعرفة الميدانية للوكالات
& ص 15--16 \\
\hline

3.2.1 تنظيم الرحلات والبرامج السياحية &
الرحلات الشاملة (الباقات السياحية ومزاياها)، الرحلات المصممة حسب الطلب
& ص 16 \\
\hline

4.2.1 الخدمات الإدارية والاستشارية &
التأشيرات والوثائق، التأمين على السفر، الاستشارات والنصائح
& ص 16--17 \\
\hline

5.2.1 خدمات سياحة الأعمال &
إدارة سفرات الأعمال، تنظيم المؤتمرات والمعارض، حوافز الشركات
& ص 17--18 \\
\hline

6.2.1 خدمات الحج والعمرة &
تنظيم رحلات الحج والعمرة في العالم العربي والإسلامي + جدول ملخص الخدمات
& ص 18 \\
\hline

1.3.1 تعريف المنصات الإلكترونية للسفر &
المفهوم (تعريف بوهاليس)، الأنواع: OTAs (بوكينغ، إكسبيديا، تريب دوت كوم)، Meta-search (سكاي سكانر، غوغل فلايتس، تريفاغو، كاياك)، الاقتصاد التشاركي (إير بي إن بي)، منصات التقييم (تريب أدفايزر)
& ص 18--20 \\
\hline

2.3.1 مزايا المنصات الإلكترونية &
سهولة الاستخدام (التوفر 24/7، الوصول من أي مكان)، المعلومات (تقييمات المستخدمين، شفافية الأسعار)، التكلفة (أسعار تنافسية، عروض وخصومات)، التكنولوجيا (التخصيص الذكي، تطبيقات الهاتف) + جدول مقارنة + شكل بياني
& ص 20--23 \\
\hline

3.3.1 نماذج الأعمال للمنصات الإلكترونية &
نموذج العمولة، نموذج التاجر، نموذج الإعلانات، نموذج الاشتراك
& ص 23--24 \\
\hline

4.3.1 الاتجاهات الحديثة في المنصات الإلكترونية &
الذكاء الاصطناعي والروبوتات، الواقع الافتراضي، التخصيص الفائق، الاستدامة، المدفوعات الرقمية
& ص 24--25 \\
\hline

5.3.1 البنية التحتية التكنولوجية للمنصات &
البنية التحتية المتقدمة، الحوسبة السحابية، معالجة البيانات الضخمة
& ص 25--27 \\
\hline

6.3.1 وضع وكالات الأسفار في العالم العربي &
واقع الوكالات في الأسواق العربية، التحديات الخاصة بالمنطقة
& ص 27--29 \\
\hline

7.3.1 الدراسات السابقة المتعلقة بالموضوع &
مراجعة الأدبيات والبحوث السابقة حول وكالات الأسفار والمنصات الإلكترونية
& ص 29--31 \\
\hline

% ========== الفصل الثاني ==========
\multicolumn{3}{|r|}{\textbf{الفصل الثاني: المنافسة بين وكالات الأسفار والمنصات الإلكترونية (ص 32--51)}} \\
\hline

تمهيد الفصل الثاني &
تقديم طبيعة المنافسة غير المتكافئة وأهداف الفصل
& ص 32 \\
\hline

1.1.2 الإطار النظري للمنافسة في قطاع السياحة &
تعريف المنافسة وأنواعها (مباشرة، غير مباشرة، بديلة)، تحليل نموذج بورتر للقوى الخمس مطبقاً على قطاع السفر + جدول + شكل بياني
& ص 32--35 \\
\hline

2.1.2 أبعاد المنافسة بين الوكالات والمنصات &
البعد التكنولوجي (البنية التحتية، الذكاء الاصطناعي، تجربة المستخدم)، البعد التسويقي (ميزانيات التسويق الضخمة، SEO، برامج الولاء)، البعد السعري (وفورات الحجم، التسعير الديناميكي)، البعد الجغرافي (التغطية العالمية)
& ص 35--38 \\
\hline

3.1.2 نقاط قوة وكالات الأسفار في المنافسة &
الخدمة الشخصية والعلاقة الإنسانية، الخبرة والمعرفة المتخصصة، الأمان والثقة، الحماية القانونية، خدمة ما بعد البيع + جدول مقارنة
& ص 38--40 \\
\hline

1.2.2 التأثير على الحصة السوقية &
تراجع حصة الوكالات (من 60\% إلى 15\%)، تراجع عدد الوكالات (أمريكا، بريطانيا، فرنسا، إفلاس توماس كوك 2019)، إعادة التوزيع السوقي + جدول تطور الحصص + شكل بياني
& ص 40--42 \\
\hline

2.2.2 التأثير على نموذج الأعمال &
انهيار نظام العمولات (من 10\% إلى 0\%)، الضغط على هوامش الربح (من 8-10\% إلى 2-4\%)، تغير تركيبة الإيرادات
& ص 42--43 \\
\hline

3.2.2 التأثير على سلوك المستهلك &
تحول أنماط البحث والحجز (5 مراحل)، ظاهرة ROBO (البحث في الوكالة والحجز عبر الإنترنت)، تغير توقعات المستهلك
& ص 43--45 \\
\hline

4.2.2 التأثير على سوق العمل في قطاع السفر &
تراجع الوظائف التقليدية (من 124,000 إلى 65,000 في أمريكا)، تغير المهارات المطلوبة، ظهور أدوار جديدة
& ص 45--46 \\
\hline

5.2.2 تأثير جائحة كوفيد-19 &
التأثير السلبي (توقف النشاط، إغلاق الوكالات)، الفرص (إبراز قيمة الخدمة الشخصية في الأزمات)
& ص 46--47 \\
\hline

6.2.2 التأثير على سلسلة القيمة في صناعة السفر &
إعادة هيكلة قنوات التوزيع (7 قنوات بديلة)، ظاهرة إلغاء الوساطة (Disintermediation)، ظاهرة إعادة الوساطة (Reintermediation)
& ص 47--48 \\
\hline

7.2.2 تحليل معمّق لسلوك المستهلك السياحي الرقمي &
نموذج رحلة العميل الرقمي (5 مراحل)، عوامل اختيار قناة الحجز + جدول، الفجوة بين الأجيال (الصامت، إكس، الألفية، زد)
& ص 48--50 \\
\hline

8.2.2 السيناريوهات المستقبلية &
سيناريو الزوال التدريجي، سيناريو التحول والتكيف، سيناريو النموذج الهجين (الأكثر ترجيحاً) + خلاصة الفصل الثاني
& ص 50--51 \\
\hline

% ========== الفصل الثالث ==========
\multicolumn{3}{|r|}{\textbf{الفصل الثالث: التحديات والحلول المقترحة لوكالات السفر (ص 52--74)}} \\
\hline

تمهيد الفصل الثالث &
تقديم التحديات كفرصة للتحول إذا أُحسن التعامل معها
& ص 52 \\
\hline

1.1.3 التحديات التكنولوجية &
الفجوة الرقمية (غياب التواجد الرقمي، عدم تبني أنظمة CRM)، سرعة التطور التكنولوجي (الذكاء الاصطناعي، المساعدون الصوتيون، البلوك تشين)، تحديات الأمن السيبراني
& ص 52--54 \\
\hline

2.1.3 التحديات الاقتصادية والمالية &
تراجع مصادر الدخل (إلغاء العمولات، المنافسة السعرية)، ارتفاع تكاليف التشغيل، صعوبة الحصول على التمويل
& ص 54--55 \\
\hline

3.1.3 التحديات المرتبطة بسلوك المستهلك &
استقلالية المسافر الرقمي، تأثير الأجيال الجديدة (جيل الألفية وجيل Z: أكثر من 50\% من المسافرين)، اقتصاد التجربة والمشاركة
& ص 55--57 \\
\hline

4.1.3 التحديات التنظيمية والبيئية &
عدم تكافؤ الإطار التنظيمي، ضعف الدعم المؤسسي، المنافسة من داخل القطاع
& ص 57 \\
\hline

5.1.3 ملخص التحديات &
جدول شامل يلخص التحديات الرئيسية ودرجة تأثيرها (مرتفعة جداً / مرتفعة / متوسطة)
& ص 57--58 \\
\hline

1.2.3 استراتيجية التحول الرقمي &
بناء منصة رقمية متكاملة (موقع إلكتروني، تطبيق، تكامل مع GDS)، تبني أنظمة CRM، الاستفادة من الذكاء الاصطناعي
& ص 58--60 \\
\hline

2.2.3 استراتيجية التمايز والتخصص &
التخصص في أسواق محددة (فاخرة، مغامرات، أعمال، حج وعمرة، ذوي الاحتياجات الخاصة)، تقديم تجارب فريدة، بناء العلامة التجارية
& ص 60--62 \\
\hline

3.2.3 استراتيجية التسويق الرقمي &
تحسين محركات البحث SEO، التسويق عبر وسائل التواصل (إنستغرام، فيسبوك، يوتيوب، تيك توك)، التسويق بالمحتوى، التسويق عبر البريد الإلكتروني
& ص 62--63 \\
\hline

4.2.3 استراتيجية تحسين تجربة العميل &
تطوير الخدمة الاستشارية (التحول لنموذج مستشار السفر)، تحسين خدمة ما بعد البيع، إنشاء برامج ولاء
& ص 63--65 \\
\hline

5.2.3 استراتيجية الشراكات والتحالفات &
الشراكات بين الوكالات، مع مقدمي الخدمات، مع المنصات الإلكترونية (تعاون بدل عداء)
& ص 65--66 \\
\hline

6.2.3 استراتيجية تطوير الموارد البشرية &
التكوين والتدريب المستمر، استقطاب الكفاءات الجديدة، تحسين بيئة العمل
& ص 66--67 \\
\hline

7.2.3 استراتيجية التنويع وتوسيع مصادر الدخل &
خدمات ذات قيمة مضافة عالية (استشارات مدفوعة، كونسيرج)، مصادر دخل جديدة + جدول الاستراتيجيات + شكل بياني
& ص 67--68 \\
\hline

8.2.3 نماذج ناجحة في التكيف &
النموذج الهجين، نموذج التخصص العميق، نموذج مستشار السفر المستقل، نموذج الوكالة المتخصصة في السوق المحلي
& ص 68--69 \\
\hline

9.2.3 إطار عمل مقترح للتحول الاستراتيجي &
6 مراحل: التشخيص والتقييم، بناء الرؤية والاستراتيجية، البناء التكنولوجي، تطوير الكفاءات، الإطلاق والتنفيذ، التقييم والتحسين المستمر
& ص 69--71 \\
\hline

10.2.3 قياس الأداء ومؤشرات النجاح &
جدول مؤشرات الأداء الرئيسية KPIs (المبيعات، الرقمنة، العملاء، التسويق)
& ص 71 \\
\hline

11.2.3 دور الدولة والمؤسسات في دعم التحول &
الدعم التشريعي والتنظيمي، الدعم المالي والتقني، الدور الجمعوي والمهني
& ص 72 \\
\hline

12.2.3 السياحة المستدامة كفرصة استراتيجية &
البعد البيئي (البصمة الكربونية)، البعد الاجتماعي والثقافي، البعد الاقتصادي + إحصائيات (73\% يرغبون في سفر مستدام) + خلاصة الفصل الثالث
& ص 73--74 \\
\hline

% ========== الخاتمة ==========
\multicolumn{3}{|r|}{\textbf{الخاتمة العامة (ص 75--78)}} \\
\hline

ملخص نتائج الدراسة &
نتائج كل فصل من الفصول الثلاثة بشكل مركّز
& ص 75--76 \\
\hline

مناقشة الفرضيات &
تأكيد الفرضية الأولى (انخفاض الحجوزات)، تأكيد الفرضية الثانية (ضعف الرقمنة)، تأكيد الفرضية الثالثة (تغير سلوك المستهلك)
& ص 76--77 \\
\hline

التوصيات &
توصيات لوكالات الأسفار (5 توصيات)، توصيات للجهات الحكومية (4 توصيات)
& ص 77--78 \\
\hline

آفاق الدراسة &
6 محاور بحثية مستقبلية مقترحة
& ص 78 \\
\hline

% ========== المراجع ==========
\multicolumn{3}{|r|}{\textbf{قائمة المراجع}} \\
\hline

المراجع &
المراجع العربية والأجنبية المستخدمة في الدراسة
& ص 79 \\
\hline

\end{longtable}

\end{document}
